
\documentclass[letterpaper,hide notes,xcolor={table,svgnames},pdftex]{beamer}
\def\showexamples{t}


%\usepackage[svgnames]{xcolor}

%% Demo talk
%\documentclass[letterpaper,notes=show]{beamer}

\usecolortheme{crane}%seahorse crane
\setbeamertemplate{navigation symbols}{}

\usetheme{MyPittsburgh}
%\usetheme{Frankfurt}

%\usepackage{tipa}

\usepackage{hyperref}
\usepackage{graphicx,xspace}
\usepackage[normalem]{ulem}

\newcommand\SF[1]{$\bigstar$\footnote{SF: #1}}

\usepackage{paratype}
\renewcommand*\familydefault{\sfdefault} %% Only if the base font of the document is to be sans serif
\usepackage[zerostyle=c]{newtxtt}
\usepackage[T1]{fontenc}

\newcounter{tmpnumSlide}
\newcounter{tmpnumNote}

\usepackage{xcolor}
\usepackage{tabu}
\definecolor{light-gray}{gray}{0.75}
\taburulecolor{light-gray}

% old question code
%\newcommand\question[1]{{$\bigstar$ \small \onlySlide{2}{#1}}}
% \newcommand\nquestion[1]{\ifdefined \presentationonly \textcircled{?} \fi \note{\par{\Large \textbf{?}} #1}}
% \newcommand\nanswer[1]{\note{\par{\Large \textbf{A}} #1}}


 \newcommand\mnote[1]{%
   \addtocounter{tmpnumSlide}{1}
   \ifdefined\showcues {~\tiny\fbox{\arabic{tmpnumSlide}}}\fi
   \note{\setlength{\parskip}{1ex}\addtocounter{tmpnumNote}{1}\textbf{\Large \arabic{tmpnumNote}:} {#1\par}}}

\newcommand\mmnote[1]{\note{\setlength{\parskip}{1ex}#1\par}}

%\newcommand\mnote[2][]{\ifdefined\handoutwithnotes {~\tiny\fbox{#1}}\fi
% \note{\setlength{\parskip}{1ex}\textbf{\Large #1:} #2\par}}

%\newcommand\mnote[2][]{{\tiny\fbox{#1}} \note{\setlength{\parskip}{1ex}\textbf{\Large #1:} #2\par}}

\newcommand\mquestion[2]{{~\color{red}\fbox{?}}\note{\setlength{\parskip}{1ex}\par{\Large \textbf{?}} #1} \note{\setlength{\parskip}{1ex}\par{\Large \textbf{A}} #2\par}\ifdefined \presentationonly \pause \fi}

\newcommand\blackboard[1]{%
\ifdefined   \showblackboard
  {#1}
  \else {\begin{center} \fbox{\colorbox{blue!30}{%
         \begin{minipage}{.95\linewidth}%
           \hspace{\stretch{1}} Some space intentionally left blank; done at the blackboard.%
         \end{minipage}}}\end{center}}%
         \fi%
}



%\newcommand\q{\tikz \node[thick,color=black,shape=circle]{?};}
%\newcommand\q{\ifdefined \presentationonly \textcircled{?} \fi}

\usepackage{listings}
\lstset{%
  keywordstyle=\bfseries,
  aboveskip=15pt,
  belowskip=15pt,
  captionpos=b,
  identifierstyle=\ttfamily,
  escapeinside={(*@}{@*)},
  stringstyle=\ttfamiliy,
  frame=lines,
  numbers=left, basicstyle=\scriptsize, numberstyle=\tiny, stepnumber=0, numbersep=2pt}

\usepackage{siunitx}
\newcommand\sius[1]{\num[group-separator = {,}]{#1}\si{\micro\second}}
\newcommand\sims[1]{\num[group-separator = {,}]{#1}\si{\milli\second}}
\newcommand\sins[1]{\num[group-separator = {,}]{#1}\si{\nano\second}}
\sisetup{group-separator = {,}, group-digits = true}

%% -------------------- tikz --------------------
\usepackage{tikz}
\usetikzlibrary{positioning}
\usetikzlibrary{arrows,backgrounds,automata,decorations.shapes,decorations.pathmorphing,decorations.markings,decorations.text}

\tikzstyle{place}=[circle,draw=blue!50,fill=blue!20,thick, inner sep=0pt,minimum size=6mm]
\tikzstyle{transition}=[rectangle,draw=black!50,fill=black!20,thick, inner sep=0pt,minimum size=4mm]

\tikzstyle{block}=[rectangle,draw=black, thick, inner sep=5pt]
\tikzstyle{bullet}=[circle,draw=black, fill=black, thin, inner sep=2pt]

\tikzstyle{pre}=[<-,shorten <=1pt,>=stealth',semithick]
\tikzstyle{post}=[->,shorten >=1pt,>=stealth',semithick]
\tikzstyle{bi}=[<->,shorten >=1pt,shorten <=1pt, >=stealth',semithick]

\tikzstyle{mut}=[-,>=stealth',semithick]

\tikzstyle{treereset}=[dashed,->, shorten >=1pt,>=stealth',thin]

\usepackage{ifmtarg}
\usepackage{xifthen}
\makeatletter
% new counter to now which frame it is within the sequence
\newcounter{multiframecounter}
% initialize buffer for previously used frame title
\gdef\lastframetitle{\textit{undefined}}
% new environment for a multi-frame
\newenvironment{multiframe}[1][]{%
\ifthenelse{\isempty{#1}}{%
% if no frame title was set via optional parameter,
% only increase sequence counter by 1
\addtocounter{multiframecounter}{1}%
}{%
% new frame title has been provided, thus
% reset sequence counter to 1 and buffer frame title for later use
\setcounter{multiframecounter}{1}%
\gdef\lastframetitle{#1}%
}%
% start conventional frame environment and
% automatically set frame title followed by sequence counter
\begin{frame}%
\frametitle{\lastframetitle~{\normalfont(\arabic{multiframecounter})}}%
}{%
\end{frame}%
}
\makeatother

\makeatletter
\newdimen\tu@tmpa%
\newdimen\ydiffl%
\newdimen\xdiffl%
\newcommand\ydiff[2]{%
    \coordinate (tmpnamea) at (#1);%
    \coordinate (tmpnameb) at (#2);%
    \pgfextracty{\tu@tmpa}{\pgfpointanchor{tmpnamea}{center}}%
    \pgfextracty{\ydiffl}{\pgfpointanchor{tmpnameb}{center}}%
    \advance\ydiffl by -\tu@tmpa%
}
\newcommand\xdiff[2]{%
    \coordinate (tmpnamea) at (#1);%
    \coordinate (tmpnameb) at (#2);%
    \pgfextractx{\tu@tmpa}{\pgfpointanchor{tmpnamea}{center}}%
    \pgfextractx{\xdiffl}{\pgfpointanchor{tmpnameb}{center}}%
    \advance\xdiffl by -\tu@tmpa%
}
\makeatother
\newcommand{\copyrightbox}[3][r]{%
\begin{tikzpicture}%
\node[inner sep=0pt,minimum size=2em](ciimage){#2};
\usefont{OT1}{phv}{n}{n}\fontsize{4}{4}\selectfont
\ydiff{ciimage.south}{ciimage.north}
\xdiff{ciimage.west}{ciimage.east}
\ifthenelse{\equal{#1}{r}}{%
\node[inner sep=0pt,right=1ex of ciimage.south east,anchor=north west,rotate=90]%
{\raggedleft\color{black!50}\parbox{\the\ydiffl}{\raggedright{}#3}};%
}{%
\ifthenelse{\equal{#1}{l}}{%
\node[inner sep=0pt,right=1ex of ciimage.south west,anchor=south west,rotate=90]%
{\raggedleft\color{black!50}\parbox{\the\ydiffl}{\raggedright{}#3}};%
}{%
\node[inner sep=0pt,below=1ex of ciimage.south west,anchor=north west]%
{\raggedleft\color{black!50}\parbox{\the\xdiffl}{\raggedright{}#3}};%
}
}
\end{tikzpicture}
}


%% --------------------

%\usepackage[excludeor]{everyhook}
%\PushPreHook{par}{\setbox0=\lastbox\llap{MUH}}\box0}

%\vspace*{\stretch{1}

%\setbox0=\lastbox \llap{\textbullet\enskip}\box0}

\setlength{\parskip}{\fill}

\newcommand\noskips{\setlength{\parskip}{1ex}}
\newcommand\doskips{\setlength{\parskip}{\fill}}

\newcommand\xx{\par\vspace*{\stretch{1}}\par}
\newcommand\xxs{\par\vspace*{2ex}\par}
\newcommand\tuple[1]{\langle #1 \rangle}
\newcommand\code[1]{{\sf \footnotesize #1}}
\newcommand\ex[1]{\uline{Example:} \ifdefined \presentationonly \pause \fi
  \ifdefined\showexamples#1\xspace\else{\uline{\hspace*{2cm}}}\fi}

\newcommand\ceil[1]{\lceil #1 \rceil}


\AtBeginSection[]
{
   \begin{frame}
       \frametitle{Outline}
       \tableofcontents[currentsection]
   \end{frame}
}



\pgfdeclarelayer{edgelayer}
\pgfdeclarelayer{nodelayer}
\pgfsetlayers{edgelayer,nodelayer,main}

\tikzstyle{none}=[inner sep=0pt]
\tikzstyle{rn}=[circle,fill=Red,draw=Black,line width=0.8 pt]
\tikzstyle{gn}=[circle,fill=Lime,draw=Black,line width=0.8 pt]
\tikzstyle{yn}=[circle,fill=Yellow,draw=Black,line width=0.8 pt]
\tikzstyle{empty}=[circle,fill=White,draw=Black]
\tikzstyle{bw} = [rectangle, draw, fill=blue!20, 
    text width=4em, text centered, rounded corners, minimum height=2em]
    
    \newcommand{\CcNote}[1]{% longname
	This work is licensed under the \textit{Creative Commons #1 3.0 License}.%
}
\newcommand{\CcImageBy}[1]{%
	\includegraphics[scale=#1]{creative_commons/cc_by_30.pdf}%
}
\newcommand{\CcImageSa}[1]{%
	\includegraphics[scale=#1]{creative_commons/cc_sa_30.pdf}%
}
\newcommand{\CcImageNc}[1]{%
	\includegraphics[scale=#1]{creative_commons/cc_nc_30.pdf}%
}
\newcommand{\CcGroupBySa}[2]{% zoom, gap
	\CcImageBy{#1}\hspace*{#2}\CcImageNc{#1}\hspace*{#2}\CcImageSa{#1}%
}
\newcommand{\CcLongnameByNcSa}{Attribution-NonCommercial-ShareAlike}


\newenvironment{changemargin}[1]{% 
  \begin{list}{}{% 
    \setlength{\topsep}{0pt}% 
    \setlength{\leftmargin}{#1}% 
    \setlength{\rightmargin}{1em}
    \setlength{\listparindent}{\parindent}% 
    \setlength{\itemindent}{\parindent}% 
    \setlength{\parsep}{\parskip}% 
  }% 
  \item[]}{\end{list}} 




\title{Lecture 14 --- More About Functions}

\author{J. Zarnett\\
\texttt{jzarnett@uwaterloo.ca}}
\institute{Department of Electrical and Computer Engineering \\
  University of Waterloo}
\date{\today}

\begin{document}

\begin{frame}
  \titlepage
  
  \begin{center}
  \small{Acknowledgments: W.D. Bishop}
  \end{center}
 \end{frame}
 
\part{Re-Using Functions}
\begin{frame}\partpage\end{frame}


\begin{frame}
\frametitle{Functions, Continued}
In the last lecture, we discussed the ``what'' of functions, but not why.

One of the strategies for engineering problem-solving is the \alert{divide-and-conquer} approach.

Break a large problem into smaller sub-problems. Solve the smaller problems, then recombine the answers to get the final answer.

A major advantage of functions is the ability to re-use some code we have already written.

\end{frame}

\begin{frame}[fragile]
\frametitle{Re-Use of Functions}

{\scriptsize
\begin{verbatim}
int students = 385;
int[] assignments = new int[students];
int[] midterms = new int[students];
int[] finals = new int[students];

for ( int i = 0; i < assignments.Length; i++ )
{
    assignments[i] = 0;
}    
for ( int j = 0; j < midterms.Length; j++ )
{
    midterms[j] = 0;
}
for ( int k = 0; k < finals.Length; k++ )
{
    finals[k] = 0;
}
\end{verbatim}
}
\end{frame}


\begin{frame}[fragile]
\frametitle{Re-Use of Functions}

First, define the function \texttt{fillWithZeros()}:

{\scriptsize
\begin{verbatim}
void fillWithZeros( int[] array ) 
{
    for ( int i = 0; i < array.Length; i++ )
    {
        array[i] = 0;
    }
}    
\end{verbatim}
}

Then, use it:
{\scriptsize
\begin{verbatim}
int students = 385;
int[] assignments = new int[students];
int[] midterms = new int[students];
int[] finals = new int[students];

fillWithZeros( assignments );
fillWithZeros( midterms );
fillWithZeros( finals );
\end{verbatim}
}

\end{frame}

\begin{frame}
\frametitle{Advantages of Re-Using Functions}

\begin{enumerate}
	\item Decompose the problem into smaller subproblems
	\item Reduce duplicate code in the program
	\item Allow re-use of code elsewhere in the program
	\item Hide the implementation details
\end{enumerate}

Duplicate code is usually produced by copy-and-pasting existing code.

That means twice as many places to change to fix something.

Avoid duplicating code where possible.

\end{frame}

\begin{frame}
\frametitle{Structuring Your Program}
Functions also give us the ability to structure our program, rather than just have all parts of it jammed together in \texttt{Main}.

It should be possible to summarize the purpose of a function in a sentence or two. 

If a function is long or does multiple things, chances are that there are smaller functions within the code that can be separated out. 

Writing code with good structure will make your program:
\begin{itemize}
	\item Easier to understand
	\item Easier to update in the future
	\item Easier to fix if it's not correct
	\item More re-usable
\end{itemize}

\end{frame}

\begin{frame}
\frametitle{On Good Structure}
Using good structure may not seem like it makes a difference in programming for ECE~150 since the amount of code written is small.

Get into proper habits now and it will save you headaches later on.

In later courses (operating systems, fourth year design project), you will write a lot more code and you'll need to manage it properly.

In industry, codebases can be hundreds of thousands of lines. Without structure, the program will quickly become a disaster.

\end{frame}

\part{Documenting Functions}
\begin{frame}\partpage\end{frame}

\begin{frame}
\frametitle{Documentation}
This is not intended as a long essay on documentation; just a strategy that will help you write better functions.

Using comments, above each function you write, take a moment to write the \alert{precondition} and \alert{postcondition}.

Precondition: anything that is assumed to be true when the function is called. 

Postcondition: the effect of the function call (i.e., what happens if you call it when the precondition is true).

These two pieces of documentation can tell you how to use a function properly, without having to examine the implementation.

\end{frame}

\begin{frame}[fragile]
\frametitle{Using Pre- and Postconditions}

\begin{verbatim}
// Precondition: there is at least one grade in the array
// Postcondition: the average of all the grades in the 
//                array is returned
static double calculateAverage( double[] grades )
{
    double sum = 0;
    for ( int i = 0; i < grades.Length; i++ )
    {
        sum += grades[i];
    }
    return ( sum / grades.Length ); 
}
\end{verbatim}

\end{frame}

\begin{frame}
\frametitle{Violating the Precondition}
What happens if we violate the precondition of \texttt{calculateAverage()} and \texttt{grades} is an array of capacity 0?

And who is at fault if the precondition is violated and an error occurs?

Think of pre- and postconditions as a contract. 

If the caller of the function lives up to the precondition, the function is responsible for fulfilling the postcondition.

\end{frame}

\part{Variable Scope}
\begin{frame}\partpage\end{frame}


\begin{frame}[fragile]
\frametitle{Variable Scope}
You may have noticed:

\begin{verbatim}
for ( int i = 0; i < 10; i++ )
{
    // Loop Body
}

Console.WriteLine( i );
\end{verbatim}

... produces a compile-time error at the \texttt{Console.WriteLine} statement, because the compiler does not know what \texttt{i} is.

\end{frame}

\begin{frame}[fragile]
\frametitle{Variable Scope}
The \alert{scope} of a variable is the context in which that variable is declared and available for use.

Here, variable \texttt{i} below is available only within the body of the loop.

\begin{verbatim}
for ( int i = 0; i < 10; i++ )
{
    Console.WriteLine( i ); // Valid
}

Console.WriteLine( i ); // Compile time error
\end{verbatim}

\end{frame}

\begin{frame}[fragile]
\frametitle{Variable Scope in Functions}
This same principle applies to functions.\\
\quad A variable defined in \texttt{Main} is not defined in another function.

A variable defined inside a function is called a \alert{local variable}.

{\scriptsize
\begin{verbatim}
static void Main ()
{
    double val = 0;
    int[] zeroFill = new int[42];
    fillWithZeros( zeroFill );
}
\end{verbatim}
}

{\scriptsize
\begin{verbatim}
void fillWithZeros( int[] array ) 
{
    for ( int i = 0; i < array.Length; i++ )
    {
        array[i] = 0;
    }
}    
\end{verbatim}
}

The variable \texttt{val} is not available in \texttt{fillWithZeros}.

\end{frame}

\begin{frame}[fragile]
\frametitle{Variable Scope in Functions}
Further observations on that same code:

{\scriptsize
\begin{verbatim}
static void Main ()
{
    double val = 0;
    int[] zeroFill = new int[42];
    fillWithZeros(zeroFill);
}
\end{verbatim}
}

{\scriptsize
\begin{verbatim}
void fillWithZeros(int[] array) 
{
    for (int i = 0; i < array.Length; i++)
    {
        array[i] = 0;
    }
}    
\end{verbatim}
}


The array \texttt{zeroFill} is a parameter to \texttt{fillWithZeros} so the will be accessible, but we must refer to it using the name \texttt{array}.

The name \texttt{zeroFill} is not in scope inside \texttt{fillWithZeros}.

\end{frame}

\begin{frame}
\frametitle{Knowing if a Variable is in Scope}
The good news is that it is a compile-time error if we try to use a variable that is not in scope.\\
\quad This means we know right away this needs to be fixed.

The use of \{ \} braces defines the scope of a variable.\\
\quad A variable is in scope within the braces that contain its definition.

It is possible to define a variable within a function, a loop, or even a selection statement. 

\end{frame}

\begin{frame}[fragile]
\frametitle{Defining a Variable within an \texttt{if}}

\begin{verbatim}
if ( x < 0 )
{
    bool invalid = ( x == -1 );
    // Block 1
    
} else if ( x > 100 ) {
    // Block 2
}
\end{verbatim}

The \texttt{bool} variable is in scope only within the \texttt{if ( x < 0 )} block (1).

\end{frame}

\begin{frame}[fragile]
\frametitle{Manual Variable Scope}
It is possible to manually define the scope of a variable using braces.

As before, the scope of the variable is defined by the \{ \} braces it is in.

\begin{verbatim}
int x = 100;
{
    int y = 54;
    Console.WriteLine( y );
}
Console.WriteLine( x ); // y unavailable in this context
\end{verbatim}

But there is unlikely to be a good reason to do this.


\end{frame}

\part{Function Overloading}
\begin{frame}\partpage\end{frame}

\begin{frame}
\frametitle{Function Overloading}
If you just copy and paste a function, you will see a compile-time error, because you have two functions with the same signature.

Yes, the signature is what matters, even if the implementations differ.

If two functions have the same signature, the compiler can't tell which one you mean when you call that function.

Could two functions have the same name, but different signatures?\\
\quad Yes -- this is called \alert{function overloading}.

\end{frame}

\begin{frame}
\frametitle{Function Overloading}

A function is \alert{overloaded} if one function name has more than one implementation. Consider the two function signatures below:

\texttt{double average( double n1, double n2 )}

\texttt{double average( double n1, double n2, double n3) }

The compiler looks at the actual parameters when the function is called to figure out which implementation should execute.

Example: if \texttt{average( 7.5, 12.5 );} is called, the first of the two signatures above will be the function that is called.

\end{frame}

\begin{frame}
\frametitle{Rules for Overloading}

If two functions have the same name, they must have either:\\
\quad different numbers of formal parameters, or\\
\quad formal parameters of different types.

You cannot overload a function by giving two definitions that differ only in the return type. So this is an error:\\
\texttt{double average( int n1, int n2 )}\\
\texttt{int average( int n1, int n2 )}

Why not? Because the compiler won't know which \texttt{average} you mean.

\end{frame}

\begin{frame}
\frametitle{Overloading and Type Promotion}

Type promotion is still in effect when calling overloaded functions.

Picture these two functions:\\
\texttt{int compute( int n1 )}\\
\texttt{int compute( double n1 )}

If the call is \texttt{compute( 10 )}, it's possible both of these will match: 10 may be an \texttt{int} but it could be automatically promoted to \texttt{double}.

C\# will choose the ``best'' match: the first function is chosen because 
the compiler will try to find a match without conversion, if it can.

That is ``better'' than converting \texttt{int} to \texttt{double}.

This may result in some unexpected behaviour, but it is consistent.

\end{frame}

\begin{frame}
\frametitle{Overloading and Type Promotion}
Sometimes, however, C\# can't figure out which choice is ``best''.

Picture these two function signatures:\\
\texttt{int compute( int n1, double n2 )}\\
\texttt{int compute( double n1, int n2 )}

If the call is \texttt{compute( 10, 10 )} there are two ways to resolve this:\\
\quad promote the first parameter to \texttt{double}; or\\
\quad promote the second parameter to \texttt{double}

C\# can't figure out which of these is better, because they are equal. This will be a compile-time error because the situation is ambiguous.

\end{frame}

\begin{frame}[fragile]
\frametitle{Proper Use of Overloading}
You can make proper use of overloading as a way to have some ``default'' values in function calls.

Here are two function signatures:\\
\texttt{double calculate( double x, double y )}\\
\texttt{double calculate( double x, double y, bool option )}

The implementation of the first function might look like this:
\begin{verbatim}
double calculate( double x, double y )
{
    return calculate( x, y, true );
}
\end{verbatim}
\end{frame}


\begin{frame}
\frametitle{About Overloading}

Overloading may be useful in some circumstances.

You can use overloading to have some ``default'' values, or in cases where you want to perform the same operation using different inputs.

Avoid overloading where there is the possibility for confusion, such as in the presence of type promotion.

\end{frame}


\end{document}

