
\documentclass[letterpaper,hide notes,xcolor={table,svgnames},pdftex]{beamer}
\def\showexamples{t}


%\usepackage[svgnames]{xcolor}

%% Demo talk
%\documentclass[letterpaper,notes=show]{beamer}

\usecolortheme{crane}%seahorse crane
\setbeamertemplate{navigation symbols}{}

\usetheme{MyPittsburgh}
%\usetheme{Frankfurt}

%\usepackage{tipa}

\usepackage{hyperref}
\usepackage{graphicx,xspace}
\usepackage[normalem]{ulem}

\newcommand\SF[1]{$\bigstar$\footnote{SF: #1}}

\usepackage{paratype}
\renewcommand*\familydefault{\sfdefault} %% Only if the base font of the document is to be sans serif
\usepackage[zerostyle=c]{newtxtt}
\usepackage[T1]{fontenc}

\newcounter{tmpnumSlide}
\newcounter{tmpnumNote}

\usepackage{xcolor}
\usepackage{tabu}
\definecolor{light-gray}{gray}{0.75}
\taburulecolor{light-gray}

% old question code
%\newcommand\question[1]{{$\bigstar$ \small \onlySlide{2}{#1}}}
% \newcommand\nquestion[1]{\ifdefined \presentationonly \textcircled{?} \fi \note{\par{\Large \textbf{?}} #1}}
% \newcommand\nanswer[1]{\note{\par{\Large \textbf{A}} #1}}


 \newcommand\mnote[1]{%
   \addtocounter{tmpnumSlide}{1}
   \ifdefined\showcues {~\tiny\fbox{\arabic{tmpnumSlide}}}\fi
   \note{\setlength{\parskip}{1ex}\addtocounter{tmpnumNote}{1}\textbf{\Large \arabic{tmpnumNote}:} {#1\par}}}

\newcommand\mmnote[1]{\note{\setlength{\parskip}{1ex}#1\par}}

%\newcommand\mnote[2][]{\ifdefined\handoutwithnotes {~\tiny\fbox{#1}}\fi
% \note{\setlength{\parskip}{1ex}\textbf{\Large #1:} #2\par}}

%\newcommand\mnote[2][]{{\tiny\fbox{#1}} \note{\setlength{\parskip}{1ex}\textbf{\Large #1:} #2\par}}

\newcommand\mquestion[2]{{~\color{red}\fbox{?}}\note{\setlength{\parskip}{1ex}\par{\Large \textbf{?}} #1} \note{\setlength{\parskip}{1ex}\par{\Large \textbf{A}} #2\par}\ifdefined \presentationonly \pause \fi}

\newcommand\blackboard[1]{%
\ifdefined   \showblackboard
  {#1}
  \else {\begin{center} \fbox{\colorbox{blue!30}{%
         \begin{minipage}{.95\linewidth}%
           \hspace{\stretch{1}} Some space intentionally left blank; done at the blackboard.%
         \end{minipage}}}\end{center}}%
         \fi%
}



%\newcommand\q{\tikz \node[thick,color=black,shape=circle]{?};}
%\newcommand\q{\ifdefined \presentationonly \textcircled{?} \fi}

\usepackage{listings}
\lstset{%
  keywordstyle=\bfseries,
  aboveskip=15pt,
  belowskip=15pt,
  captionpos=b,
  identifierstyle=\ttfamily,
  escapeinside={(*@}{@*)},
  stringstyle=\ttfamiliy,
  frame=lines,
  numbers=left, basicstyle=\scriptsize, numberstyle=\tiny, stepnumber=0, numbersep=2pt}

\usepackage{siunitx}
\newcommand\sius[1]{\num[group-separator = {,}]{#1}\si{\micro\second}}
\newcommand\sims[1]{\num[group-separator = {,}]{#1}\si{\milli\second}}
\newcommand\sins[1]{\num[group-separator = {,}]{#1}\si{\nano\second}}
\sisetup{group-separator = {,}, group-digits = true}

%% -------------------- tikz --------------------
\usepackage{tikz}
\usetikzlibrary{positioning}
\usetikzlibrary{arrows,backgrounds,automata,decorations.shapes,decorations.pathmorphing,decorations.markings,decorations.text}

\tikzstyle{place}=[circle,draw=blue!50,fill=blue!20,thick, inner sep=0pt,minimum size=6mm]
\tikzstyle{transition}=[rectangle,draw=black!50,fill=black!20,thick, inner sep=0pt,minimum size=4mm]

\tikzstyle{block}=[rectangle,draw=black, thick, inner sep=5pt]
\tikzstyle{bullet}=[circle,draw=black, fill=black, thin, inner sep=2pt]

\tikzstyle{pre}=[<-,shorten <=1pt,>=stealth',semithick]
\tikzstyle{post}=[->,shorten >=1pt,>=stealth',semithick]
\tikzstyle{bi}=[<->,shorten >=1pt,shorten <=1pt, >=stealth',semithick]

\tikzstyle{mut}=[-,>=stealth',semithick]

\tikzstyle{treereset}=[dashed,->, shorten >=1pt,>=stealth',thin]

\usepackage{ifmtarg}
\usepackage{xifthen}
\makeatletter
% new counter to now which frame it is within the sequence
\newcounter{multiframecounter}
% initialize buffer for previously used frame title
\gdef\lastframetitle{\textit{undefined}}
% new environment for a multi-frame
\newenvironment{multiframe}[1][]{%
\ifthenelse{\isempty{#1}}{%
% if no frame title was set via optional parameter,
% only increase sequence counter by 1
\addtocounter{multiframecounter}{1}%
}{%
% new frame title has been provided, thus
% reset sequence counter to 1 and buffer frame title for later use
\setcounter{multiframecounter}{1}%
\gdef\lastframetitle{#1}%
}%
% start conventional frame environment and
% automatically set frame title followed by sequence counter
\begin{frame}%
\frametitle{\lastframetitle~{\normalfont(\arabic{multiframecounter})}}%
}{%
\end{frame}%
}
\makeatother

\makeatletter
\newdimen\tu@tmpa%
\newdimen\ydiffl%
\newdimen\xdiffl%
\newcommand\ydiff[2]{%
    \coordinate (tmpnamea) at (#1);%
    \coordinate (tmpnameb) at (#2);%
    \pgfextracty{\tu@tmpa}{\pgfpointanchor{tmpnamea}{center}}%
    \pgfextracty{\ydiffl}{\pgfpointanchor{tmpnameb}{center}}%
    \advance\ydiffl by -\tu@tmpa%
}
\newcommand\xdiff[2]{%
    \coordinate (tmpnamea) at (#1);%
    \coordinate (tmpnameb) at (#2);%
    \pgfextractx{\tu@tmpa}{\pgfpointanchor{tmpnamea}{center}}%
    \pgfextractx{\xdiffl}{\pgfpointanchor{tmpnameb}{center}}%
    \advance\xdiffl by -\tu@tmpa%
}
\makeatother
\newcommand{\copyrightbox}[3][r]{%
\begin{tikzpicture}%
\node[inner sep=0pt,minimum size=2em](ciimage){#2};
\usefont{OT1}{phv}{n}{n}\fontsize{4}{4}\selectfont
\ydiff{ciimage.south}{ciimage.north}
\xdiff{ciimage.west}{ciimage.east}
\ifthenelse{\equal{#1}{r}}{%
\node[inner sep=0pt,right=1ex of ciimage.south east,anchor=north west,rotate=90]%
{\raggedleft\color{black!50}\parbox{\the\ydiffl}{\raggedright{}#3}};%
}{%
\ifthenelse{\equal{#1}{l}}{%
\node[inner sep=0pt,right=1ex of ciimage.south west,anchor=south west,rotate=90]%
{\raggedleft\color{black!50}\parbox{\the\ydiffl}{\raggedright{}#3}};%
}{%
\node[inner sep=0pt,below=1ex of ciimage.south west,anchor=north west]%
{\raggedleft\color{black!50}\parbox{\the\xdiffl}{\raggedright{}#3}};%
}
}
\end{tikzpicture}
}


%% --------------------

%\usepackage[excludeor]{everyhook}
%\PushPreHook{par}{\setbox0=\lastbox\llap{MUH}}\box0}

%\vspace*{\stretch{1}

%\setbox0=\lastbox \llap{\textbullet\enskip}\box0}

\setlength{\parskip}{\fill}

\newcommand\noskips{\setlength{\parskip}{1ex}}
\newcommand\doskips{\setlength{\parskip}{\fill}}

\newcommand\xx{\par\vspace*{\stretch{1}}\par}
\newcommand\xxs{\par\vspace*{2ex}\par}
\newcommand\tuple[1]{\langle #1 \rangle}
\newcommand\code[1]{{\sf \footnotesize #1}}
\newcommand\ex[1]{\uline{Example:} \ifdefined \presentationonly \pause \fi
  \ifdefined\showexamples#1\xspace\else{\uline{\hspace*{2cm}}}\fi}

\newcommand\ceil[1]{\lceil #1 \rceil}


\AtBeginSection[]
{
   \begin{frame}
       \frametitle{Outline}
       \tableofcontents[currentsection]
   \end{frame}
}



\pgfdeclarelayer{edgelayer}
\pgfdeclarelayer{nodelayer}
\pgfsetlayers{edgelayer,nodelayer,main}

\tikzstyle{none}=[inner sep=0pt]
\tikzstyle{rn}=[circle,fill=Red,draw=Black,line width=0.8 pt]
\tikzstyle{gn}=[circle,fill=Lime,draw=Black,line width=0.8 pt]
\tikzstyle{yn}=[circle,fill=Yellow,draw=Black,line width=0.8 pt]
\tikzstyle{empty}=[circle,fill=White,draw=Black]
\tikzstyle{bw} = [rectangle, draw, fill=blue!20, 
    text width=4em, text centered, rounded corners, minimum height=2em]
    
    \newcommand{\CcNote}[1]{% longname
	This work is licensed under the \textit{Creative Commons #1 3.0 License}.%
}
\newcommand{\CcImageBy}[1]{%
	\includegraphics[scale=#1]{creative_commons/cc_by_30.pdf}%
}
\newcommand{\CcImageSa}[1]{%
	\includegraphics[scale=#1]{creative_commons/cc_sa_30.pdf}%
}
\newcommand{\CcImageNc}[1]{%
	\includegraphics[scale=#1]{creative_commons/cc_nc_30.pdf}%
}
\newcommand{\CcGroupBySa}[2]{% zoom, gap
	\CcImageBy{#1}\hspace*{#2}\CcImageNc{#1}\hspace*{#2}\CcImageSa{#1}%
}
\newcommand{\CcLongnameByNcSa}{Attribution-NonCommercial-ShareAlike}


\newenvironment{changemargin}[1]{% 
  \begin{list}{}{% 
    \setlength{\topsep}{0pt}% 
    \setlength{\leftmargin}{#1}% 
    \setlength{\rightmargin}{1em}
    \setlength{\listparindent}{\parindent}% 
    \setlength{\itemindent}{\parindent}% 
    \setlength{\parsep}{\parskip}% 
  }% 
  \item[]}{\end{list}} 




\title{Lecture 15 --- Parameter Passing}

\author{J. Zarnett\\
\texttt{jzarnett@uwaterloo.ca}}
\institute{Department of Electrical and Computer Engineering \\
  University of Waterloo}
\date{\today}

\begin{document}

\begin{frame}
  \titlepage
  
  \begin{center}
  \small{Acknowledgments: W.D. Bishop}
  \end{center}
 \end{frame}

\part{Optional Parameters}
\begin{frame}\partpage\end{frame}

\begin{frame}[fragile]
\frametitle{Optional Parameters}
Last time, we examined function overloading as a way of having a default value for a function.

\begin{verbatim}
double calculate( double x, double y )
{
    return calculate( x, y, true );
}
\end{verbatim}

It is also possible to specify in a function's signature that one or more parameters are optional.

\begin{verbatim}
double compute( double a, double b,  double c = 1.0d )
{
    return a * b / c;
}
\end{verbatim}

\end{frame}

\begin{frame}[fragile]
\frametitle{Optional Parameters}
In the previous example, the parameter \texttt{c} is optional:\\
\quad If the caller provides a value, that value is used.\\
\quad Otherwise, the default value of 1 is used.

Thus, both of these function calls are valid:\\
\quad \texttt{compute(5, 9, 2.5);}\\
\quad \texttt{compute(45, 12);}

This is like having an overloaded function, except in this case the compiler generates it for you. If we wrote it explicitly, it would be:

\begin{verbatim}
double compute( double a, double b )
{
    return compute( a, b, 1.0d );
}
\end{verbatim}

\end{frame}

\begin{frame}[fragile]
\frametitle{Optional Parameter Rules}
We may have multiple optional parameters in the same function.
\vspace{-5em}

\begin{verbatim}
double compute( double a, double b = 2d,  double c = 1.0d )
\end{verbatim}

\vspace{-2em}

When default values are added, they always start at the last parameter and move right to left.\\
\quad This avoids developer and compiler confusion.

The following are errors:\\
\quad \texttt{compute( double a = 2d, double b,  double c )}\\
\quad \texttt{compute( double a = 2d, double b,  double c = 1.0d )}\\

\end{frame}



\part{Pass-By-Value, Pass-By-Reference}
\begin{frame}\partpage\end{frame}

\begin{frame}[fragile]
\frametitle{Pass-By-Value}
Consider the function shown below:

\begin{verbatim}
static void changeValueTo99 ( int number )
{
    number = 99;
}
\end{verbatim}

And suppose this code appears in \texttt{Main}:
\begin{verbatim}
int score = 0;
changeValueTo99( score );
Console.WriteLine( score );
\end{verbatim}

What do you expect the output will be?

\end{frame}

\begin{frame}
\frametitle{Pass-By-Value}
Why did \texttt{score} stay at 0 and not get changed to 99?

Answer: because the default behaviour is \alert{pass-by-value}.

When \texttt{changeValueTo99()} is called, the parameter \texttt{number} is created and initialized with 0; the same \underline{value} as \texttt{score} holds.

However, \texttt{number} is its own copy, so a change made to \texttt{number} is not reflected in \texttt{score}. Two separate boxes in memory.

By default, all simple types and \texttt{struct} are passed by value.\\
\quad A copy is made when the function is called.

\end{frame}

\begin{frame}
\frametitle{Pass-By-Value Alternative}
Is there an alternative to pass-by-value? Yes: \alert{pass-by-reference}.

First, we need to define a \alert{reference}.

Recall our model for a variable \texttt{x}: it is a box labelled x.\\
\quad This box is located somewhere in memory.

The computer will keep track of the memory address of \texttt{x}.\\
\quad When the code uses \texttt{x}, the computer knows where it is.

A reference is, quite simply, a memory address.\\
\quad A reference to \texttt{x} is the location of where to find \texttt{x} in memory.

\end{frame}

\begin{frame}
\frametitle{Using a Reference}
Suppose a function has a formal parameter \texttt{y} and we provide an actual parameter \texttt{x}, both of type \texttt{int}.

When we pass \texttt{x} by value to a function, \texttt{y} is created as a copy of \texttt{x}; variables \texttt{x} and \texttt{y} have different memory addresses.

If we pass \texttt{x} by reference, instead of making a copy, we instead tell the function, ``here's where \texttt{x} is'' and \texttt{y} gets the same location.

Because \texttt{x} and \texttt{y} refer to the same thing, a change made to \texttt{y} is automatically a change to \texttt{x}.

\end{frame}

\begin{frame}
\frametitle{Rules for Passing By Reference}

The keyword in C\# for passing by reference is \alert{\texttt{ref}}.

The \texttt{ref} keyword on a function's formal parameter causes the parameter to be passed by reference.

%Any changes made to the variable in the function will be reflected in the variable when control passes back to the calling function.

The \texttt{ref} keyword must appear in both the declaration of the function and the invocation of the function.

%The \texttt{ref} keyword can only be applied to a variable or array that has been initialized.

\end{frame}

\begin{frame}[fragile]
\frametitle{Pass-By-Reference}
Consider the modified function shown below:

\begin{verbatim}
static void changeValueTo99 ( ref int number )
{
    number = 99;
}
\end{verbatim}

And suppose this code appears in \texttt{Main}:
\begin{verbatim}
int score = 0;
changeValueTo99( ref score );
Console.WriteLine( score );
\end{verbatim}

What do you expect the output will be now?

\end{frame}

\begin{frame}
\frametitle{Pass-By-Reference}

Why is \texttt{score} now 99 rather than 0?

When we passed \texttt{score} by reference, no copy of it was made. 

Instead, \texttt{number} was initialized in such a way that it is the same box in memory as \texttt{score}.\\
\quad Thus a change to \texttt{number} is automatically a change to \texttt{score}.

\end{frame}

\begin{frame}[fragile]
\frametitle{Value, Reference, and Arrays}

Consider the function shown below:

\begin{verbatim}
static void setFirstEntryToOne ( int[] numbers )
{
    numbers[0] = 1;
}
\end{verbatim}

And suppose this code appears in \texttt{Main}:
\begin{verbatim}
int[] scores = new int[10];
scores[0] = 0;
setFirstEntryToOne( scores );
Console.WriteLine( scores[0] );
\end{verbatim}

What do you expect the output will be?

\end{frame}

\begin{frame}
\frametitle{Value, Reference, and Arrays}
We did not make use of the \texttt{ref} keyword, so why did the value of \texttt{scores[0]} change to 1?

Unlike an \texttt{int}, \texttt{double}, or \texttt{struct}, in C\# an array is a \alert{reference type}.

\texttt{int[] scores} is a reference to a memory location for the array.\\
\quad It's directions telling us how to get to a memory location.\\
\quad Think of it like written directions on how to get to a restaurant.

When we access \texttt{scores[0]}, the computer looks up the location of the array, goes to the location, and then takes the first entry there.

\end{frame}

\begin{frame}
\frametitle{Value, Reference, and Arrays}
\texttt{scores} is passed by value, but this just makes a copy of the reference.\\
\quad In other words, \texttt{numbers} contains a copy of the address information.\\
\quad Extending the analogy: write out another copy of the directions.

Thus, \texttt{numbers[0]} is the same location in memory as \texttt{scores[0]}.\\
\quad Both sets of directions lead us to the same place.\\
\quad Once there, examine the first entry: it's the same memory address.

\end{frame}

\begin{frame}[fragile]
\frametitle{Value, Reference, and Arrays}

Consider the revised function shown below:

\begin{verbatim}
static void setFirstEntryToOne ( int[] numbers )
{
    numbers = new int[10];
    numbers[0] = 1;
}
\end{verbatim}

And suppose this code appears in \texttt{Main}:
\begin{verbatim}
int[] scores = new int[10];
scores[0] = 0;
setFirstEntryToOne( scores );
Console.WriteLine( scores[0] );
\end{verbatim}

What do you expect the output will be?

\end{frame}

\begin{frame}
\frametitle{Value, Reference, and Arrays}
This time \texttt{scores[0]} is not changed. Why?

In \texttt{setFirstEntryToOne()} the \texttt{numbers} reference changed.\\
\quad We erased the directions and wrote new directions. 

Thus when we follow the directions to \texttt{numbers} we don't end up at the same place as we would if we followed the directions to \texttt{scores}.

Because we made another copy of the directions, the original directions are not affected.

Thus \texttt{scores} still contains directions to the original memory location.

\end{frame}

\begin{frame}[fragile]
\frametitle{Value, Reference, and Arrays}

Now let's add \texttt{ref} to the array example:

\begin{verbatim}
static void setFirstEntryToOne ( ref int[] numbers )
{
    numbers = new int[10];
    numbers[0] = 1;
}
\end{verbatim}

And suppose this code appears in \texttt{Main}:
\begin{verbatim}
int[] scores = new int[10];
scores[0] = 0;
setFirstEntryToOne( ref scores );
Console.WriteLine( scores[0] );
\end{verbatim}

What do you expect the output will be?

\end{frame}

\begin{frame}
\frametitle{Value, Reference, and Arrays}
Why did \texttt{scores[0]} change this time?

This time the reference \texttt{scores} was passed by reference.

Instead of two copies of the directions, there is only one copy.

Thus when the statement \texttt{numbers = new int[10];} changes the directions to \texttt{numbers}, the directions to \texttt{score} are changed too.

\end{frame}

\begin{frame}
\frametitle{Comments on Parameter Passing}
Passing parameters by reference can be useful when swapping things.\\
\quad Example: exchange two integer values (see the next slide).

Passing parameters by reference allows a software designer to build a function that returns several results or changes several variables.

In reality, this is rarely necessary since you can always create more functions to produce more results.

Passing by reference should be avoided unless absolutely necessary.

We still need to examine pass by reference vs. pass by value because of the behaviour of arrays when used as function parameters.\\
\quad Strings are immutable; to ``change'' one, we update the reference.
\end{frame}

\begin{frame}[fragile]
\frametitle{Example: Swapping Two Integers}
{\scriptsize
\begin{verbatim}
static void swap ( ref int r1, ref int r2 )
{
    int tmp = r1;
    r1 = r2;
    r2 = tmp;
}

static void Main( )
{
    int a = 5;
    int b = 12;
    
    Console.WriteLine( a );
    Console.WriteLine( b );
    
    swap( ref a, ref b );
    
    Console.WriteLine( a );
    Console.WriteLine( b );
}       
\end{verbatim}
}
\end{frame}

\end{document}

