
\documentclass[letterpaper,hide notes,xcolor={table,svgnames},pdftex]{beamer}
\def\showexamples{t}


%\usepackage[svgnames]{xcolor}

%% Demo talk
%\documentclass[letterpaper,notes=show]{beamer}

\usecolortheme{crane}%seahorse crane
\setbeamertemplate{navigation symbols}{}

\usetheme{MyPittsburgh}
%\usetheme{Frankfurt}

%\usepackage{tipa}

\usepackage{hyperref}
\usepackage{graphicx,xspace}
\usepackage[normalem]{ulem}

\newcommand\SF[1]{$\bigstar$\footnote{SF: #1}}

\usepackage{paratype}
\renewcommand*\familydefault{\sfdefault} %% Only if the base font of the document is to be sans serif
\usepackage[zerostyle=c]{newtxtt}
\usepackage[T1]{fontenc}

\newcounter{tmpnumSlide}
\newcounter{tmpnumNote}

\usepackage{xcolor}
\usepackage{tabu}
\definecolor{light-gray}{gray}{0.75}
\taburulecolor{light-gray}

% old question code
%\newcommand\question[1]{{$\bigstar$ \small \onlySlide{2}{#1}}}
% \newcommand\nquestion[1]{\ifdefined \presentationonly \textcircled{?} \fi \note{\par{\Large \textbf{?}} #1}}
% \newcommand\nanswer[1]{\note{\par{\Large \textbf{A}} #1}}


 \newcommand\mnote[1]{%
   \addtocounter{tmpnumSlide}{1}
   \ifdefined\showcues {~\tiny\fbox{\arabic{tmpnumSlide}}}\fi
   \note{\setlength{\parskip}{1ex}\addtocounter{tmpnumNote}{1}\textbf{\Large \arabic{tmpnumNote}:} {#1\par}}}

\newcommand\mmnote[1]{\note{\setlength{\parskip}{1ex}#1\par}}

%\newcommand\mnote[2][]{\ifdefined\handoutwithnotes {~\tiny\fbox{#1}}\fi
% \note{\setlength{\parskip}{1ex}\textbf{\Large #1:} #2\par}}

%\newcommand\mnote[2][]{{\tiny\fbox{#1}} \note{\setlength{\parskip}{1ex}\textbf{\Large #1:} #2\par}}

\newcommand\mquestion[2]{{~\color{red}\fbox{?}}\note{\setlength{\parskip}{1ex}\par{\Large \textbf{?}} #1} \note{\setlength{\parskip}{1ex}\par{\Large \textbf{A}} #2\par}\ifdefined \presentationonly \pause \fi}

\newcommand\blackboard[1]{%
\ifdefined   \showblackboard
  {#1}
  \else {\begin{center} \fbox{\colorbox{blue!30}{%
         \begin{minipage}{.95\linewidth}%
           \hspace{\stretch{1}} Some space intentionally left blank; done at the blackboard.%
         \end{minipage}}}\end{center}}%
         \fi%
}



%\newcommand\q{\tikz \node[thick,color=black,shape=circle]{?};}
%\newcommand\q{\ifdefined \presentationonly \textcircled{?} \fi}

\usepackage{listings}
\lstset{%
  keywordstyle=\bfseries,
  aboveskip=15pt,
  belowskip=15pt,
  captionpos=b,
  identifierstyle=\ttfamily,
  escapeinside={(*@}{@*)},
  stringstyle=\ttfamiliy,
  frame=lines,
  numbers=left, basicstyle=\scriptsize, numberstyle=\tiny, stepnumber=0, numbersep=2pt}

\usepackage{siunitx}
\newcommand\sius[1]{\num[group-separator = {,}]{#1}\si{\micro\second}}
\newcommand\sims[1]{\num[group-separator = {,}]{#1}\si{\milli\second}}
\newcommand\sins[1]{\num[group-separator = {,}]{#1}\si{\nano\second}}
\sisetup{group-separator = {,}, group-digits = true}

%% -------------------- tikz --------------------
\usepackage{tikz}
\usetikzlibrary{positioning}
\usetikzlibrary{arrows,backgrounds,automata,decorations.shapes,decorations.pathmorphing,decorations.markings,decorations.text}

\tikzstyle{place}=[circle,draw=blue!50,fill=blue!20,thick, inner sep=0pt,minimum size=6mm]
\tikzstyle{transition}=[rectangle,draw=black!50,fill=black!20,thick, inner sep=0pt,minimum size=4mm]

\tikzstyle{block}=[rectangle,draw=black, thick, inner sep=5pt]
\tikzstyle{bullet}=[circle,draw=black, fill=black, thin, inner sep=2pt]

\tikzstyle{pre}=[<-,shorten <=1pt,>=stealth',semithick]
\tikzstyle{post}=[->,shorten >=1pt,>=stealth',semithick]
\tikzstyle{bi}=[<->,shorten >=1pt,shorten <=1pt, >=stealth',semithick]

\tikzstyle{mut}=[-,>=stealth',semithick]

\tikzstyle{treereset}=[dashed,->, shorten >=1pt,>=stealth',thin]

\usepackage{ifmtarg}
\usepackage{xifthen}
\makeatletter
% new counter to now which frame it is within the sequence
\newcounter{multiframecounter}
% initialize buffer for previously used frame title
\gdef\lastframetitle{\textit{undefined}}
% new environment for a multi-frame
\newenvironment{multiframe}[1][]{%
\ifthenelse{\isempty{#1}}{%
% if no frame title was set via optional parameter,
% only increase sequence counter by 1
\addtocounter{multiframecounter}{1}%
}{%
% new frame title has been provided, thus
% reset sequence counter to 1 and buffer frame title for later use
\setcounter{multiframecounter}{1}%
\gdef\lastframetitle{#1}%
}%
% start conventional frame environment and
% automatically set frame title followed by sequence counter
\begin{frame}%
\frametitle{\lastframetitle~{\normalfont(\arabic{multiframecounter})}}%
}{%
\end{frame}%
}
\makeatother

\makeatletter
\newdimen\tu@tmpa%
\newdimen\ydiffl%
\newdimen\xdiffl%
\newcommand\ydiff[2]{%
    \coordinate (tmpnamea) at (#1);%
    \coordinate (tmpnameb) at (#2);%
    \pgfextracty{\tu@tmpa}{\pgfpointanchor{tmpnamea}{center}}%
    \pgfextracty{\ydiffl}{\pgfpointanchor{tmpnameb}{center}}%
    \advance\ydiffl by -\tu@tmpa%
}
\newcommand\xdiff[2]{%
    \coordinate (tmpnamea) at (#1);%
    \coordinate (tmpnameb) at (#2);%
    \pgfextractx{\tu@tmpa}{\pgfpointanchor{tmpnamea}{center}}%
    \pgfextractx{\xdiffl}{\pgfpointanchor{tmpnameb}{center}}%
    \advance\xdiffl by -\tu@tmpa%
}
\makeatother
\newcommand{\copyrightbox}[3][r]{%
\begin{tikzpicture}%
\node[inner sep=0pt,minimum size=2em](ciimage){#2};
\usefont{OT1}{phv}{n}{n}\fontsize{4}{4}\selectfont
\ydiff{ciimage.south}{ciimage.north}
\xdiff{ciimage.west}{ciimage.east}
\ifthenelse{\equal{#1}{r}}{%
\node[inner sep=0pt,right=1ex of ciimage.south east,anchor=north west,rotate=90]%
{\raggedleft\color{black!50}\parbox{\the\ydiffl}{\raggedright{}#3}};%
}{%
\ifthenelse{\equal{#1}{l}}{%
\node[inner sep=0pt,right=1ex of ciimage.south west,anchor=south west,rotate=90]%
{\raggedleft\color{black!50}\parbox{\the\ydiffl}{\raggedright{}#3}};%
}{%
\node[inner sep=0pt,below=1ex of ciimage.south west,anchor=north west]%
{\raggedleft\color{black!50}\parbox{\the\xdiffl}{\raggedright{}#3}};%
}
}
\end{tikzpicture}
}


%% --------------------

%\usepackage[excludeor]{everyhook}
%\PushPreHook{par}{\setbox0=\lastbox\llap{MUH}}\box0}

%\vspace*{\stretch{1}

%\setbox0=\lastbox \llap{\textbullet\enskip}\box0}

\setlength{\parskip}{\fill}

\newcommand\noskips{\setlength{\parskip}{1ex}}
\newcommand\doskips{\setlength{\parskip}{\fill}}

\newcommand\xx{\par\vspace*{\stretch{1}}\par}
\newcommand\xxs{\par\vspace*{2ex}\par}
\newcommand\tuple[1]{\langle #1 \rangle}
\newcommand\code[1]{{\sf \footnotesize #1}}
\newcommand\ex[1]{\uline{Example:} \ifdefined \presentationonly \pause \fi
  \ifdefined\showexamples#1\xspace\else{\uline{\hspace*{2cm}}}\fi}

\newcommand\ceil[1]{\lceil #1 \rceil}


\AtBeginSection[]
{
   \begin{frame}
       \frametitle{Outline}
       \tableofcontents[currentsection]
   \end{frame}
}



\pgfdeclarelayer{edgelayer}
\pgfdeclarelayer{nodelayer}
\pgfsetlayers{edgelayer,nodelayer,main}

\tikzstyle{none}=[inner sep=0pt]
\tikzstyle{rn}=[circle,fill=Red,draw=Black,line width=0.8 pt]
\tikzstyle{gn}=[circle,fill=Lime,draw=Black,line width=0.8 pt]
\tikzstyle{yn}=[circle,fill=Yellow,draw=Black,line width=0.8 pt]
\tikzstyle{empty}=[circle,fill=White,draw=Black]
\tikzstyle{bw} = [rectangle, draw, fill=blue!20, 
    text width=4em, text centered, rounded corners, minimum height=2em]
    
    \newcommand{\CcNote}[1]{% longname
	This work is licensed under the \textit{Creative Commons #1 3.0 License}.%
}
\newcommand{\CcImageBy}[1]{%
	\includegraphics[scale=#1]{creative_commons/cc_by_30.pdf}%
}
\newcommand{\CcImageSa}[1]{%
	\includegraphics[scale=#1]{creative_commons/cc_sa_30.pdf}%
}
\newcommand{\CcImageNc}[1]{%
	\includegraphics[scale=#1]{creative_commons/cc_nc_30.pdf}%
}
\newcommand{\CcGroupBySa}[2]{% zoom, gap
	\CcImageBy{#1}\hspace*{#2}\CcImageNc{#1}\hspace*{#2}\CcImageSa{#1}%
}
\newcommand{\CcLongnameByNcSa}{Attribution-NonCommercial-ShareAlike}


\newenvironment{changemargin}[1]{% 
  \begin{list}{}{% 
    \setlength{\topsep}{0pt}% 
    \setlength{\leftmargin}{#1}% 
    \setlength{\rightmargin}{1em}
    \setlength{\listparindent}{\parindent}% 
    \setlength{\itemindent}{\parindent}% 
    \setlength{\parsep}{\parskip}% 
  }% 
  \item[]}{\end{list}} 




\title{Lecture 32 --- Sorting }

\author{J. Zarnett\\
\texttt{jzarnett@uwaterloo.ca}}
\institute{Department of Electrical and Computer Engineering \\
  University of Waterloo}
\date{\today}

\begin{document}

\begin{frame}
  \titlepage
  
  \begin{center}
  \small{Acknowledgments: W.D. Bishop}
  \end{center}
\end{frame}

\begin{frame}
\frametitle{The Sorting Problem}
We saw in the discussion about searching that sorting is worthwhile because it dramatically improves the speed of searching.

Sorted sequences of objects can often be manipulated quickly.\\
\quad Example: the binary search algorithm.

We might have an intuitive idea about what it means for data to be sorted, but let's agree on a formal definition.

\end{frame}
%Algorithms Book

\begin{frame}
\frametitle{The Sorting Problem}

The sorting problem, formally:\\
\quad \textbf{Input}: A sequence of $n$ numbers $\{a_{1}, a_{2}, ...,  a_{n}\}$\\
\quad \textbf{Output}: A reordering $\{a_{1}', a_{2}', ... a_{n}'\}$ of the input sequence \\
\quad \quad \quad such that $a_{1}' \leq a_{2}' \leq ... \leq a_{n}'$

In less formal terms, a solution to the sorting problem:\\
\quad takes a sequence of $n$ numbers as input; and \\
\quad returns those numbers in non-decreasing order.

For input of $\{31, 41, 59, 26, 41\}$, the output is $\{26, 31, 41, 41, 59 \}$

The step-by-step process by which we reorder these values is the \alert{sorting algorithm}.



\end{frame}
%Algorithms Book

\begin{frame}
\frametitle{More General Sorting}

Of course, it's possible to sort things other than numbers (e.g., names) and it might be desirable to sort in another order (e.g., descending).

Thus, a more general definition of a sorting algorithm:\\
 A sorting algorithm places a sequence of values in a predefined order.

To keep things simple while learning about sorting, we'll use:
\begin{itemize}
	\item numbers; and
	\item non-descending order
\end{itemize}

\end{frame}

\begin{frame}
\frametitle{Sorting Algorithm Discussion}
Many algorithms exist for sorting data:

\begin{itemize}
	\item Bogo sort
	\item Insertion sort
	\item Selection sort
	\item Shellsort
	\item Heapsort
	\item Quicksort
	\item Bubblesort
	\item Merge sort
\end{itemize}

... and more.

\end{frame}

\begin{frame}
\frametitle{Sorting Algorithm Discussion}

Ultimately, which algorithm is best for a given application may depend on a number of factors.

The above algorithms all have different characteristics:

\begin{itemize}
	\item Algorithm complexity
	\item Space complexity (storage needed to sort data)
	\item Performance (how long it takes)
	\item Data structure (what types it can be used on)
\end{itemize}

A sorting algorithm is correct if, for every input, it halts with the correct output.

An incorrect algorithm might not halt at all for some input sequences (infinite loop) or might return the output not sorted as expected.


\end{frame}



\begin{frame}
\frametitle{Simple Sorting Techniques}

In this course, we examine only a few of the sorting techniques:
\begin{enumerate}
	\item Bogo sort
	\item Selection sort
	\item Insertion sort
%	\item Shell sort
\end{enumerate}

The algorithms are (relatively) easy to understand.

The algorithms can be applied to arrays and linked lists.

More complex algorithms are introduced \& analyzed in ECE~250.

\end{frame}

\begin{frame}
\frametitle{Bogo Sort}
The Bogo sort can be summarized as follows:

While the array is unsorted, randomly order the entries.

Through random chance, we'll eventually end up with a sorted array.\\
\quad But how long will it take...?

This isn't a serious sorting algorithm. It's just used as a baseline to contrast with more realistic algorithms.

\end{frame}

\begin{frame}
\frametitle{Selection Sort}
Imagine that you have been taking notes for a course on loose-leaf paper and have been numbering the pages.

You accidentally knock over the pile of pages and they scatter and fly everywhere in your room, landing all over.

You gather them up, but they're out of order \& you want to sort them.

A strategy you may choose: selection sort.

\end{frame}

\begin{frame}
\frametitle{Selection Sort}
Selection Sort is based on searching: find the minimum value.

You have an unsorted pile and a (currently-empty) sorted pile.

Search the unsorted pile until you find page 1.\\
\quad Add it to the sorted pile.

Then look for page 2; add it to the sorted pile after page 1.

Repeat this procedure until all the pages are sorted again.

\end{frame}


\begin{frame}
\frametitle{Selection Sort Algorithm}


Given an array of $n$ entries:

For $index$ from 0 to $(n - 1)$\\
	\begin{enumerate}
		\item Find the index of the minimum value ($minIndex$) in the subset of the array from $index$ to $(n-1)$.
		\item Swap the value at $minIndex$ with the value at $index$
\end{enumerate}

To find the minimum index, we must use a linear search (because the binary search requires sorted data and we're trying to sort...).

\end{frame}

\begin{frame}
\frametitle{Selection Sort Illustration}
Consider sorting $\{12, 36, 20, -1\}$ using the selection sort.

After iteration $x$, the first $x$ values have been sorted.\\
Progress is shown by the red highlighting.

Iteration 1: Array Contents = $\{12, 36, 20, -1\}$\\
\quad Find minimum value between index 0 and (length - 1)\\
\quad Minimum value is -1, at index 3\\
\quad Swap minimum value of -1 with value of 12 in index 0.

\end{frame}

\begin{frame}
\frametitle{Selection Sort Illustration}
Iteration 2: Array Contents = $\{\alert{-1}, 36, 20, 12\}$\\
\quad Find minimum value between index 1 and (length - 1)\\
\quad Minimum value is 12, at index 3\\
\quad Swap minimum value of 12 with value of 36 in index 1
		
Iteration 3: Array Contents = $\{\alert{-1}, \alert{12}, 20, 36\}$\\
\quad Find minimum value between index 2 and (length - 1)\\
\quad Minimum value is 20, at index 2\\
\quad Swap minimum value of 20 with value of 20 in index 2

The array, after sorting, contains: $\{-1, 12, 20, 36\}$.

\end{frame}

\begin{frame}
\frametitle{Larger Array Example}

Consider sorting \{12, 36, 42, -1, 99, 20, 10, 19, 70\}.

	\{12, 36, 42, -1, 99, 20, 10, 19, 70\} 	// Initial array values\\
	\{\alert{-1}, 36, 42, 12, 99, 20, 10, 19, 70\}	// After iteration 1\\
	\{\alert{-1, 10}, 42, 12, 99, 20, 36, 19, 70\}	// After iteration 2\\
	\{\alert{-1, 10, 12}, 42, 99, 20, 36, 19, 70\}	// After iteration 3\\
	\{\alert{-1, 10, 12, 19}, 99, 20, 36, 42, 70\}	// After iteration 4\\
	\{\alert{-1, 10, 12, 19, 20}, 99, 36, 42, 70\}	// After iteration 5\\
	\{\alert{-1, 10, 12, 19, 20, 36}, 99, 42, 70\}	// After iteration 6\\
	\{\alert{-1, 10, 12, 19, 20, 36, 42}, 99, 70\}	// After iteration 7\\
	\{\alert{-1, 10, 12, 19, 20, 36, 42, 70, 99}\}	// After iteration 8

Note that a total of 8 iterations are required to sort 9 values.

In general, selection sort requires $n-1$ iterations to sort $n$ values.


\end{frame}

\begin{frame}[fragile]
\frametitle{Selection Sort Code}

Here is a simple code implementation of selection sort:

\begin{verbatim}
public static void SelectionSort( int[] data )
{
    int minIndex;

    for( int index = 0; index < data.Length - 1; index++ )
    {
        minIndex = Min( data, index );
        Swap( data, minIndex, index );
    }
}
\end{verbatim}

Implementations of \texttt{Min} and \texttt{Swap} are shown on the next slide\\ \quad ...but we (should) already know how to implement those!

\end{frame}

\begin{frame}[fragile]
\frametitle{Selection Sort Code}

{\scriptsize
\begin{verbatim}
static int Min( int[] data, int start )
{
    int minIndex = start;
		
    for( int index = start + 1; index < data.Length; index++ )
    {
        if( data[index] < data[minIndex] )
        {
            minIndex = index;
        }
    }
    return minIndex;
}

static void Swap( int[] data, int index1, int index2 )
{
    int tmp = data[index1];
    data[index1] = data[index2];
    data[index2] = tmp;
}
\end{verbatim}
}
\end{frame}

\begin{frame}[fragile]
\frametitle{Alternative Implementation}
The previous code example is written in such a way as to separate out the different parts, such as finding the minimum and the swap.

Consider a more compact implementation of selection sort:
\vspace{-1em}
{\scriptsize
\begin{verbatim}
public static void SelectionSort( int[] data )
{
  int minIndex;
  int temp;
  for (int i = 0; i < data.Length; i++) {
    for (int j = i, minIndex = i; j < data.Length; j++) {
        if (data[j] < data[minIndex])
        {
            minIndex = j;
        }
    }
    temp = data[i];
    data[i] = data[minIndex];
    data[minIndex] = temp;
  }
}
\end{verbatim}
}

\end{frame}


\begin{frame}
\frametitle{In-Place Sorting}

The implementation of selection sort we have examined is an example of an \alert{in-place} sort: no additional memory is allocated.

In-place means we don't have to allocate a new array for output.\\
\quad There is only one array and we re-order the values within that.

As an alternative strategy that takes more memory: 

For an input array of capacity $n$, allocate an output array of capacity $n$.\\
Then, for each entry of the input array: 
\begin{enumerate}
\item Find the minimum value in the input array
\item Remove it from the input array
\item Put it in the output array
\end{enumerate}

\end{frame}

\begin{frame}
\frametitle{In-Place Sorting}

Why would we choose the alternative that uses more memory?

There are very often space-time tradeoffs.\\
\quad An algorithm that takes more memory may run faster; \\
\quad An algorithm that doesn't use as much memory may run slower.

\end{frame}


\end{document}

