
\documentclass[letterpaper,hide notes,xcolor={table,svgnames},pdftex]{beamer}
\def\showexamples{t}


%\usepackage[svgnames]{xcolor}

%% Demo talk
%\documentclass[letterpaper,notes=show]{beamer}

\usecolortheme{crane}%seahorse crane
\setbeamertemplate{navigation symbols}{}

\usetheme{MyPittsburgh}
%\usetheme{Frankfurt}

%\usepackage{tipa}

\usepackage{hyperref}
\usepackage{graphicx,xspace}
\usepackage[normalem]{ulem}

\newcommand\SF[1]{$\bigstar$\footnote{SF: #1}}

\usepackage{paratype}
\renewcommand*\familydefault{\sfdefault} %% Only if the base font of the document is to be sans serif
\usepackage[zerostyle=c]{newtxtt}
\usepackage[T1]{fontenc}

\newcounter{tmpnumSlide}
\newcounter{tmpnumNote}

\usepackage{xcolor}
\usepackage{tabu}
\definecolor{light-gray}{gray}{0.75}
\taburulecolor{light-gray}

% old question code
%\newcommand\question[1]{{$\bigstar$ \small \onlySlide{2}{#1}}}
% \newcommand\nquestion[1]{\ifdefined \presentationonly \textcircled{?} \fi \note{\par{\Large \textbf{?}} #1}}
% \newcommand\nanswer[1]{\note{\par{\Large \textbf{A}} #1}}


 \newcommand\mnote[1]{%
   \addtocounter{tmpnumSlide}{1}
   \ifdefined\showcues {~\tiny\fbox{\arabic{tmpnumSlide}}}\fi
   \note{\setlength{\parskip}{1ex}\addtocounter{tmpnumNote}{1}\textbf{\Large \arabic{tmpnumNote}:} {#1\par}}}

\newcommand\mmnote[1]{\note{\setlength{\parskip}{1ex}#1\par}}

%\newcommand\mnote[2][]{\ifdefined\handoutwithnotes {~\tiny\fbox{#1}}\fi
% \note{\setlength{\parskip}{1ex}\textbf{\Large #1:} #2\par}}

%\newcommand\mnote[2][]{{\tiny\fbox{#1}} \note{\setlength{\parskip}{1ex}\textbf{\Large #1:} #2\par}}

\newcommand\mquestion[2]{{~\color{red}\fbox{?}}\note{\setlength{\parskip}{1ex}\par{\Large \textbf{?}} #1} \note{\setlength{\parskip}{1ex}\par{\Large \textbf{A}} #2\par}\ifdefined \presentationonly \pause \fi}

\newcommand\blackboard[1]{%
\ifdefined   \showblackboard
  {#1}
  \else {\begin{center} \fbox{\colorbox{blue!30}{%
         \begin{minipage}{.95\linewidth}%
           \hspace{\stretch{1}} Some space intentionally left blank; done at the blackboard.%
         \end{minipage}}}\end{center}}%
         \fi%
}



%\newcommand\q{\tikz \node[thick,color=black,shape=circle]{?};}
%\newcommand\q{\ifdefined \presentationonly \textcircled{?} \fi}

\usepackage{listings}
\lstset{%
  keywordstyle=\bfseries,
  aboveskip=15pt,
  belowskip=15pt,
  captionpos=b,
  identifierstyle=\ttfamily,
  escapeinside={(*@}{@*)},
  stringstyle=\ttfamiliy,
  frame=lines,
  numbers=left, basicstyle=\scriptsize, numberstyle=\tiny, stepnumber=0, numbersep=2pt}

\usepackage{siunitx}
\newcommand\sius[1]{\num[group-separator = {,}]{#1}\si{\micro\second}}
\newcommand\sims[1]{\num[group-separator = {,}]{#1}\si{\milli\second}}
\newcommand\sins[1]{\num[group-separator = {,}]{#1}\si{\nano\second}}
\sisetup{group-separator = {,}, group-digits = true}

%% -------------------- tikz --------------------
\usepackage{tikz}
\usetikzlibrary{positioning}
\usetikzlibrary{arrows,backgrounds,automata,decorations.shapes,decorations.pathmorphing,decorations.markings,decorations.text}

\tikzstyle{place}=[circle,draw=blue!50,fill=blue!20,thick, inner sep=0pt,minimum size=6mm]
\tikzstyle{transition}=[rectangle,draw=black!50,fill=black!20,thick, inner sep=0pt,minimum size=4mm]

\tikzstyle{block}=[rectangle,draw=black, thick, inner sep=5pt]
\tikzstyle{bullet}=[circle,draw=black, fill=black, thin, inner sep=2pt]

\tikzstyle{pre}=[<-,shorten <=1pt,>=stealth',semithick]
\tikzstyle{post}=[->,shorten >=1pt,>=stealth',semithick]
\tikzstyle{bi}=[<->,shorten >=1pt,shorten <=1pt, >=stealth',semithick]

\tikzstyle{mut}=[-,>=stealth',semithick]

\tikzstyle{treereset}=[dashed,->, shorten >=1pt,>=stealth',thin]

\usepackage{ifmtarg}
\usepackage{xifthen}
\makeatletter
% new counter to now which frame it is within the sequence
\newcounter{multiframecounter}
% initialize buffer for previously used frame title
\gdef\lastframetitle{\textit{undefined}}
% new environment for a multi-frame
\newenvironment{multiframe}[1][]{%
\ifthenelse{\isempty{#1}}{%
% if no frame title was set via optional parameter,
% only increase sequence counter by 1
\addtocounter{multiframecounter}{1}%
}{%
% new frame title has been provided, thus
% reset sequence counter to 1 and buffer frame title for later use
\setcounter{multiframecounter}{1}%
\gdef\lastframetitle{#1}%
}%
% start conventional frame environment and
% automatically set frame title followed by sequence counter
\begin{frame}%
\frametitle{\lastframetitle~{\normalfont(\arabic{multiframecounter})}}%
}{%
\end{frame}%
}
\makeatother

\makeatletter
\newdimen\tu@tmpa%
\newdimen\ydiffl%
\newdimen\xdiffl%
\newcommand\ydiff[2]{%
    \coordinate (tmpnamea) at (#1);%
    \coordinate (tmpnameb) at (#2);%
    \pgfextracty{\tu@tmpa}{\pgfpointanchor{tmpnamea}{center}}%
    \pgfextracty{\ydiffl}{\pgfpointanchor{tmpnameb}{center}}%
    \advance\ydiffl by -\tu@tmpa%
}
\newcommand\xdiff[2]{%
    \coordinate (tmpnamea) at (#1);%
    \coordinate (tmpnameb) at (#2);%
    \pgfextractx{\tu@tmpa}{\pgfpointanchor{tmpnamea}{center}}%
    \pgfextractx{\xdiffl}{\pgfpointanchor{tmpnameb}{center}}%
    \advance\xdiffl by -\tu@tmpa%
}
\makeatother
\newcommand{\copyrightbox}[3][r]{%
\begin{tikzpicture}%
\node[inner sep=0pt,minimum size=2em](ciimage){#2};
\usefont{OT1}{phv}{n}{n}\fontsize{4}{4}\selectfont
\ydiff{ciimage.south}{ciimage.north}
\xdiff{ciimage.west}{ciimage.east}
\ifthenelse{\equal{#1}{r}}{%
\node[inner sep=0pt,right=1ex of ciimage.south east,anchor=north west,rotate=90]%
{\raggedleft\color{black!50}\parbox{\the\ydiffl}{\raggedright{}#3}};%
}{%
\ifthenelse{\equal{#1}{l}}{%
\node[inner sep=0pt,right=1ex of ciimage.south west,anchor=south west,rotate=90]%
{\raggedleft\color{black!50}\parbox{\the\ydiffl}{\raggedright{}#3}};%
}{%
\node[inner sep=0pt,below=1ex of ciimage.south west,anchor=north west]%
{\raggedleft\color{black!50}\parbox{\the\xdiffl}{\raggedright{}#3}};%
}
}
\end{tikzpicture}
}


%% --------------------

%\usepackage[excludeor]{everyhook}
%\PushPreHook{par}{\setbox0=\lastbox\llap{MUH}}\box0}

%\vspace*{\stretch{1}

%\setbox0=\lastbox \llap{\textbullet\enskip}\box0}

\setlength{\parskip}{\fill}

\newcommand\noskips{\setlength{\parskip}{1ex}}
\newcommand\doskips{\setlength{\parskip}{\fill}}

\newcommand\xx{\par\vspace*{\stretch{1}}\par}
\newcommand\xxs{\par\vspace*{2ex}\par}
\newcommand\tuple[1]{\langle #1 \rangle}
\newcommand\code[1]{{\sf \footnotesize #1}}
\newcommand\ex[1]{\uline{Example:} \ifdefined \presentationonly \pause \fi
  \ifdefined\showexamples#1\xspace\else{\uline{\hspace*{2cm}}}\fi}

\newcommand\ceil[1]{\lceil #1 \rceil}


\AtBeginSection[]
{
   \begin{frame}
       \frametitle{Outline}
       \tableofcontents[currentsection]
   \end{frame}
}



\pgfdeclarelayer{edgelayer}
\pgfdeclarelayer{nodelayer}
\pgfsetlayers{edgelayer,nodelayer,main}

\tikzstyle{none}=[inner sep=0pt]
\tikzstyle{rn}=[circle,fill=Red,draw=Black,line width=0.8 pt]
\tikzstyle{gn}=[circle,fill=Lime,draw=Black,line width=0.8 pt]
\tikzstyle{yn}=[circle,fill=Yellow,draw=Black,line width=0.8 pt]
\tikzstyle{empty}=[circle,fill=White,draw=Black]
\tikzstyle{bw} = [rectangle, draw, fill=blue!20, 
    text width=4em, text centered, rounded corners, minimum height=2em]
    
    \newcommand{\CcNote}[1]{% longname
	This work is licensed under the \textit{Creative Commons #1 3.0 License}.%
}
\newcommand{\CcImageBy}[1]{%
	\includegraphics[scale=#1]{creative_commons/cc_by_30.pdf}%
}
\newcommand{\CcImageSa}[1]{%
	\includegraphics[scale=#1]{creative_commons/cc_sa_30.pdf}%
}
\newcommand{\CcImageNc}[1]{%
	\includegraphics[scale=#1]{creative_commons/cc_nc_30.pdf}%
}
\newcommand{\CcGroupBySa}[2]{% zoom, gap
	\CcImageBy{#1}\hspace*{#2}\CcImageNc{#1}\hspace*{#2}\CcImageSa{#1}%
}
\newcommand{\CcLongnameByNcSa}{Attribution-NonCommercial-ShareAlike}


\newenvironment{changemargin}[1]{% 
  \begin{list}{}{% 
    \setlength{\topsep}{0pt}% 
    \setlength{\leftmargin}{#1}% 
    \setlength{\rightmargin}{1em}
    \setlength{\listparindent}{\parindent}% 
    \setlength{\itemindent}{\parindent}% 
    \setlength{\parsep}{\parskip}% 
  }% 
  \item[]}{\end{list}} 




\title{Lecture 13 --- Functions }
\author{J. Zarnett\\
\texttt{jzarnett@uwaterloo.ca}}
\institute{Department of Electrical and Computer Engineering \\
  University of Waterloo}
\date{\today}

\begin{document}

\begin{frame}
  \titlepage
  
  \begin{center}
  \small{Acknowledgments: D.W. Harder, W.D. Bishop}
  \end{center}
 \end{frame}
 
\begin{frame}
\frametitle{Mathematical Functions}

You are no doubt familiar with the concept of a mathematical function: a relation between a set of inputs and output.

Mathematical functions take arguments in specific domains and return a value in another domain.

Consider $sin : R \rightarrow R$.\\
\quad Given a right-angled triangle with an angle $\theta$, return the ratio of the length of the opposite side to the length of the hypotenuse.


It takes a real number input  (in $R$) and returns a real number output.\\
\quad Each input is related to exactly one output. 

\end{frame}

\begin{frame}
\frametitle{Mathematical Functions}

Then there is evaluation of a function, such as $f(x) = x^{2}$.

$x$ is the input to this function, and the function returns some output, like $f(-5) = 25$. 

We often graph (on an x-y plane) the output of a function: $y = f(x)$.

\end{frame}

\begin{frame}
\frametitle{Mathematical Functions}

You've used mathematical functions a lot over the years:

Trigonometric functions: $cos(x), sin(x), tan(x)$\\
Absolute value function: $|x|$\\
Square root function: $\sqrt{x}$

These examples all take a single input, but there's no reason this has to be the case. Imagine a function $g(x, y) = x^{y}$, for example.

\end{frame}

\begin{frame}
\frametitle{Functions in Programs}
In a program, a function is much like a mathematical function.

A function takes 0 or more inputs and returns 0 or 1 values.

A function is sometimes also called a procedure or subroutine.

\end{frame}

\begin{frame}
\frametitle{Using a Function}
Assume for now that there is a defined function in the system for the absolute value function, $|x|$.

To make use of this function, we need to know its \alert{signature}.

The function signature tells us the name, what type(s) the input to the function is, and the type of the output.

The format is: \textit{returnType name} ( \textit{formalParameters...} )

\end{frame}

\begin{frame}
\frametitle{Formal Parameters}

We may have zero or more \alert{formal parameters}: a placeholder for whatever value will be given as input. 

The \textit{formalParameters} is a list of variables in the form \textit{type name1, type name2, ... type nameN}.

In $f(x)$, we state that $x$ is a formal parameter: it's a variable that we use in defining how $f(x)$ works.

\end{frame}

\begin{frame}
\frametitle{The \texttt{abs} Function}

Suppose this is the function signature: \texttt{double abs ( double x )}

Let's look at it from left to right:

The return type is \texttt{double}: the output will be a \texttt{double}.

The name of the function is \texttt{abs} (for \textbf{abs}olute value).

The formal parameters (inputs) are in the (~) brackets. There is one input, a \texttt{double} named \texttt{x}.


\end{frame}

\begin{frame}
\frametitle{Calling a Function}

Now that we know the function signature, we know how to use it.

Function \texttt{abs} expects the input of a \texttt{double} \& will output a \texttt{double}.

We are now ready to \alert{call} (or \alert{invoke}) this function.

To do so, we use the name \texttt{abs} followed by \texttt{(~)} brackets.\\
\quad Inside those brackets, we put the \alert{actual parameters}.

\texttt{double absX = abs( -9.5 );}

The output is the \alert{return value}, which is then stored in \texttt{absX}.

\end{frame}


\begin{frame}
\frametitle{Actual Parameters}
We saw already the formal parameters for \texttt{abs} is one variable: \texttt{double~x}.

The \alert{actual parameters} are the values actually given as input to the function at the time we want the function to execute.

We must provide one actual parameter value for each of the formal parameter values in the signature. 

The types must match (therefore order matters).

In the case of \texttt{abs}, we provide one actual parameter of type \texttt{double}; in this case, the literal \texttt{-9.5}.

\end{frame}

\begin{frame}
\frametitle{Call Me Maybe?}

We can also use a variable as an actual parameter:\\
\quad \texttt{double absY = abs( y );}

Variable \texttt{absY} will hold the absolute value of whatever \texttt{y} contained at the time the function was called.

Same thing with expressions:\\
\quad \texttt{double absZ = abs( y + 7 )}

Type promotion rules apply, so if \texttt{y} is an \texttt{int}, the actual parameter will be promoted to \texttt{double} before it is used in the function call.



\end{frame}

\begin{frame}
\frametitle{Using vs. Defining a Function}
Thus far it was assumed that there exists the \texttt{abs} function.\\
\quad i.e., someone else wrote it.

We now understand how to use an already-existing function.

Most of the time, however, we must define a function if we want to use it, since it won't already exist for us.

But: don't re-invent the wheel. If there already is an implementation, it is usually better to use that than to write your own.

\end{frame}

\begin{frame}[fragile]
\frametitle{Defining a Function}


A function definition has two parts: the signature and the body.

\begin{verbatim}
static double abs ( double x ) 
{
    // TODO: Implement Body
}
\end{verbatim}

We've already examined the signature; the first line of the definition.

For now we have to precede a function signature with \texttt{static}.\\
\quad Later in the course we'll find out what it means.

The body is some block of statements that we have to write to make the function do something useful.

\end{frame}

\begin{frame}
\frametitle{The \texttt{return} Keyword}

There's a keyword needed to implement a function: \texttt{return}.\\
\quad The \texttt{return} keyword is followed by the \alert{return value}.

The return value is the output of the function.

The \texttt{return} statement could appear anywhere within the function (but commonly it's at the end).

When a \texttt{return} statement is encountered, the return value becomes the function output and the execution of the function ends.

In this respect, \texttt{return} is a bit like the \texttt{break} statement.

\end{frame}

\begin{frame}[fragile]
\frametitle{The \texttt{abs} Function Implemented}

\begin{verbatim}
static double abs ( double x ) 
{
    if ( x >= 0 ) 
    {
        return x;
    }  else  {
        return -x;
    }
}
\end{verbatim}

In the definition, we use the formal parameter names. 

(In $f(x)$: the formal parameter $x$ is used to define the function ($x^{2}$).)

\end{frame}

\begin{frame}[fragile]
\frametitle{Execution Semantics}
When we call \texttt{double y = abs( -5.0 );}, what happens?

Much like when we evaluate an expression:\\
\quad \texttt{ a = b + 7;}\\
before we can do the assignment we have to evaluate the right side.

When we evaluate a function call, we go to the function definition and execute the lines of code there.

\end{frame}

\begin{frame}[fragile]
\frametitle{Execution Semantics}

\begin{verbatim}
static double abs ( double x ) 
{
    if ( x >= 0 ) 
    {
        return x;
    }  else  {
        return -x;
    }
}
\end{verbatim}

If \texttt{abs} was called with the actual parameter of \texttt{-5.0}, the value of \texttt{x} is \texttt{-5.0} and that is the value used in expressions within the function.

Here, the statement \texttt{return -x;} executes, which evaluates to \texttt{return - (-5.0)} $\rightarrow$ \texttt{return 5.0;}

\end{frame}

\begin{frame}
\frametitle{Execution Semantics}

Because we encountered a \texttt{return} statement, execution of the function is finished and we take the output of 5.0.

Our original assignment statement was \texttt{double y = abs( -5.0 );}

After evaluating \texttt{abs( -5.0 )}, the assignment is \texttt{double y = 5.0;}

And of course, we execute this assignment and store 5.0 in variable \texttt{y}.

\end{frame}

\begin{frame}
\frametitle{Fulfilling the Contract}

The function signature holds us to an important condition: we have to send back output in the type of the return type.

So for \texttt{abs}, because the return type is \texttt{double}, a call to that function must always return a \texttt{double}.


Our implementation is valid; it lives up to the condition that whatever the control flow in the function, a value of type \texttt{double} is returned.

If we forgot the \texttt{return} statement in the \texttt{else} block, that would be a compile-time error.

Of course, the compiler doesn't check the semantics: if you put \texttt{return 0.0d;} as the body of the function, it compiles (but is wrong).

\end{frame}

\begin{frame}
\frametitle{Return Types}
The \texttt{abs} example has a return type of \texttt{double}, but any type can be the return type of a function.

This includes enumerated and programmer-defined (\texttt{struct}) types.

Every function must have a return type, but not all functions have a return value.

A function that does not return a value has a return type of \texttt{void}.

\end{frame}

\begin{frame}
\frametitle{Into the \texttt{void}}
If a function has return type \texttt{void}, it can still take input of zero or more input parameters. It simply returns no output value.

The \texttt{return} statement can appear in a \texttt{void} function, but when it does so, there's no return value following it.

The statement just appears as: \texttt{return;}

When a \texttt{return} statement is encountered in a \texttt{void} function, it just means to stop function execution at that point.

\end{frame}

\begin{frame}[fragile]
\frametitle{Into the \texttt{void}}

A function with return type \texttt{void} doesn't need a \texttt{return} statement.

If the end of the statement block is reached without encountering \texttt{return}, it's as if there is a \texttt{return} statement there.

\begin{verbatim}
public void PrintData( )
{
    // Additional statements not shown for space reasons
    Console.WriteLine( "-- End of List --" );
}
\end{verbatim}

After the \texttt{Console.WriteLine} statement, the function will return.\\
\quad The compiler will put \texttt{return;} there if it's not explicitly written.

\end{frame}


\begin{frame}
\frametitle{\texttt{Main}}

We've been using a function in our code all along, although we never really discussed it: \texttt{static void Main()}.

Ignoring the keyword \texttt{static} for now, which we'll come to in a later lecture, let's look at the parts of this.

The return type is \texttt{void}. Thus, it returns no value.

The name is \texttt{Main}. No surprise there.

The list of formal parameters is enclosed in \texttt{( )} brackets, and for \texttt{Main}, it is empty (so there are none).

And the block of statements enclosed in \texttt{\{ \}} braces is the body.

\end{frame}

\begin{frame}[fragile]
\frametitle{Function Example: Temperature Converter}

This function converts temperatures in fahrenheit to celsius.

\begin{verbatim}
// Precondition: provided temperature is in Fahrenheit
// Postcondition: returns equivalent temp in Celsius
public double ConvertToCelsius ( double temperatureInF )
{
    return ( temperatureInF - 32 ) * 5.0d / 9.0d;
}
\end{verbatim}

We can then call this function:\\
\quad \texttt{double bodyTemp = ConvertToCelsius( 98.6 );}

\end{frame}

\begin{frame}[fragile]
\frametitle{Re-Usable Elements}
We already saw in the \texttt{struct} that we could re-use some already existing constructs like the \texttt{Date} structure.

Functions allow us to re-use groups of statements.

In this lecture we have examined the ``what'' of functions, but have not yet examined the ``why''.

In the next lecture, we will discuss the motivation for using functions.


\end{frame}


\end{document}

