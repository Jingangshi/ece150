
\documentclass[letterpaper,hide notes,xcolor={table,svgnames},pdftex]{beamer}
\def\showexamples{t}


%\usepackage[svgnames]{xcolor}

%% Demo talk
%\documentclass[letterpaper,notes=show]{beamer}

\usecolortheme{crane}%seahorse crane
\setbeamertemplate{navigation symbols}{}

\usetheme{MyPittsburgh}
%\usetheme{Frankfurt}

%\usepackage{tipa}

\usepackage{hyperref}
\usepackage{graphicx,xspace}
\usepackage[normalem]{ulem}

\newcommand\SF[1]{$\bigstar$\footnote{SF: #1}}

\usepackage{paratype}
\renewcommand*\familydefault{\sfdefault} %% Only if the base font of the document is to be sans serif
\usepackage[zerostyle=c]{newtxtt}
\usepackage[T1]{fontenc}

\newcounter{tmpnumSlide}
\newcounter{tmpnumNote}

\usepackage{xcolor}
\usepackage{tabu}
\definecolor{light-gray}{gray}{0.75}
\taburulecolor{light-gray}

% old question code
%\newcommand\question[1]{{$\bigstar$ \small \onlySlide{2}{#1}}}
% \newcommand\nquestion[1]{\ifdefined \presentationonly \textcircled{?} \fi \note{\par{\Large \textbf{?}} #1}}
% \newcommand\nanswer[1]{\note{\par{\Large \textbf{A}} #1}}


 \newcommand\mnote[1]{%
   \addtocounter{tmpnumSlide}{1}
   \ifdefined\showcues {~\tiny\fbox{\arabic{tmpnumSlide}}}\fi
   \note{\setlength{\parskip}{1ex}\addtocounter{tmpnumNote}{1}\textbf{\Large \arabic{tmpnumNote}:} {#1\par}}}

\newcommand\mmnote[1]{\note{\setlength{\parskip}{1ex}#1\par}}

%\newcommand\mnote[2][]{\ifdefined\handoutwithnotes {~\tiny\fbox{#1}}\fi
% \note{\setlength{\parskip}{1ex}\textbf{\Large #1:} #2\par}}

%\newcommand\mnote[2][]{{\tiny\fbox{#1}} \note{\setlength{\parskip}{1ex}\textbf{\Large #1:} #2\par}}

\newcommand\mquestion[2]{{~\color{red}\fbox{?}}\note{\setlength{\parskip}{1ex}\par{\Large \textbf{?}} #1} \note{\setlength{\parskip}{1ex}\par{\Large \textbf{A}} #2\par}\ifdefined \presentationonly \pause \fi}

\newcommand\blackboard[1]{%
\ifdefined   \showblackboard
  {#1}
  \else {\begin{center} \fbox{\colorbox{blue!30}{%
         \begin{minipage}{.95\linewidth}%
           \hspace{\stretch{1}} Some space intentionally left blank; done at the blackboard.%
         \end{minipage}}}\end{center}}%
         \fi%
}



%\newcommand\q{\tikz \node[thick,color=black,shape=circle]{?};}
%\newcommand\q{\ifdefined \presentationonly \textcircled{?} \fi}

\usepackage{listings}
\lstset{%
  keywordstyle=\bfseries,
  aboveskip=15pt,
  belowskip=15pt,
  captionpos=b,
  identifierstyle=\ttfamily,
  escapeinside={(*@}{@*)},
  stringstyle=\ttfamiliy,
  frame=lines,
  numbers=left, basicstyle=\scriptsize, numberstyle=\tiny, stepnumber=0, numbersep=2pt}

\usepackage{siunitx}
\newcommand\sius[1]{\num[group-separator = {,}]{#1}\si{\micro\second}}
\newcommand\sims[1]{\num[group-separator = {,}]{#1}\si{\milli\second}}
\newcommand\sins[1]{\num[group-separator = {,}]{#1}\si{\nano\second}}
\sisetup{group-separator = {,}, group-digits = true}

%% -------------------- tikz --------------------
\usepackage{tikz}
\usetikzlibrary{positioning}
\usetikzlibrary{arrows,backgrounds,automata,decorations.shapes,decorations.pathmorphing,decorations.markings,decorations.text}

\tikzstyle{place}=[circle,draw=blue!50,fill=blue!20,thick, inner sep=0pt,minimum size=6mm]
\tikzstyle{transition}=[rectangle,draw=black!50,fill=black!20,thick, inner sep=0pt,minimum size=4mm]

\tikzstyle{block}=[rectangle,draw=black, thick, inner sep=5pt]
\tikzstyle{bullet}=[circle,draw=black, fill=black, thin, inner sep=2pt]

\tikzstyle{pre}=[<-,shorten <=1pt,>=stealth',semithick]
\tikzstyle{post}=[->,shorten >=1pt,>=stealth',semithick]
\tikzstyle{bi}=[<->,shorten >=1pt,shorten <=1pt, >=stealth',semithick]

\tikzstyle{mut}=[-,>=stealth',semithick]

\tikzstyle{treereset}=[dashed,->, shorten >=1pt,>=stealth',thin]

\usepackage{ifmtarg}
\usepackage{xifthen}
\makeatletter
% new counter to now which frame it is within the sequence
\newcounter{multiframecounter}
% initialize buffer for previously used frame title
\gdef\lastframetitle{\textit{undefined}}
% new environment for a multi-frame
\newenvironment{multiframe}[1][]{%
\ifthenelse{\isempty{#1}}{%
% if no frame title was set via optional parameter,
% only increase sequence counter by 1
\addtocounter{multiframecounter}{1}%
}{%
% new frame title has been provided, thus
% reset sequence counter to 1 and buffer frame title for later use
\setcounter{multiframecounter}{1}%
\gdef\lastframetitle{#1}%
}%
% start conventional frame environment and
% automatically set frame title followed by sequence counter
\begin{frame}%
\frametitle{\lastframetitle~{\normalfont(\arabic{multiframecounter})}}%
}{%
\end{frame}%
}
\makeatother

\makeatletter
\newdimen\tu@tmpa%
\newdimen\ydiffl%
\newdimen\xdiffl%
\newcommand\ydiff[2]{%
    \coordinate (tmpnamea) at (#1);%
    \coordinate (tmpnameb) at (#2);%
    \pgfextracty{\tu@tmpa}{\pgfpointanchor{tmpnamea}{center}}%
    \pgfextracty{\ydiffl}{\pgfpointanchor{tmpnameb}{center}}%
    \advance\ydiffl by -\tu@tmpa%
}
\newcommand\xdiff[2]{%
    \coordinate (tmpnamea) at (#1);%
    \coordinate (tmpnameb) at (#2);%
    \pgfextractx{\tu@tmpa}{\pgfpointanchor{tmpnamea}{center}}%
    \pgfextractx{\xdiffl}{\pgfpointanchor{tmpnameb}{center}}%
    \advance\xdiffl by -\tu@tmpa%
}
\makeatother
\newcommand{\copyrightbox}[3][r]{%
\begin{tikzpicture}%
\node[inner sep=0pt,minimum size=2em](ciimage){#2};
\usefont{OT1}{phv}{n}{n}\fontsize{4}{4}\selectfont
\ydiff{ciimage.south}{ciimage.north}
\xdiff{ciimage.west}{ciimage.east}
\ifthenelse{\equal{#1}{r}}{%
\node[inner sep=0pt,right=1ex of ciimage.south east,anchor=north west,rotate=90]%
{\raggedleft\color{black!50}\parbox{\the\ydiffl}{\raggedright{}#3}};%
}{%
\ifthenelse{\equal{#1}{l}}{%
\node[inner sep=0pt,right=1ex of ciimage.south west,anchor=south west,rotate=90]%
{\raggedleft\color{black!50}\parbox{\the\ydiffl}{\raggedright{}#3}};%
}{%
\node[inner sep=0pt,below=1ex of ciimage.south west,anchor=north west]%
{\raggedleft\color{black!50}\parbox{\the\xdiffl}{\raggedright{}#3}};%
}
}
\end{tikzpicture}
}


%% --------------------

%\usepackage[excludeor]{everyhook}
%\PushPreHook{par}{\setbox0=\lastbox\llap{MUH}}\box0}

%\vspace*{\stretch{1}

%\setbox0=\lastbox \llap{\textbullet\enskip}\box0}

\setlength{\parskip}{\fill}

\newcommand\noskips{\setlength{\parskip}{1ex}}
\newcommand\doskips{\setlength{\parskip}{\fill}}

\newcommand\xx{\par\vspace*{\stretch{1}}\par}
\newcommand\xxs{\par\vspace*{2ex}\par}
\newcommand\tuple[1]{\langle #1 \rangle}
\newcommand\code[1]{{\sf \footnotesize #1}}
\newcommand\ex[1]{\uline{Example:} \ifdefined \presentationonly \pause \fi
  \ifdefined\showexamples#1\xspace\else{\uline{\hspace*{2cm}}}\fi}

\newcommand\ceil[1]{\lceil #1 \rceil}


\AtBeginSection[]
{
   \begin{frame}
       \frametitle{Outline}
       \tableofcontents[currentsection]
   \end{frame}
}



\pgfdeclarelayer{edgelayer}
\pgfdeclarelayer{nodelayer}
\pgfsetlayers{edgelayer,nodelayer,main}

\tikzstyle{none}=[inner sep=0pt]
\tikzstyle{rn}=[circle,fill=Red,draw=Black,line width=0.8 pt]
\tikzstyle{gn}=[circle,fill=Lime,draw=Black,line width=0.8 pt]
\tikzstyle{yn}=[circle,fill=Yellow,draw=Black,line width=0.8 pt]
\tikzstyle{empty}=[circle,fill=White,draw=Black]
\tikzstyle{bw} = [rectangle, draw, fill=blue!20, 
    text width=4em, text centered, rounded corners, minimum height=2em]
    
    \newcommand{\CcNote}[1]{% longname
	This work is licensed under the \textit{Creative Commons #1 3.0 License}.%
}
\newcommand{\CcImageBy}[1]{%
	\includegraphics[scale=#1]{creative_commons/cc_by_30.pdf}%
}
\newcommand{\CcImageSa}[1]{%
	\includegraphics[scale=#1]{creative_commons/cc_sa_30.pdf}%
}
\newcommand{\CcImageNc}[1]{%
	\includegraphics[scale=#1]{creative_commons/cc_nc_30.pdf}%
}
\newcommand{\CcGroupBySa}[2]{% zoom, gap
	\CcImageBy{#1}\hspace*{#2}\CcImageNc{#1}\hspace*{#2}\CcImageSa{#1}%
}
\newcommand{\CcLongnameByNcSa}{Attribution-NonCommercial-ShareAlike}


\newenvironment{changemargin}[1]{% 
  \begin{list}{}{% 
    \setlength{\topsep}{0pt}% 
    \setlength{\leftmargin}{#1}% 
    \setlength{\rightmargin}{1em}
    \setlength{\listparindent}{\parindent}% 
    \setlength{\itemindent}{\parindent}% 
    \setlength{\parsep}{\parskip}% 
  }% 
  \item[]}{\end{list}} 




\title{Lecture 8 --- Loops }

\author{J. Zarnett\\
\texttt{jzarnett@uwaterloo.ca}}
\institute{Department of Electrical and Computer Engineering \\
  University of Waterloo}
\date{\today}

\begin{document}

\begin{frame}
  \titlepage
  
  \begin{center}
  \small{Acknowledgments: W.D. Bishop}
  \end{center}
 \end{frame}
 
\begin{frame}
\frametitle{Loops}

Loops are another kind of control statement: \alert{iteration} statements.

Iteration: the repetition of a group of statements.

Using a loop statement in code means a block of statements is repeated for some number of iterations.

This may be a fixed number or vary based on the state of the program.

\end{frame}

\begin{frame}
\frametitle{Loops: Fixed vs. Variable}

Some examples:

A loop with fixed iterations might repeat a block of statements exactly 10 times.

A loop with a variable number of iterations might repeat a block of statements until the user enters the letter 'Q' to quit.

\end{frame}


\begin{frame}
\frametitle{Loop Terminology}

\alert{Pretest} loops evaluate one or more expressions prior to executing the statement block.

\alert{Posttest} loops execute the statement block once prior to evaluating one or more expressions.

Loops come in a few different varieties:

\begin{itemize}
	\item Counter-controlled loops increment / decrement a counter until the counter reaches a threshold
	\item Logically-controlled loops evaluate an expression and terminate when the expression is false
	\item User-controlled loops can break out of the loop anywhere within the statement block
\end{itemize}

\end{frame}

\begin{frame}
\frametitle{C\# Loop Types}
C\# provides four types of loops:

\begin{itemize}
\item The \texttt{while} loop
\item The \texttt{do-while} loop
\item The \texttt{for} loop
\item The \texttt{foreach} loop
\end{itemize}

Let's just jump right in and look at the syntax.

\end{frame}

\begin{frame}[fragile]
\frametitle{The \texttt{while} Loop}

This is the syntax for a \texttt{while} loop.

\begin{verbatim}
while ( condition ) 
{
    // Loop Body statements
}
\end{verbatim}

Like the if-statement, \textit{condition} in this is a boolean expression.

The \texttt{while} loop is a pretest loop: the condition is evaluated first.

\end{frame}

\begin{frame}
\frametitle{The \texttt{while} Loop}

Like the if-statement, \textit{condition} is evaluated and if it is true, the block of statements in the \{ \} are executed.

That block of statements is referred to as the \alert{loop body}.

If \textit{condition} is false, the statements of the loop body are not executed.

It may happen that the body of the loop never executes.



\end{frame}

\begin{frame}
\frametitle{The \texttt{while} Loop: Repetition}

What makes the \texttt{while} different from \texttt{if} is the \alert{repetition}.

At the end of the statement body (the \} character), control goes back to the \texttt{while} statement.

The condition is evaluated again.\\
	\quad Same applies: if true, the loop body is executed.
	
This continues until the condition evaluates to false.


\end{frame}


\begin{frame}[fragile]
\frametitle{The \texttt{while} Loop: Countdown}

Here's another example of a \texttt{while} loop:

\begin{verbatim}

int countdown = 10;

while ( countdown > 0 ) 
{
    Console.WriteLine( countdown );
    countdown--;
}

\end{verbatim}

[Demo: output of this code.]

\end{frame}

\begin{frame}
\frametitle{The Infinite Loop}

What happens if we forget the \texttt{countdown{-}{-};} statement in that loop?

The variable \texttt{countdown} remains at 10 and the \texttt{while} condition will always evaluate to true.

This will go on indefinitely (or until you get frustrated and close the program). The term for this is an \alert{infinite loop}.

\end{frame}

\begin{frame}
\frametitle{The Infinite Loop}

An infinite loop is very often an error condition.

Most programs should terminate at some point.

In some circumstances, however, the infinite loop is intended: the program should never terminate.

Example: the software in your router. On boot up it starts running its program, and continues, never ending (until the plug is pulled).

To create an infinite loop, write \texttt{while( true )}.

\end{frame}

\begin{frame}
\frametitle{The \texttt{break} Statement}

There are two special statements that can be written in the loop body that control the flow of execution.

The first of these is the \texttt{break} statement.

Yes, this is the same keyword as in the \texttt{switch} statement.\\
\quad The context indicates it means something slightly different.

When the break statement executes, it means ``exit the loop now''.

\end{frame}

\begin{frame}
\frametitle{Proper use of \texttt{break}}

A \texttt{break} statement may be used to jump out of a loop, even an infinite one, in the middle of a statement block.

If possible, \texttt{break} statements should be avoided as they can result in code that is more difficult to debug.

The rule of thumb is that the \texttt{break} statement should be used if the code is clearer with it than it would be without it.

Multiple \texttt{break} statements can exist in a loop if multiple conditions for exiting the loop need to be evaluated.

\end{frame}

\begin{frame}[fragile]
\frametitle{While loop with Break}

\begin{verbatim}
static void Main( )
{
    int counter = 0;

    while( counter < 500 )
    {
        counter++;
        if( counter > 4 )
        {
            break;
        }
        Console.WriteLine( counter );
    }
}
\end{verbatim}

[Demo: output of this program]

\end{frame}

\begin{frame}
\frametitle{Proper Use of Break}
A \texttt{break} statement completely ends the loop, no matter if the loop condition is true or not.

If you write an infinite loop (\texttt{while( true )}) one way to use this properly is to have a condition inside that uses \texttt{break}.

\end{frame}


\begin{frame}[fragile]
\frametitle{While loop with User Input}

\begin{verbatim}
static void Main ( )
{
   Console.WriteLine( "Enter a negative number to exit." );
   int number = 0;
    
   while (number >= 0) 
   {
       Console.Write( "Enter a number: " );
       number = int.Parse( Console.ReadLine() );
   } // End of loop
    
   Console.WriteLine( "Negative number entered." );
}    

\end{verbatim}
\end{frame}

\begin{frame}[fragile]
\frametitle{While loop with User Input \& Break}

Let's rewrite this with \texttt{break}:

\begin{verbatim}
static void Main ( )
{
   Console.WriteLine( "Enter a negative number to exit." );
   int number = 0;
    
   while (true) 
   {
       Console.Write( "Enter a number: " );
       number = int.Parse( Console.ReadLine() );
       if ( number < 0 )
       {
           break;
       }
    }
    Console.WriteLine( "Negative number entered." );
}    

\end{verbatim}
\end{frame}

\begin{frame}
\frametitle{The \texttt{continue} Statement}
The other loop body statement that controls execution is the \texttt{continue} statement.

The \texttt{continue} statement works a lot like the \texttt{break} statement, except instead of exiting the loop, it means ``go back to the start of the loop''.

In the \texttt{while} loop, the condition is tested, and if it's still true, the next iteration of the loop executes.

\end{frame}

\begin{frame}[fragile]
\frametitle{While loop with Continue}

\begin{verbatim}
static void Main( )
{
    int counter = 0;

    while( counter < 10 )
    {
        counter++;
        if( counter == 4 )
        {
            continue;
        }
        Console.WriteLine( counter );
    }
}s
\end{verbatim}

[Demo: output of this program]

\end{frame}

\begin{frame}
\frametitle{Use of Continue}
Like the \texttt{break} statement, \texttt{continue} should be avoided if possible, as they can result in code that is more difficult to debug.

Multiple \texttt{continue} statements can exist in a loop if multiple conditions for going to the next iteration of the loop exist.

If the loop condition is no longer true, use of \texttt{continue} takes us to testing the condition; it will evaluate to false, and the loop ends.

In this way, use of \texttt{continue} may have the same outcome as \texttt{break}, though it gets there by a different path.

\end{frame}

\begin{frame}[fragile]
\frametitle{The \texttt{do-while} Loop}
A variant of the \texttt{while} loop that remains in the language for historical reasons is the \texttt{do-while} loop.

Its syntax is a lot like the \texttt{while} loop:

\begin{verbatim}

do 
{
    // Loop body

} while ( condition );

\end{verbatim}

Note the semicolon that appears after the condition's closing bracket.

\end{frame}

\begin{frame}
\frametitle{The \texttt{do-while} Loop}

Some important things to observe about the \texttt{do-while} loop.

The loop body is preceded by \texttt{do} to indicate the start of the loop.

The condition is checked at the \underline{end} of the loop body, not beginning.\\
\quad This is a posttest loop.

Key observation: the loop body will execute at least once, even if the condition is false.

\end{frame}

\begin{frame}[fragile]
\frametitle{While vs. Do-While}

Let's compare this while loop:

\begin{verbatim}
int count = int.Parse( Console.ReadLine() );
while ( count > 0 ) 
{
    Console.WriteLine( count );
    count--;
}
\end{verbatim}

...with this one:

\begin{verbatim}
int count = int.Parse( Console.ReadLine() );
do 
{
    Console.WriteLine( count );
    count--;
} while ( count > 0 );
\end{verbatim}


\end{frame}

\begin{frame}
\frametitle{The \texttt{do-while} Loop}

It is still possible to write infinite loops with \texttt{do-while}.

Similarly, the \texttt{break} and \texttt{continue} statements work the same way.

The use of \texttt{do-while} is not recommended as it is really only in the language for historical reasons; use the \texttt{while} loop instead.

\end{frame}

\end{document}

