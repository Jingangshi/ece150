
\documentclass[letterpaper,hide notes,xcolor={table,svgnames},pdftex]{beamer}
\def\showexamples{t}


%\usepackage[svgnames]{xcolor}

%% Demo talk
%\documentclass[letterpaper,notes=show]{beamer}

\usecolortheme{crane}%seahorse crane
\setbeamertemplate{navigation symbols}{}

\usetheme{MyPittsburgh}
%\usetheme{Frankfurt}

%\usepackage{tipa}

\usepackage{hyperref}
\usepackage{graphicx,xspace}
\usepackage[normalem]{ulem}

\newcommand\SF[1]{$\bigstar$\footnote{SF: #1}}

\usepackage{paratype}
\renewcommand*\familydefault{\sfdefault} %% Only if the base font of the document is to be sans serif
\usepackage[zerostyle=c]{newtxtt}
\usepackage[T1]{fontenc}

\newcounter{tmpnumSlide}
\newcounter{tmpnumNote}

\usepackage{xcolor}
\usepackage{tabu}
\definecolor{light-gray}{gray}{0.75}
\taburulecolor{light-gray}

% old question code
%\newcommand\question[1]{{$\bigstar$ \small \onlySlide{2}{#1}}}
% \newcommand\nquestion[1]{\ifdefined \presentationonly \textcircled{?} \fi \note{\par{\Large \textbf{?}} #1}}
% \newcommand\nanswer[1]{\note{\par{\Large \textbf{A}} #1}}


 \newcommand\mnote[1]{%
   \addtocounter{tmpnumSlide}{1}
   \ifdefined\showcues {~\tiny\fbox{\arabic{tmpnumSlide}}}\fi
   \note{\setlength{\parskip}{1ex}\addtocounter{tmpnumNote}{1}\textbf{\Large \arabic{tmpnumNote}:} {#1\par}}}

\newcommand\mmnote[1]{\note{\setlength{\parskip}{1ex}#1\par}}

%\newcommand\mnote[2][]{\ifdefined\handoutwithnotes {~\tiny\fbox{#1}}\fi
% \note{\setlength{\parskip}{1ex}\textbf{\Large #1:} #2\par}}

%\newcommand\mnote[2][]{{\tiny\fbox{#1}} \note{\setlength{\parskip}{1ex}\textbf{\Large #1:} #2\par}}

\newcommand\mquestion[2]{{~\color{red}\fbox{?}}\note{\setlength{\parskip}{1ex}\par{\Large \textbf{?}} #1} \note{\setlength{\parskip}{1ex}\par{\Large \textbf{A}} #2\par}\ifdefined \presentationonly \pause \fi}

\newcommand\blackboard[1]{%
\ifdefined   \showblackboard
  {#1}
  \else {\begin{center} \fbox{\colorbox{blue!30}{%
         \begin{minipage}{.95\linewidth}%
           \hspace{\stretch{1}} Some space intentionally left blank; done at the blackboard.%
         \end{minipage}}}\end{center}}%
         \fi%
}



%\newcommand\q{\tikz \node[thick,color=black,shape=circle]{?};}
%\newcommand\q{\ifdefined \presentationonly \textcircled{?} \fi}

\usepackage{listings}
\lstset{%
  keywordstyle=\bfseries,
  aboveskip=15pt,
  belowskip=15pt,
  captionpos=b,
  identifierstyle=\ttfamily,
  escapeinside={(*@}{@*)},
  stringstyle=\ttfamiliy,
  frame=lines,
  numbers=left, basicstyle=\scriptsize, numberstyle=\tiny, stepnumber=0, numbersep=2pt}

\usepackage{siunitx}
\newcommand\sius[1]{\num[group-separator = {,}]{#1}\si{\micro\second}}
\newcommand\sims[1]{\num[group-separator = {,}]{#1}\si{\milli\second}}
\newcommand\sins[1]{\num[group-separator = {,}]{#1}\si{\nano\second}}
\sisetup{group-separator = {,}, group-digits = true}

%% -------------------- tikz --------------------
\usepackage{tikz}
\usetikzlibrary{positioning}
\usetikzlibrary{arrows,backgrounds,automata,decorations.shapes,decorations.pathmorphing,decorations.markings,decorations.text}

\tikzstyle{place}=[circle,draw=blue!50,fill=blue!20,thick, inner sep=0pt,minimum size=6mm]
\tikzstyle{transition}=[rectangle,draw=black!50,fill=black!20,thick, inner sep=0pt,minimum size=4mm]

\tikzstyle{block}=[rectangle,draw=black, thick, inner sep=5pt]
\tikzstyle{bullet}=[circle,draw=black, fill=black, thin, inner sep=2pt]

\tikzstyle{pre}=[<-,shorten <=1pt,>=stealth',semithick]
\tikzstyle{post}=[->,shorten >=1pt,>=stealth',semithick]
\tikzstyle{bi}=[<->,shorten >=1pt,shorten <=1pt, >=stealth',semithick]

\tikzstyle{mut}=[-,>=stealth',semithick]

\tikzstyle{treereset}=[dashed,->, shorten >=1pt,>=stealth',thin]

\usepackage{ifmtarg}
\usepackage{xifthen}
\makeatletter
% new counter to now which frame it is within the sequence
\newcounter{multiframecounter}
% initialize buffer for previously used frame title
\gdef\lastframetitle{\textit{undefined}}
% new environment for a multi-frame
\newenvironment{multiframe}[1][]{%
\ifthenelse{\isempty{#1}}{%
% if no frame title was set via optional parameter,
% only increase sequence counter by 1
\addtocounter{multiframecounter}{1}%
}{%
% new frame title has been provided, thus
% reset sequence counter to 1 and buffer frame title for later use
\setcounter{multiframecounter}{1}%
\gdef\lastframetitle{#1}%
}%
% start conventional frame environment and
% automatically set frame title followed by sequence counter
\begin{frame}%
\frametitle{\lastframetitle~{\normalfont(\arabic{multiframecounter})}}%
}{%
\end{frame}%
}
\makeatother

\makeatletter
\newdimen\tu@tmpa%
\newdimen\ydiffl%
\newdimen\xdiffl%
\newcommand\ydiff[2]{%
    \coordinate (tmpnamea) at (#1);%
    \coordinate (tmpnameb) at (#2);%
    \pgfextracty{\tu@tmpa}{\pgfpointanchor{tmpnamea}{center}}%
    \pgfextracty{\ydiffl}{\pgfpointanchor{tmpnameb}{center}}%
    \advance\ydiffl by -\tu@tmpa%
}
\newcommand\xdiff[2]{%
    \coordinate (tmpnamea) at (#1);%
    \coordinate (tmpnameb) at (#2);%
    \pgfextractx{\tu@tmpa}{\pgfpointanchor{tmpnamea}{center}}%
    \pgfextractx{\xdiffl}{\pgfpointanchor{tmpnameb}{center}}%
    \advance\xdiffl by -\tu@tmpa%
}
\makeatother
\newcommand{\copyrightbox}[3][r]{%
\begin{tikzpicture}%
\node[inner sep=0pt,minimum size=2em](ciimage){#2};
\usefont{OT1}{phv}{n}{n}\fontsize{4}{4}\selectfont
\ydiff{ciimage.south}{ciimage.north}
\xdiff{ciimage.west}{ciimage.east}
\ifthenelse{\equal{#1}{r}}{%
\node[inner sep=0pt,right=1ex of ciimage.south east,anchor=north west,rotate=90]%
{\raggedleft\color{black!50}\parbox{\the\ydiffl}{\raggedright{}#3}};%
}{%
\ifthenelse{\equal{#1}{l}}{%
\node[inner sep=0pt,right=1ex of ciimage.south west,anchor=south west,rotate=90]%
{\raggedleft\color{black!50}\parbox{\the\ydiffl}{\raggedright{}#3}};%
}{%
\node[inner sep=0pt,below=1ex of ciimage.south west,anchor=north west]%
{\raggedleft\color{black!50}\parbox{\the\xdiffl}{\raggedright{}#3}};%
}
}
\end{tikzpicture}
}


%% --------------------

%\usepackage[excludeor]{everyhook}
%\PushPreHook{par}{\setbox0=\lastbox\llap{MUH}}\box0}

%\vspace*{\stretch{1}

%\setbox0=\lastbox \llap{\textbullet\enskip}\box0}

\setlength{\parskip}{\fill}

\newcommand\noskips{\setlength{\parskip}{1ex}}
\newcommand\doskips{\setlength{\parskip}{\fill}}

\newcommand\xx{\par\vspace*{\stretch{1}}\par}
\newcommand\xxs{\par\vspace*{2ex}\par}
\newcommand\tuple[1]{\langle #1 \rangle}
\newcommand\code[1]{{\sf \footnotesize #1}}
\newcommand\ex[1]{\uline{Example:} \ifdefined \presentationonly \pause \fi
  \ifdefined\showexamples#1\xspace\else{\uline{\hspace*{2cm}}}\fi}

\newcommand\ceil[1]{\lceil #1 \rceil}


\AtBeginSection[]
{
   \begin{frame}
       \frametitle{Outline}
       \tableofcontents[currentsection]
   \end{frame}
}



\pgfdeclarelayer{edgelayer}
\pgfdeclarelayer{nodelayer}
\pgfsetlayers{edgelayer,nodelayer,main}

\tikzstyle{none}=[inner sep=0pt]
\tikzstyle{rn}=[circle,fill=Red,draw=Black,line width=0.8 pt]
\tikzstyle{gn}=[circle,fill=Lime,draw=Black,line width=0.8 pt]
\tikzstyle{yn}=[circle,fill=Yellow,draw=Black,line width=0.8 pt]
\tikzstyle{empty}=[circle,fill=White,draw=Black]
\tikzstyle{bw} = [rectangle, draw, fill=blue!20, 
    text width=4em, text centered, rounded corners, minimum height=2em]
    
    \newcommand{\CcNote}[1]{% longname
	This work is licensed under the \textit{Creative Commons #1 3.0 License}.%
}
\newcommand{\CcImageBy}[1]{%
	\includegraphics[scale=#1]{creative_commons/cc_by_30.pdf}%
}
\newcommand{\CcImageSa}[1]{%
	\includegraphics[scale=#1]{creative_commons/cc_sa_30.pdf}%
}
\newcommand{\CcImageNc}[1]{%
	\includegraphics[scale=#1]{creative_commons/cc_nc_30.pdf}%
}
\newcommand{\CcGroupBySa}[2]{% zoom, gap
	\CcImageBy{#1}\hspace*{#2}\CcImageNc{#1}\hspace*{#2}\CcImageSa{#1}%
}
\newcommand{\CcLongnameByNcSa}{Attribution-NonCommercial-ShareAlike}


\newenvironment{changemargin}[1]{% 
  \begin{list}{}{% 
    \setlength{\topsep}{0pt}% 
    \setlength{\leftmargin}{#1}% 
    \setlength{\rightmargin}{1em}
    \setlength{\listparindent}{\parindent}% 
    \setlength{\itemindent}{\parindent}% 
    \setlength{\parsep}{\parskip}% 
  }% 
  \item[]}{\end{list}} 




\title{Lecture 7 --- Selection Statements}

\author{J. Zarnett\\
\texttt{jzarnett@uwaterloo.ca}}
\institute{Department of Electrical and Computer Engineering \\
  University of Waterloo}
\date{\today}

\begin{document}

\begin{frame}
  \titlepage
  
  \begin{center}
  \small{Acknowledgments: W.D. Bishop}
  \end{center}
 \end{frame}
 
\begin{frame}
\frametitle{Selection Statements}
Thus far, our programs execute every statement, sequentially, from top to bottom.

Sometimes we have to make a decision about what to do next.

\alert{Selection Statements} allow a program to decide what instructions to execute next, based on the current state of the program.

Selection statements are an example of \alert{control statements}.

\end{frame}

\begin{frame}
\frametitle{Types of Selection Statements}

There are three kinds of selection statement in C++:

\begin{enumerate}
	\item \texttt{if}
	\item \texttt{if-else}
	\item \texttt{switch}
\end{enumerate}

We will examine each of these.

\end{frame}

\part{The \texttt{if} Statement}
\begin{frame}\partpage\end{frame}

\begin{frame}[fragile]
\frametitle{The \texttt{if} Statement}

The simplest of the three is the \texttt{if} statement.

The basic format of this statement is as follows:

\begin{verbatim}
if ( condition ) {
    // Statement Block
}
\end{verbatim}

If the \textit{condition} is true, then the statement block will execute.

If the \textit{condition} is false, the statement block is skipped.\\
\quad The statements in that block are not executed.

\end{frame}

\begin{frame}[fragile]
\frametitle{Use of the if-Statement}

\begin{verbatim}
#include <iostream>

using namespace std;

int main()
{
    int x;
    cin >> x;
    
    if ( x >= 25 ) {
        cout << "Condition is true." << endl;
    }
    cout << "Program Finished." << endl;    
    return 0;
}
\end{verbatim}

[In-Class Demo: the output of this program]

\end{frame}

\begin{frame}
\frametitle{The \texttt{if} Statement}
A condition may be a boolean variable, or it can be an expression that evaluates to \texttt{true} or \texttt{false}.

It may be a simple condition expression (\texttt{ x > 0 }) or a more complex one (\texttt{ y < 100 \&\& z > 0 }).

Remember that the condition will be short-circuit evaluated.

A more complicated example follows on the next slide:

\end{frame}

\begin{frame}[fragile]
\frametitle{Use of the if-Statement}

{\scriptsize
\begin{verbatim}
#include <iostream>
using namespace std;

int main() {
    int output = 0;
    int input1;
    int input2;
    
    cout << "Enter Input 1: ";
    cin >> input1;
    cout << "Enter Input 2: ";
    
    if ( (input1 == 0 && input2 == 1) || ( input1 == 1 && input2 == 0 ) ) {
            output = 1; 
    }
    cout << "Output = " << output << endl;
    return 0;    
}
\end{verbatim}
}

[In-Class Demo: the output of this program]

\end{frame}

\begin{frame}
\frametitle{if Statement Pitfall}
There is a potential pitfall to the if statement. Use of the \{ \} braces for the statement block following the if statement is technically optional.

This is not a syntax error:\\
\texttt{if (x > 0)}\\
\quad\texttt{y = 1;}

The statement \texttt{y = 1;} is executed only if the condition \texttt{x > 0} is true.

If later we edit this code and add the following:

\texttt{if (x > 0)}\\
\quad\texttt{y = 1;}\\
\quad\texttt{z = 2;}

This is a potential source of error, but why?

\end{frame}

\begin{frame}
\frametitle{if Statement Pitfall}

\texttt{if (x > 0)}\\
\quad\texttt{y = 1;}\\
\quad\texttt{z = 2;}

\texttt{z = 2;} is executed regardless of whether \texttt{x} is greater than zero.

That might be what you intended, but it might also be an error.

Solution: always use the \{ and \} braces when writing an if-statement.

\end{frame}

\begin{frame}[fragile]
\frametitle{The \texttt{if-else} Statement}

We can build on the if statement with the \texttt{else} keyword.

\begin{verbatim}
if ( condition ) {
    // Statement Block 1
} else {
    // Statement Block 2
}
\end{verbatim}

If the \textit{condition} is true, then statement block 1 will execute; statement block 2 will not execute.

If the \textit{condition} is false, then statement block 2 will execute; statement block 1 will not execute.


\end{frame}

\begin{frame}[fragile]
\frametitle{The \texttt{if-else} Statement}

\begin{verbatim}

cout << "Enter your age: ";
int age;
cin >> age;

if ( age >= 16 ) {
 cout << "You may take the driving test." << endl;
} else {
 cout << "You are not old enough. Sorry." << endl;
}
\end{verbatim}

\end{frame}


\begin{frame}[fragile]
\frametitle{Use of the if-Statement}

{\scriptsize
\begin{verbatim}
int main() {
    int output = 0;
    int input1;
    int input2;
    
    cout << "Enter Input 1: ";
    cin >> input1;
    cout << "Enter Input 2: ";
    
    if ( input1 == 0 ) {
        output = 1; 
    }

    if ( input1 != 0 && input2 == 1 ) {				
        output = 2; 			
    }				
    cout << "Output = " << output << endl;
    return 0;    
}
\end{verbatim}
}

[In-Class Demo: the output of this program]

\end{frame}

\begin{frame}
\frametitle{Comments on the Previous Program}

You may have noticed some redundancy in the previous program:


Two if conditions, one with \texttt{input == 0} and one with \texttt{input != 0}.

We have the else-if statement to deal with this situation.

\end{frame}

\begin{frame}[fragile]
\frametitle{Use of the if-Statement}

{\scriptsize
\begin{verbatim}
int main() {
    int output = 0;
    int input1;
    int input2;
    
    cout << "Enter Input 1: ";
    cin >> input1;
    cout << "Enter Input 2: ";
    
    if ( input1 == 0 ) {
        output = 1; 
    } else if ( input2 == 1 ) {				
        output = 2; 			
    }				
    cout << "Output = " << output << endl;
    return 0;    
}
\end{verbatim}
}

[In-Class Demo: the output of this program]

\end{frame}

\begin{frame}
\frametitle{Use of if, else if, else}

We always have to start with an if statement.

Zero or more ``else if'' statements can be added on.

At the end, we may optionally put the else statement.

What if some of the conditions are the same?

\end{frame}

\begin{frame}[fragile]
\frametitle{Mutually Exclusive If/Else-If}

{\scriptsize
\begin{verbatim}
int main() {
    int output = 0;
    int input1;
    int input2;
    
    cout << "Enter Input 1: ";
    cin >> input1;
    cout << "Enter Input 2: ";
    
    if ( input1 == 0 ) {
        output = 1; 
    } else if ( input1 == 0 ) {				
        output = 2; 			
    }				
    cout << "Output = " << output << endl;
    return 0;    
}
\end{verbatim}
}

[In-Class Demo: the output of this program]

\end{frame}

\begin{frame}
\frametitle{Mutually Exclusive}

Only one of the blocks will execute; they are all mutually exclusive.\\
\quad In fact, the condition of the second block will not be evaluated.\\
\quad Another example of short-circuit evaluation.

In the previous slide, we had two checks of \texttt{input1 == 0}.

The first one encountered resolves to true and that block executed.\\
\quad \texttt{output} receives a value of 1.

The next statement executed is the \texttt{Console.WriteLine} statement.

\end{frame}

\begin{frame}[fragile]
\frametitle{Nested if Statements}
It is certainly permitted to have \alert{nested} if statements.

\begin{verbatim}
if ( x > 0 ) {
    if ( y < 100 ) {
        output = 7;
    } else {
        output = 10;
    }
}
\end{verbatim}

Writing a condition in this way may be clearer than having a lot of else if statements.

There is no effective limit on how many nested if statements you can have, but sometimes it is sensible to combine them for clarity.

\end{frame}

\part{The \texttt{switch} Statement}
\begin{frame}\partpage\end{frame}

\begin{frame}[fragile]
\frametitle{The \texttt{switch} Statement}

The switch statement evaluates a single variable against a large range of alternatives and selects which statement block to execute.

\begin{verbatim}
switch ( selector )
{
    case label1:
        // Statement block 1
        break;
    case label2:
        // Statement block 2
        break;
    default:
        // Default statement block
        break;
}
\end{verbatim}

There can be as many cases as we like.

\end{frame}

\begin{frame}
\frametitle{The \texttt{switch} Statement}

The \textit{selector} must be an expression that evaluates to one of: \\
\quad \texttt{bool}, \texttt{char}, \texttt{int}, or \texttt{string}.


The value of the selector is compared to whatever comes after the keyword \texttt{case} (e.g., \texttt{label1}).

If they are equal, that statement block is executed.

The statement \texttt{break;} is used to indicate the end of that option's statement block.

If none of the case options match the selector, the \texttt{default} statement block is executed.

\end{frame}

\begin{frame}[fragile]
\frametitle{The \texttt{switch} Example}

{\scriptsize
\begin{verbatim}
char keystroke;
cin >> keystroke;

switch( keystroke )
{
    case 'A':
        Console.Write( "1" );
        break;
    case 'B':
        Console.Write( "2" );
        break;
    case 'C':
        Console.Write( "3" );
        break;
    default:
        Console.Write( "0" );
        break;
}
\end{verbatim}
}

\end{frame}

\begin{frame}
\frametitle{Analysis of Previous Example}

It would be an error if we had two cases labelled with \texttt{'A'}.

Like the \texttt{if} statement, the cases are mutually exclusive.\\
\quad Although for \texttt{switch}, this is enforced by the compiler.

The \texttt{break;} statement is needed to indicate the end of an option. 

The \texttt{default} block executes if the input didn't match \texttt{'A'}, \texttt{'B'}, or \texttt{'C'}.

\end{frame}

\begin{frame}
\frametitle{The Default Block}
The \texttt{default} block executes if the input matches none of the labels.

However, the \texttt{default} block is optional; it doesn't have to appear.

If it is not present, and the input does not match any of the labels, none of the blocks will execute. In other words: nothing happens.

\end{frame}


\begin{frame}[fragile]
\frametitle{Multiple-Case Example}

To avoid having to copy and paste code, you can associate a block of statements with multiple labels.

\begin{verbatim}
char keystroke = (char) Console.ReadKey( );
switch( keystroke )
{
    case 'a':
    case 'A':
        Console.Write( "1" );
        break;
}
\end{verbatim}
\end{frame}

\begin{frame}[fragile]
\frametitle{Remember to Take a \texttt{break}}

This, however, is valid syntax, but a source of many errors in C++:

\begin{verbatim}
char keystroke = (char) Console.ReadKey( );
switch( keystroke )
{
    case 'a':
        Console.Write( "7" );
    case 'A':
        Console.Write( "1" );
        break;
}
\end{verbatim}

The \texttt{break;} statement is missing after case \texttt{'a'}.

This would mean that the output would be ``71''.

\end{frame}

\begin{frame}[fragile]
\frametitle{Remember to Take a \texttt{break}}

The concept of going on from one case to the one below is called ``fall through''.

The designers of C\#, for example, recognized that this was a common source of programmer error.

They therefore chose to explicitly forbid it. But it is a problem in C++.

\end{frame}

\begin{frame}
\frametitle{\texttt{switch} vs \texttt{if}}
The \texttt{switch} and \texttt{if} statements are two different ways to represent the same idea: selection statements.

The \texttt{switch} statement can be rewritten as an \texttt{if-else} statement.

Let's see an example of \texttt{switch} and its equivalent \texttt{if}.


\end{frame}

\begin{frame}[fragile]
\frametitle{Making the \texttt{switch}}

{\scriptsize
\begin{verbatim}
int switchExpression = 3;
switch (switchExpression)
{
    case 0:
    case 1:
        cout << "Case 0 or 1" << endl;
        break;
    case 2:
        cout << "Case 2" << endl;
        break;
     // 7 - 4 in the following line evaluates to 3. 
     case 7 - 4:
         cout << "Case 3" << endl;
         break;   
     default:
         cout << "Default case (optional)" << endl;
         break;
}
\end{verbatim}
}

\end{frame}

\begin{frame}[fragile]
\frametitle{Making the \texttt{switch}}

{\scriptsize
\begin{verbatim}
int switchExpression = 3;

if ( switchExpression == 0 || switchExpression == 1 ) {
    cout << "Case 0 or 1" << endl;
} else if ( switchExpression == 2 ) {
    cout << "Case 2" << endl;
} else if ( switchExpression == (7 - 4) ) {
    cout << "Case 3" << endl;
} else {
    cout << "Default case (optional)" << endl;
}
\end{verbatim}
}

\end{frame}

\begin{frame}
\frametitle{When to Use \texttt{if} vs \texttt{switch}}

You can use the \texttt{if-else} statement to replace a \texttt{switch} statement so the \texttt{if} statement is always applicable.

The \texttt{switch} statement may be better when repeatedly checking the value of a single variable.

\end{frame} 


\end{document}

