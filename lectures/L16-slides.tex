
\documentclass[letterpaper,hide notes,xcolor={table,svgnames},pdftex]{beamer}
\def\showexamples{t}


%\usepackage[svgnames]{xcolor}

%% Demo talk
%\documentclass[letterpaper,notes=show]{beamer}

\usecolortheme{crane}%seahorse crane
\setbeamertemplate{navigation symbols}{}

\usetheme{MyPittsburgh}
%\usetheme{Frankfurt}

%\usepackage{tipa}

\usepackage{hyperref}
\usepackage{graphicx,xspace}
\usepackage[normalem]{ulem}

\newcommand\SF[1]{$\bigstar$\footnote{SF: #1}}

\usepackage{paratype}
\renewcommand*\familydefault{\sfdefault} %% Only if the base font of the document is to be sans serif
\usepackage[zerostyle=c]{newtxtt}
\usepackage[T1]{fontenc}

\newcounter{tmpnumSlide}
\newcounter{tmpnumNote}

\usepackage{xcolor}
\usepackage{tabu}
\definecolor{light-gray}{gray}{0.75}
\taburulecolor{light-gray}

% old question code
%\newcommand\question[1]{{$\bigstar$ \small \onlySlide{2}{#1}}}
% \newcommand\nquestion[1]{\ifdefined \presentationonly \textcircled{?} \fi \note{\par{\Large \textbf{?}} #1}}
% \newcommand\nanswer[1]{\note{\par{\Large \textbf{A}} #1}}


 \newcommand\mnote[1]{%
   \addtocounter{tmpnumSlide}{1}
   \ifdefined\showcues {~\tiny\fbox{\arabic{tmpnumSlide}}}\fi
   \note{\setlength{\parskip}{1ex}\addtocounter{tmpnumNote}{1}\textbf{\Large \arabic{tmpnumNote}:} {#1\par}}}

\newcommand\mmnote[1]{\note{\setlength{\parskip}{1ex}#1\par}}

%\newcommand\mnote[2][]{\ifdefined\handoutwithnotes {~\tiny\fbox{#1}}\fi
% \note{\setlength{\parskip}{1ex}\textbf{\Large #1:} #2\par}}

%\newcommand\mnote[2][]{{\tiny\fbox{#1}} \note{\setlength{\parskip}{1ex}\textbf{\Large #1:} #2\par}}

\newcommand\mquestion[2]{{~\color{red}\fbox{?}}\note{\setlength{\parskip}{1ex}\par{\Large \textbf{?}} #1} \note{\setlength{\parskip}{1ex}\par{\Large \textbf{A}} #2\par}\ifdefined \presentationonly \pause \fi}

\newcommand\blackboard[1]{%
\ifdefined   \showblackboard
  {#1}
  \else {\begin{center} \fbox{\colorbox{blue!30}{%
         \begin{minipage}{.95\linewidth}%
           \hspace{\stretch{1}} Some space intentionally left blank; done at the blackboard.%
         \end{minipage}}}\end{center}}%
         \fi%
}



%\newcommand\q{\tikz \node[thick,color=black,shape=circle]{?};}
%\newcommand\q{\ifdefined \presentationonly \textcircled{?} \fi}

\usepackage{listings}
\lstset{%
  keywordstyle=\bfseries,
  aboveskip=15pt,
  belowskip=15pt,
  captionpos=b,
  identifierstyle=\ttfamily,
  escapeinside={(*@}{@*)},
  stringstyle=\ttfamiliy,
  frame=lines,
  numbers=left, basicstyle=\scriptsize, numberstyle=\tiny, stepnumber=0, numbersep=2pt}

\usepackage{siunitx}
\newcommand\sius[1]{\num[group-separator = {,}]{#1}\si{\micro\second}}
\newcommand\sims[1]{\num[group-separator = {,}]{#1}\si{\milli\second}}
\newcommand\sins[1]{\num[group-separator = {,}]{#1}\si{\nano\second}}
\sisetup{group-separator = {,}, group-digits = true}

%% -------------------- tikz --------------------
\usepackage{tikz}
\usetikzlibrary{positioning}
\usetikzlibrary{arrows,backgrounds,automata,decorations.shapes,decorations.pathmorphing,decorations.markings,decorations.text}

\tikzstyle{place}=[circle,draw=blue!50,fill=blue!20,thick, inner sep=0pt,minimum size=6mm]
\tikzstyle{transition}=[rectangle,draw=black!50,fill=black!20,thick, inner sep=0pt,minimum size=4mm]

\tikzstyle{block}=[rectangle,draw=black, thick, inner sep=5pt]
\tikzstyle{bullet}=[circle,draw=black, fill=black, thin, inner sep=2pt]

\tikzstyle{pre}=[<-,shorten <=1pt,>=stealth',semithick]
\tikzstyle{post}=[->,shorten >=1pt,>=stealth',semithick]
\tikzstyle{bi}=[<->,shorten >=1pt,shorten <=1pt, >=stealth',semithick]

\tikzstyle{mut}=[-,>=stealth',semithick]

\tikzstyle{treereset}=[dashed,->, shorten >=1pt,>=stealth',thin]

\usepackage{ifmtarg}
\usepackage{xifthen}
\makeatletter
% new counter to now which frame it is within the sequence
\newcounter{multiframecounter}
% initialize buffer for previously used frame title
\gdef\lastframetitle{\textit{undefined}}
% new environment for a multi-frame
\newenvironment{multiframe}[1][]{%
\ifthenelse{\isempty{#1}}{%
% if no frame title was set via optional parameter,
% only increase sequence counter by 1
\addtocounter{multiframecounter}{1}%
}{%
% new frame title has been provided, thus
% reset sequence counter to 1 and buffer frame title for later use
\setcounter{multiframecounter}{1}%
\gdef\lastframetitle{#1}%
}%
% start conventional frame environment and
% automatically set frame title followed by sequence counter
\begin{frame}%
\frametitle{\lastframetitle~{\normalfont(\arabic{multiframecounter})}}%
}{%
\end{frame}%
}
\makeatother

\makeatletter
\newdimen\tu@tmpa%
\newdimen\ydiffl%
\newdimen\xdiffl%
\newcommand\ydiff[2]{%
    \coordinate (tmpnamea) at (#1);%
    \coordinate (tmpnameb) at (#2);%
    \pgfextracty{\tu@tmpa}{\pgfpointanchor{tmpnamea}{center}}%
    \pgfextracty{\ydiffl}{\pgfpointanchor{tmpnameb}{center}}%
    \advance\ydiffl by -\tu@tmpa%
}
\newcommand\xdiff[2]{%
    \coordinate (tmpnamea) at (#1);%
    \coordinate (tmpnameb) at (#2);%
    \pgfextractx{\tu@tmpa}{\pgfpointanchor{tmpnamea}{center}}%
    \pgfextractx{\xdiffl}{\pgfpointanchor{tmpnameb}{center}}%
    \advance\xdiffl by -\tu@tmpa%
}
\makeatother
\newcommand{\copyrightbox}[3][r]{%
\begin{tikzpicture}%
\node[inner sep=0pt,minimum size=2em](ciimage){#2};
\usefont{OT1}{phv}{n}{n}\fontsize{4}{4}\selectfont
\ydiff{ciimage.south}{ciimage.north}
\xdiff{ciimage.west}{ciimage.east}
\ifthenelse{\equal{#1}{r}}{%
\node[inner sep=0pt,right=1ex of ciimage.south east,anchor=north west,rotate=90]%
{\raggedleft\color{black!50}\parbox{\the\ydiffl}{\raggedright{}#3}};%
}{%
\ifthenelse{\equal{#1}{l}}{%
\node[inner sep=0pt,right=1ex of ciimage.south west,anchor=south west,rotate=90]%
{\raggedleft\color{black!50}\parbox{\the\ydiffl}{\raggedright{}#3}};%
}{%
\node[inner sep=0pt,below=1ex of ciimage.south west,anchor=north west]%
{\raggedleft\color{black!50}\parbox{\the\xdiffl}{\raggedright{}#3}};%
}
}
\end{tikzpicture}
}


%% --------------------

%\usepackage[excludeor]{everyhook}
%\PushPreHook{par}{\setbox0=\lastbox\llap{MUH}}\box0}

%\vspace*{\stretch{1}

%\setbox0=\lastbox \llap{\textbullet\enskip}\box0}

\setlength{\parskip}{\fill}

\newcommand\noskips{\setlength{\parskip}{1ex}}
\newcommand\doskips{\setlength{\parskip}{\fill}}

\newcommand\xx{\par\vspace*{\stretch{1}}\par}
\newcommand\xxs{\par\vspace*{2ex}\par}
\newcommand\tuple[1]{\langle #1 \rangle}
\newcommand\code[1]{{\sf \footnotesize #1}}
\newcommand\ex[1]{\uline{Example:} \ifdefined \presentationonly \pause \fi
  \ifdefined\showexamples#1\xspace\else{\uline{\hspace*{2cm}}}\fi}

\newcommand\ceil[1]{\lceil #1 \rceil}


\AtBeginSection[]
{
   \begin{frame}
       \frametitle{Outline}
       \tableofcontents[currentsection]
   \end{frame}
}



\pgfdeclarelayer{edgelayer}
\pgfdeclarelayer{nodelayer}
\pgfsetlayers{edgelayer,nodelayer,main}

\tikzstyle{none}=[inner sep=0pt]
\tikzstyle{rn}=[circle,fill=Red,draw=Black,line width=0.8 pt]
\tikzstyle{gn}=[circle,fill=Lime,draw=Black,line width=0.8 pt]
\tikzstyle{yn}=[circle,fill=Yellow,draw=Black,line width=0.8 pt]
\tikzstyle{empty}=[circle,fill=White,draw=Black]
\tikzstyle{bw} = [rectangle, draw, fill=blue!20, 
    text width=4em, text centered, rounded corners, minimum height=2em]
    
    \newcommand{\CcNote}[1]{% longname
	This work is licensed under the \textit{Creative Commons #1 3.0 License}.%
}
\newcommand{\CcImageBy}[1]{%
	\includegraphics[scale=#1]{creative_commons/cc_by_30.pdf}%
}
\newcommand{\CcImageSa}[1]{%
	\includegraphics[scale=#1]{creative_commons/cc_sa_30.pdf}%
}
\newcommand{\CcImageNc}[1]{%
	\includegraphics[scale=#1]{creative_commons/cc_nc_30.pdf}%
}
\newcommand{\CcGroupBySa}[2]{% zoom, gap
	\CcImageBy{#1}\hspace*{#2}\CcImageNc{#1}\hspace*{#2}\CcImageSa{#1}%
}
\newcommand{\CcLongnameByNcSa}{Attribution-NonCommercial-ShareAlike}


\newenvironment{changemargin}[1]{% 
  \begin{list}{}{% 
    \setlength{\topsep}{0pt}% 
    \setlength{\leftmargin}{#1}% 
    \setlength{\rightmargin}{1em}
    \setlength{\listparindent}{\parindent}% 
    \setlength{\itemindent}{\parindent}% 
    \setlength{\parsep}{\parskip}% 
  }% 
  \item[]}{\end{list}} 




\title{Lecture 16 --- Recursive Functions }

\author{J. Zarnett\\
\texttt{jzarnett@uwaterloo.ca}}
\institute{Department of Electrical and Computer Engineering \\
  University of Waterloo}
\date{\today}

\begin{document}

\begin{frame}
  \titlepage
  
 \end{frame}

\begin{frame}
\frametitle{Recursion}

A function that includes a call to itself is said to be \alert{recursive}.

At first glance, this might seem like a really strange idea.\\
\quad Why would a function call itself?

Recall that we said earlier that a common engineering strategy is to break a big problem down into some smaller problems.

Recursion is useful if we are breaking down a problem so that the subproblems are smaller versions of the same problem.

\end{frame}

\begin{frame}
\frametitle{Recursion Example: Factorial}

You may have learned in math class about the factorial.\\
\quad Commonly written $n!$ in mathematical notation for a number $n$.

The factorial of a non-negative integer $n$ is defined as the product of all positive integers less than or equal to $n$.

Examples: \\
\quad $1! = 1$\\
\quad $2! = 2 \times 1 = 2$\\
\quad $3! = 3 \times 2 \times 1 = 6$\\
\quad $4! = 4 \times 3 \times 2 \times 1 = 24$\\
\quad $5! = 5 \times 4 \times 3 \times 2 \times 1 = 120$

\end{frame}

\begin{frame}
\frametitle{Recursion Example: Factorial}

Observe that $5!$ could be rewritten as $5 \times 4!$.

The problem of $5!$ can be broken down into two subproblems:

\begin{enumerate}
	\item $4!$
	\item $5 \times$ the result of (1).
\end{enumerate}

The subproblem $4!$ is a smaller version of the problem we're working on ($5!$), so this problem is a good candidate for recursion.

\end{frame}

\begin{frame}[fragile]
\frametitle{Recursion Example: Factorial}
Suppose now you are going to implement the factorial function.

\begin{verbatim}
static int factorial ( int n )
{
    return n * factorial( n - 1 );
}
\end{verbatim}

There is a problem with this implementation. 

\end{frame}

\begin{frame}
\frametitle{The Factorial Problem}

What happens when the expression n - 1 becomes zero or negative?

This loop will continue forever...\\
\quad Except in practice this will be stopped by an error.

Running this program produces an error called a \alert{Stack Overflow}.

In a later lecture we will discuss what a stack is and how it works.\\
\quad For now, a simplified view of the problem.


\end{frame}

\begin{frame}
\frametitle{Stack Overflow}

When \texttt{Main()} calls \texttt{factorial()}, the computer needs to keep track of where it was in \texttt{Main()} at the time that it went to the other function.

It puts that information in a designated memory area called the stack.

Each time \texttt{factorial()} calls itself, more information is added to the stack to keep track of where it was in \texttt{factorial()}.

If we do this too many times, the stack gets ``full'' (exceeds available memory for it) and this results in a stack overflow error.

\end{frame}

\begin{frame}[fragile]
\frametitle{Factorial Problem}
\begin{verbatim}
static int factorial ( int n )
{
    return n * factorial( n - 1 );
}
\end{verbatim}

The above implementation lacks a \alert{stopping condition}.

The function will keep calling itself until a stack overflow occurs.

\end{frame}

\begin{frame}
\frametitle{Factorial Problem}

Even if program execution did not terminate abnormally as a result of a stack overflow, there's another problem, mathematically.\\
\quad $5! = 5 \times 4 \times 3 \times 2 \times 1$

When we reach 1, we no longer multiply by the next smallest integer.

Our stopping case is therefore when \texttt{n} is 1.

The way the code is written, when \texttt{n} is 1 the evaluation goes on and the expression to the right of \texttt{return} is \texttt{1 * factorial( 0 )}. 

\end{frame}

\begin{frame}[fragile]
\frametitle{Factorial Fixed}
\begin{verbatim}
static int factorial ( int n )
{
    if ( n == 1 )
    {
        return 1;
    }
    return n * factorial( n - 1 );
}
\end{verbatim}

The revised implementation has the stopping condition of \texttt{n} equals 1.

The function will keep calling itself until \texttt{n} is 1.\\
\quad When that happens, 1 is returned.

\end{frame}

\begin{frame}
\frametitle{Recursion}
To make proper use of recursion, we need:
\begin{enumerate}
	\item One or more cases in which the function calls itself; and
	\item One or more cases in which the function does not call itself.
\end{enumerate}

Point (2) is called the stopping case or base case.

As we saw, without a properly defined stopping case, recursion will result in a stack overflow.

\end{frame}

\begin{frame}
\frametitle{Recursion versus Iteration}

Recursion can be a difficult or confusing topic. Is it strictly necessary?

Any task that can be accomplished with recursion can be done without using recursion, such as using a loop (iteratively).

As a small note on efficiency, a recursively written function may run slower than an iteratively written one. Recursion has two advantages:

\begin{enumerate}
	\item Write less code; and \\
	\item The computer tracks state on the stack; iteratively, we must keep track of the state ourselves.
\end{enumerate}

\end{frame}

\begin{frame}[fragile]
\frametitle{Factorial Iteratively}
Here's the factorial function implemented iteratively:

\begin{verbatim}
static int factorial ( int n )
{
    int product = 1;
    for ( int i = n; i > 1; i-- )
    {
        product *= i;
    }
    return product;
}
\end{verbatim}

The iterative implementation uses a \texttt{for} loop, but it could have been written using a \texttt{while} loop.

\end{frame}

% Note: Acknowledge Absolute C++ Book for this in the notes
\begin{frame}
\frametitle{Thinking Recursively}
When designing a recursive function, there are three important criteria to consider:

\begin{enumerate}
	\item There is no infinite recursion;\\
	\quad (A chain of recursive calls eventually reaches a stopping case)
	\item Each stopping case returns the correct value for that case; and
	\item The final returned value is correct if the recursive call(s) returns the correct value(s).
\end{enumerate}

This is an example of \textit{mathematical induction}.\\
If you have not yet learned about it, ignore this for now; you will see it next term in ECE~103 (Discrete Mathematics).

\end{frame}

\begin{frame}
\frametitle{Applying Recursive Thinking}
Consider another mathematical problem, exponentiation: $x^{y}$.

If we wrote a function signature for exponentiation:\\ \texttt{int pow ( int x, int y )}

Is this problem a good candidate for recursive solution?

Mathematically, $a^{b} = a \times a^{b-1}$.

So, yes, this is the kind of problem that lends itself well to recursion.


\end{frame}

\begin{frame}[fragile]
\frametitle{Applying Recursive Thinking}
Now, let's write an implementation for \texttt{pow()}.

\begin{verbatim}
// Precondition: y >= 0
// Postcondition: returns x to the power of y
static int pow ( int x, int y )
{
    if ( y == 0 )
    {
        return 1;
    }
    return pow( x, y - 1 ) * x;
}
\end{verbatim}

\end{frame}

\begin{frame}
\frametitle{Applying the Test}
Let's examine the design criteria and see if this is going to work.

\textit{1. There is no infinite recursion.}\\
\quad The second argument to \texttt{pow(x, y)} is decreased by 1 in each call,\\ \quad so eventually we must get to \texttt{pow(x, 0)}, a stopping case.\\
\quad \quad (As long as the precondition is not violated.)

Thus, there is no infinite recursion; criterion 1 is satisfied.

\end{frame}

\begin{frame}
\frametitle{Applying the Test}

\textit{2. Each stopping case returns the correct value for that case.}\\
\quad Yes. $x^{0} = 1$ is mathematically correct.

\textit{3. The final returned value is correct if the recursive call returns the correct value.}\\
\quad \texttt{pow( x, y - 1 ) * x} follows the rule that $a^{b} = a^{b-1} \times a$.

Criteria 2 and 3 are satisfied.

Having checked those things, we can now be satisfied that the implementation of \texttt{pow()} is correct. 

\end{frame}

\begin{frame}[fragile]
\frametitle{Exponentiation Iteratively}
Here's the exponential function implemented iteratively:

\begin{verbatim}
static int pow ( int x, int y )
{
    int result = 1;
    for ( int i = 0; i < y; i++ )
    {
        result *= x;
    }
    return result;
}
\end{verbatim}

Like the factorial function, the iterative implementation uses a \texttt{for} loop, but it could have been written using a \texttt{while} loop.

\end{frame}

\end{document}

