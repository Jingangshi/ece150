
\documentclass[letterpaper,hide notes,xcolor={table,svgnames},pdftex]{beamer}
\def\showexamples{t}


%\usepackage[svgnames]{xcolor}

%% Demo talk
%\documentclass[letterpaper,notes=show]{beamer}

\usecolortheme{crane}%seahorse crane
\setbeamertemplate{navigation symbols}{}

\usetheme{MyPittsburgh}
%\usetheme{Frankfurt}

%\usepackage{tipa}

\usepackage{hyperref}
\usepackage{graphicx,xspace}
\usepackage[normalem]{ulem}

\newcommand\SF[1]{$\bigstar$\footnote{SF: #1}}

\usepackage{paratype}
\renewcommand*\familydefault{\sfdefault} %% Only if the base font of the document is to be sans serif
\usepackage[zerostyle=c]{newtxtt}
\usepackage[T1]{fontenc}

\newcounter{tmpnumSlide}
\newcounter{tmpnumNote}

\usepackage{xcolor}
\usepackage{tabu}
\definecolor{light-gray}{gray}{0.75}
\taburulecolor{light-gray}

% old question code
%\newcommand\question[1]{{$\bigstar$ \small \onlySlide{2}{#1}}}
% \newcommand\nquestion[1]{\ifdefined \presentationonly \textcircled{?} \fi \note{\par{\Large \textbf{?}} #1}}
% \newcommand\nanswer[1]{\note{\par{\Large \textbf{A}} #1}}


 \newcommand\mnote[1]{%
   \addtocounter{tmpnumSlide}{1}
   \ifdefined\showcues {~\tiny\fbox{\arabic{tmpnumSlide}}}\fi
   \note{\setlength{\parskip}{1ex}\addtocounter{tmpnumNote}{1}\textbf{\Large \arabic{tmpnumNote}:} {#1\par}}}

\newcommand\mmnote[1]{\note{\setlength{\parskip}{1ex}#1\par}}

%\newcommand\mnote[2][]{\ifdefined\handoutwithnotes {~\tiny\fbox{#1}}\fi
% \note{\setlength{\parskip}{1ex}\textbf{\Large #1:} #2\par}}

%\newcommand\mnote[2][]{{\tiny\fbox{#1}} \note{\setlength{\parskip}{1ex}\textbf{\Large #1:} #2\par}}

\newcommand\mquestion[2]{{~\color{red}\fbox{?}}\note{\setlength{\parskip}{1ex}\par{\Large \textbf{?}} #1} \note{\setlength{\parskip}{1ex}\par{\Large \textbf{A}} #2\par}\ifdefined \presentationonly \pause \fi}

\newcommand\blackboard[1]{%
\ifdefined   \showblackboard
  {#1}
  \else {\begin{center} \fbox{\colorbox{blue!30}{%
         \begin{minipage}{.95\linewidth}%
           \hspace{\stretch{1}} Some space intentionally left blank; done at the blackboard.%
         \end{minipage}}}\end{center}}%
         \fi%
}



%\newcommand\q{\tikz \node[thick,color=black,shape=circle]{?};}
%\newcommand\q{\ifdefined \presentationonly \textcircled{?} \fi}

\usepackage{listings}
\lstset{%
  keywordstyle=\bfseries,
  aboveskip=15pt,
  belowskip=15pt,
  captionpos=b,
  identifierstyle=\ttfamily,
  escapeinside={(*@}{@*)},
  stringstyle=\ttfamiliy,
  frame=lines,
  numbers=left, basicstyle=\scriptsize, numberstyle=\tiny, stepnumber=0, numbersep=2pt}

\usepackage{siunitx}
\newcommand\sius[1]{\num[group-separator = {,}]{#1}\si{\micro\second}}
\newcommand\sims[1]{\num[group-separator = {,}]{#1}\si{\milli\second}}
\newcommand\sins[1]{\num[group-separator = {,}]{#1}\si{\nano\second}}
\sisetup{group-separator = {,}, group-digits = true}

%% -------------------- tikz --------------------
\usepackage{tikz}
\usetikzlibrary{positioning}
\usetikzlibrary{arrows,backgrounds,automata,decorations.shapes,decorations.pathmorphing,decorations.markings,decorations.text}

\tikzstyle{place}=[circle,draw=blue!50,fill=blue!20,thick, inner sep=0pt,minimum size=6mm]
\tikzstyle{transition}=[rectangle,draw=black!50,fill=black!20,thick, inner sep=0pt,minimum size=4mm]

\tikzstyle{block}=[rectangle,draw=black, thick, inner sep=5pt]
\tikzstyle{bullet}=[circle,draw=black, fill=black, thin, inner sep=2pt]

\tikzstyle{pre}=[<-,shorten <=1pt,>=stealth',semithick]
\tikzstyle{post}=[->,shorten >=1pt,>=stealth',semithick]
\tikzstyle{bi}=[<->,shorten >=1pt,shorten <=1pt, >=stealth',semithick]

\tikzstyle{mut}=[-,>=stealth',semithick]

\tikzstyle{treereset}=[dashed,->, shorten >=1pt,>=stealth',thin]

\usepackage{ifmtarg}
\usepackage{xifthen}
\makeatletter
% new counter to now which frame it is within the sequence
\newcounter{multiframecounter}
% initialize buffer for previously used frame title
\gdef\lastframetitle{\textit{undefined}}
% new environment for a multi-frame
\newenvironment{multiframe}[1][]{%
\ifthenelse{\isempty{#1}}{%
% if no frame title was set via optional parameter,
% only increase sequence counter by 1
\addtocounter{multiframecounter}{1}%
}{%
% new frame title has been provided, thus
% reset sequence counter to 1 and buffer frame title for later use
\setcounter{multiframecounter}{1}%
\gdef\lastframetitle{#1}%
}%
% start conventional frame environment and
% automatically set frame title followed by sequence counter
\begin{frame}%
\frametitle{\lastframetitle~{\normalfont(\arabic{multiframecounter})}}%
}{%
\end{frame}%
}
\makeatother

\makeatletter
\newdimen\tu@tmpa%
\newdimen\ydiffl%
\newdimen\xdiffl%
\newcommand\ydiff[2]{%
    \coordinate (tmpnamea) at (#1);%
    \coordinate (tmpnameb) at (#2);%
    \pgfextracty{\tu@tmpa}{\pgfpointanchor{tmpnamea}{center}}%
    \pgfextracty{\ydiffl}{\pgfpointanchor{tmpnameb}{center}}%
    \advance\ydiffl by -\tu@tmpa%
}
\newcommand\xdiff[2]{%
    \coordinate (tmpnamea) at (#1);%
    \coordinate (tmpnameb) at (#2);%
    \pgfextractx{\tu@tmpa}{\pgfpointanchor{tmpnamea}{center}}%
    \pgfextractx{\xdiffl}{\pgfpointanchor{tmpnameb}{center}}%
    \advance\xdiffl by -\tu@tmpa%
}
\makeatother
\newcommand{\copyrightbox}[3][r]{%
\begin{tikzpicture}%
\node[inner sep=0pt,minimum size=2em](ciimage){#2};
\usefont{OT1}{phv}{n}{n}\fontsize{4}{4}\selectfont
\ydiff{ciimage.south}{ciimage.north}
\xdiff{ciimage.west}{ciimage.east}
\ifthenelse{\equal{#1}{r}}{%
\node[inner sep=0pt,right=1ex of ciimage.south east,anchor=north west,rotate=90]%
{\raggedleft\color{black!50}\parbox{\the\ydiffl}{\raggedright{}#3}};%
}{%
\ifthenelse{\equal{#1}{l}}{%
\node[inner sep=0pt,right=1ex of ciimage.south west,anchor=south west,rotate=90]%
{\raggedleft\color{black!50}\parbox{\the\ydiffl}{\raggedright{}#3}};%
}{%
\node[inner sep=0pt,below=1ex of ciimage.south west,anchor=north west]%
{\raggedleft\color{black!50}\parbox{\the\xdiffl}{\raggedright{}#3}};%
}
}
\end{tikzpicture}
}


%% --------------------

%\usepackage[excludeor]{everyhook}
%\PushPreHook{par}{\setbox0=\lastbox\llap{MUH}}\box0}

%\vspace*{\stretch{1}

%\setbox0=\lastbox \llap{\textbullet\enskip}\box0}

\setlength{\parskip}{\fill}

\newcommand\noskips{\setlength{\parskip}{1ex}}
\newcommand\doskips{\setlength{\parskip}{\fill}}

\newcommand\xx{\par\vspace*{\stretch{1}}\par}
\newcommand\xxs{\par\vspace*{2ex}\par}
\newcommand\tuple[1]{\langle #1 \rangle}
\newcommand\code[1]{{\sf \footnotesize #1}}
\newcommand\ex[1]{\uline{Example:} \ifdefined \presentationonly \pause \fi
  \ifdefined\showexamples#1\xspace\else{\uline{\hspace*{2cm}}}\fi}

\newcommand\ceil[1]{\lceil #1 \rceil}


\AtBeginSection[]
{
   \begin{frame}
       \frametitle{Outline}
       \tableofcontents[currentsection]
   \end{frame}
}



\pgfdeclarelayer{edgelayer}
\pgfdeclarelayer{nodelayer}
\pgfsetlayers{edgelayer,nodelayer,main}

\tikzstyle{none}=[inner sep=0pt]
\tikzstyle{rn}=[circle,fill=Red,draw=Black,line width=0.8 pt]
\tikzstyle{gn}=[circle,fill=Lime,draw=Black,line width=0.8 pt]
\tikzstyle{yn}=[circle,fill=Yellow,draw=Black,line width=0.8 pt]
\tikzstyle{empty}=[circle,fill=White,draw=Black]
\tikzstyle{bw} = [rectangle, draw, fill=blue!20, 
    text width=4em, text centered, rounded corners, minimum height=2em]
    
    \newcommand{\CcNote}[1]{% longname
	This work is licensed under the \textit{Creative Commons #1 3.0 License}.%
}
\newcommand{\CcImageBy}[1]{%
	\includegraphics[scale=#1]{creative_commons/cc_by_30.pdf}%
}
\newcommand{\CcImageSa}[1]{%
	\includegraphics[scale=#1]{creative_commons/cc_sa_30.pdf}%
}
\newcommand{\CcImageNc}[1]{%
	\includegraphics[scale=#1]{creative_commons/cc_nc_30.pdf}%
}
\newcommand{\CcGroupBySa}[2]{% zoom, gap
	\CcImageBy{#1}\hspace*{#2}\CcImageNc{#1}\hspace*{#2}\CcImageSa{#1}%
}
\newcommand{\CcLongnameByNcSa}{Attribution-NonCommercial-ShareAlike}


\newenvironment{changemargin}[1]{% 
  \begin{list}{}{% 
    \setlength{\topsep}{0pt}% 
    \setlength{\leftmargin}{#1}% 
    \setlength{\rightmargin}{1em}
    \setlength{\listparindent}{\parindent}% 
    \setlength{\itemindent}{\parindent}% 
    \setlength{\parsep}{\parskip}% 
  }% 
  \item[]}{\end{list}} 




\title{Lecture 12 --- More About Arrays }

\author{J. Zarnett\\
\texttt{jzarnett@uwaterloo.ca}}
\institute{Department of Electrical and Computer Engineering \\
  University of Waterloo}
\date{\today}

\begin{document}

\begin{frame}
  \titlepage
  
  \begin{center}
  \small{Acknowledgments: W.D. Bishop}
  \end{center}
 \end{frame}
 

\part{The \texttt{foreach} Loop}
\begin{frame}\partpage\end{frame}


\begin{frame}
\frametitle{Iterating Over an Array}

Last time, we iterated over entries of an array using the \texttt{for} loop.

C\# provides another construct to iterate over all the entries in an array: the \texttt{foreach} statement.

The \texttt{foreach} statement is just some slightly more convenient syntax; anything we do with it we could do with a regular \texttt{for} loop.

Let's assume we have an integer array (\texttt{int[]} grades) defined.

\end{frame}

\begin{frame}[fragile]
\frametitle{Use of the \texttt{foreach} Statement}

The syntax is: foreach (\textit{type identifier} in \textit{arrayIdentifier})

\begin{verbatim}
int[] grades;

// Initialization of grades not shown

foreach ( int num in grades )
{
    // Loop Body
}
\end{verbatim}

The \textit{type} in the expression must be the same as the type of the array.

Let's look at an example with the loop body filled in.

\end{frame}


\begin{frame}[fragile]
\frametitle{Using the \texttt{foreach} Statement}
\begin{verbatim}
string[] months = { "January", "February", "March", 
        "April",  "May", "June", "July", "August",
        "September", "October", "November", "December"
         };

foreach ( string s in months )
{
    Console.WriteLine( s );
}
\end{verbatim}
\end{frame}


\begin{frame}[fragile]
\frametitle{Use of the \texttt{foreach} Statement}

Any \texttt{foreach} loop could also be replaced with a \texttt{for} loop.

Here's a \texttt{foreach} loop to print all the grades in the array \texttt{grades}.


\begin{verbatim}
foreach ( int num in grades )
{
    Console.WriteLine( num );
}
\end{verbatim}

Now, rewrite this using \texttt{for}:

\begin{verbatim}
for ( int index = 0; index < grades.Length; index++ )
{
    Console.WriteLine( grades[index] );
}
\end{verbatim}
\end{frame}

\begin{frame}[fragile]
\frametitle{Using the \texttt{foreach} Statement}
Rewriting the months example:

\begin{verbatim}
string[] months = {"January", "February", "March", "April", 
        "May", "June", "July", "August", "September", 
        "October", "November", "December" };

for ( int i = 0; i < months.Length; i++ )
{
    Console.WriteLine( months[i] );
}
\end{verbatim}
\end{frame}


\begin{frame}
\frametitle{Comparing \texttt{for} and \texttt{foreach}}

The \texttt{for} and \texttt{foreach} loops are often functionally equivalent.

The \texttt{break} and \texttt{continue} work as expected in \texttt{foreach}.

The \texttt{foreach} syntax is focused on iterating over all the entries of an array; the \texttt{for} loop is more flexible.

Recall that the \texttt{for} loop can have any stopping condition (and therefore behave like the \texttt{while} loop).

The \texttt{for} loop can also go backwards, skip every second item, etc.

\end{frame}




\part{The String Revisited}
\begin{frame}\partpage\end{frame}

\begin{frame}
\frametitle{The \texttt{string} Revisited}

We have already learned about a string type, but we haven't examined it in much detail.

The string was just a bunch of text, but there is much more to it than we might think at first glance...

\texttt{string svar1; // Creates uninitialized string}\\
\texttt{string svar2 = "Literal"; // Initialized.}\\
\texttt{string svar3 = ""; // Initialized to the empty string.}

The \texttt{string} is a complex type (there's a reason it wasn't introduced in the simple types alongside \texttt{int} and \texttt{double}).

\end{frame}

\begin{frame}
\frametitle{String Concatenation}

Strings can appear in expressions using the \texttt{+} operator. It does not add up the values, but instead performs \alert{concatenation}.

\texttt{string verb = "fore" + "see";}

This means the variable \texttt{verb} contains \texttt{"foresee"}.

The use of \texttt{+=} can also be used for concatenation:\\
\quad \texttt{verb += "n";} $\rightarrow$ \texttt{verb} is now \texttt{foreseen}.

Remember that for concatenation, like an arithmetic expression, it is necessary to assign the value somewhere.

\end{frame}

\begin{frame}
\frametitle{Strings are Complex}

It turns out that the string has a member variable \texttt{Length} that tells you the number of characters in the \texttt{string}.

This information, combined with the fact that a \texttt{string} is a bunch of text characters should lead you to the conclusion that...

The \texttt{string} is really an \alert{array of \texttt{char}s}.

\end{frame}

\begin{frame}
\frametitle{An Array of \texttt{char}}

This means we can access individual characters within a \texttt{string} using their index values (just as if they were entries of an array).

If the string is \texttt{string ex1 = "example"}, the \texttt{char} at \texttt{ex1[3]} is \texttt{'m'}.

We could use a \texttt{for} (or \texttt{foreach}) loop to iterate over all the characters of the \texttt{string} if we are looking for something specific.

\end{frame}

\begin{frame}[fragile]
\frametitle{Iterating Over the \texttt{string}}

\begin{verbatim}
string s = "Hello World!";

for ( int i = 0; i < s.Length; i++ )
{
    Console.WriteLine( s[i] );
}

for ( int j = s.Length -1; j >= 0; j-- )
{
    Console.Write( s[j] );
}
\end{verbatim}

\end{frame}

\begin{frame}
\frametitle{Mutating \texttt{string} Elements}

Characters within a \texttt{string} cannot be changed using index values.

This is different from an array of \texttt{int} where we could assign \texttt{array[7] = -98;}

The string type is \alert{immutable}. \texttt{string} variables cannot be changed.

Every time a string must be ``changed'', a new \texttt{string} must be created.

\end{frame}

\begin{frame}
\frametitle{Immutable \texttt{string} Variables}

If we have a statement \texttt{string text = name + suffix;} after this statement, there are 3 strings in memory: \texttt{text, name, suffix}.

That's not surprising, but consider this: \texttt{name += suffix;}\\
\quad There are still three strings stored in memory after this statement.\\
\quad \texttt{suffix}, the new value of \texttt{name}, and the old value of \texttt{name}.

The \texttt{+=} operation did two things: 
\begin{enumerate}
\item Created a new string and (the concatenation of \texttt{name} and \texttt{suffix}) and stored it in memory.
\item Changed the memory location \texttt{name} is associated with to the location of the new string.
\end{enumerate}

\end{frame}


\part{Multi-Dimensional Arrays}
\begin{frame}\partpage\end{frame}

\begin{frame}
\frametitle{Arrays of Arrays}

You may have wondered if we can have an array of any type, can we have an array of arrays? Yes!

A \alert{multi-dimensional array} is an array of array types.

The following statement declares and instantiates a multi-dimensional array of \texttt{int} named \texttt{day}: \texttt{int[][] day = new int [n][]};

Is this syntax confusing? Perhaps imagine it like this: \texttt{(int[])[]}

The type is \texttt{int[]} (in brackets) and when we declare an array of a type, write \texttt{[]} after the type.

Hence, we declare an array of type \texttt{int[]} (integer array).

\end{frame}

\begin{frame}
\frametitle{Arrays of Arrays}

\texttt{day[0]} is a reference to the first integer array\\
\texttt{day[1]} is a reference to the second integer array\\
\texttt{day[n-1]} is a reference to the nth integer array


The length of \texttt{day[0]} and \texttt{day[1]} may be different.

\end{frame}

\begin{frame}[fragile]
\frametitle{Setting up A Multi-Dimensional Array}

\begin{verbatim}
int[] daysInMonth = { 31, 28, 31, 30, 31, 30, 
                     31, 31, 30, 31, 30, 31 };

int[][] year = new int[12][];

for ( int month = 0; month < 12; ++month )
{
    year[month] = new int[ daysInMonth[month] ];
	
    for ( int day = 0; d < daysInMonth[month]; ++d )
    {
        year[month][day] = 1;
    }

}
\end{verbatim}

\end{frame}

\begin{frame}[fragile]
\frametitle{Setting up A Multi-Dimensional Array}

Now let's print out this calendar.

\begin{verbatim}
for ( int month = 0; month < 12; ++month )
{
    for ( int day = 0; d < daysInMonth[month]; ++d )
    {
        Console.Write( year[month][day] );
        Console.Write( " " );
    }
    Console.WriteLine( "" );
}
\end{verbatim}

\end{frame}

\begin{frame}[fragile]
\frametitle{Further Initialization of Multi-Dimensional Arrays}
It is possible to initialize a multi-dimensional array when it is declared.

For example, the following code defines a multi-dimensional array of characters named \texttt{myChars}:

\begin{verbatim}
char[][] myChars = {
          new char[] { 'B', 'i', 'l', 'l' },
          new char[] { 'D', 'a', 'v', 'e' },
          new char[] { 'G', 'e', 'o', 'r', 'g', 'e' }
          };
\end{verbatim}

\end{frame}

\begin{frame}
\frametitle{Arrays of Arrays of Arrays...}

The multi-dimensional array examples we have shown so far are all ``two dimensional''.

We could have more, such as \texttt{int[][][] coordinates;} to describe x, y, and z co-ordinates.

In C\# the term for arrays of these kinds is ``\alert{jagged} arrays''.

C\# also supports \alert{rectangular} arrays, which we'll come back to later. 

\end{frame}

\end{document}

