
\documentclass[letterpaper,hide notes,xcolor={table,svgnames},pdftex]{beamer}
\def\showexamples{t}


%\usepackage[svgnames]{xcolor}

%% Demo talk
%\documentclass[letterpaper,notes=show]{beamer}

\usecolortheme{crane}%seahorse crane
\setbeamertemplate{navigation symbols}{}

\usetheme{MyPittsburgh}
%\usetheme{Frankfurt}

%\usepackage{tipa}

\usepackage{hyperref}
\usepackage{graphicx,xspace}
\usepackage[normalem]{ulem}

\newcommand\SF[1]{$\bigstar$\footnote{SF: #1}}

\usepackage{paratype}
\renewcommand*\familydefault{\sfdefault} %% Only if the base font of the document is to be sans serif
\usepackage[zerostyle=c]{newtxtt}
\usepackage[T1]{fontenc}

\newcounter{tmpnumSlide}
\newcounter{tmpnumNote}

\usepackage{xcolor}
\usepackage{tabu}
\definecolor{light-gray}{gray}{0.75}
\taburulecolor{light-gray}

% old question code
%\newcommand\question[1]{{$\bigstar$ \small \onlySlide{2}{#1}}}
% \newcommand\nquestion[1]{\ifdefined \presentationonly \textcircled{?} \fi \note{\par{\Large \textbf{?}} #1}}
% \newcommand\nanswer[1]{\note{\par{\Large \textbf{A}} #1}}


 \newcommand\mnote[1]{%
   \addtocounter{tmpnumSlide}{1}
   \ifdefined\showcues {~\tiny\fbox{\arabic{tmpnumSlide}}}\fi
   \note{\setlength{\parskip}{1ex}\addtocounter{tmpnumNote}{1}\textbf{\Large \arabic{tmpnumNote}:} {#1\par}}}

\newcommand\mmnote[1]{\note{\setlength{\parskip}{1ex}#1\par}}

%\newcommand\mnote[2][]{\ifdefined\handoutwithnotes {~\tiny\fbox{#1}}\fi
% \note{\setlength{\parskip}{1ex}\textbf{\Large #1:} #2\par}}

%\newcommand\mnote[2][]{{\tiny\fbox{#1}} \note{\setlength{\parskip}{1ex}\textbf{\Large #1:} #2\par}}

\newcommand\mquestion[2]{{~\color{red}\fbox{?}}\note{\setlength{\parskip}{1ex}\par{\Large \textbf{?}} #1} \note{\setlength{\parskip}{1ex}\par{\Large \textbf{A}} #2\par}\ifdefined \presentationonly \pause \fi}

\newcommand\blackboard[1]{%
\ifdefined   \showblackboard
  {#1}
  \else {\begin{center} \fbox{\colorbox{blue!30}{%
         \begin{minipage}{.95\linewidth}%
           \hspace{\stretch{1}} Some space intentionally left blank; done at the blackboard.%
         \end{minipage}}}\end{center}}%
         \fi%
}



%\newcommand\q{\tikz \node[thick,color=black,shape=circle]{?};}
%\newcommand\q{\ifdefined \presentationonly \textcircled{?} \fi}

\usepackage{listings}
\lstset{%
  keywordstyle=\bfseries,
  aboveskip=15pt,
  belowskip=15pt,
  captionpos=b,
  identifierstyle=\ttfamily,
  escapeinside={(*@}{@*)},
  stringstyle=\ttfamiliy,
  frame=lines,
  numbers=left, basicstyle=\scriptsize, numberstyle=\tiny, stepnumber=0, numbersep=2pt}

\usepackage{siunitx}
\newcommand\sius[1]{\num[group-separator = {,}]{#1}\si{\micro\second}}
\newcommand\sims[1]{\num[group-separator = {,}]{#1}\si{\milli\second}}
\newcommand\sins[1]{\num[group-separator = {,}]{#1}\si{\nano\second}}
\sisetup{group-separator = {,}, group-digits = true}

%% -------------------- tikz --------------------
\usepackage{tikz}
\usetikzlibrary{positioning}
\usetikzlibrary{arrows,backgrounds,automata,decorations.shapes,decorations.pathmorphing,decorations.markings,decorations.text}

\tikzstyle{place}=[circle,draw=blue!50,fill=blue!20,thick, inner sep=0pt,minimum size=6mm]
\tikzstyle{transition}=[rectangle,draw=black!50,fill=black!20,thick, inner sep=0pt,minimum size=4mm]

\tikzstyle{block}=[rectangle,draw=black, thick, inner sep=5pt]
\tikzstyle{bullet}=[circle,draw=black, fill=black, thin, inner sep=2pt]

\tikzstyle{pre}=[<-,shorten <=1pt,>=stealth',semithick]
\tikzstyle{post}=[->,shorten >=1pt,>=stealth',semithick]
\tikzstyle{bi}=[<->,shorten >=1pt,shorten <=1pt, >=stealth',semithick]

\tikzstyle{mut}=[-,>=stealth',semithick]

\tikzstyle{treereset}=[dashed,->, shorten >=1pt,>=stealth',thin]

\usepackage{ifmtarg}
\usepackage{xifthen}
\makeatletter
% new counter to now which frame it is within the sequence
\newcounter{multiframecounter}
% initialize buffer for previously used frame title
\gdef\lastframetitle{\textit{undefined}}
% new environment for a multi-frame
\newenvironment{multiframe}[1][]{%
\ifthenelse{\isempty{#1}}{%
% if no frame title was set via optional parameter,
% only increase sequence counter by 1
\addtocounter{multiframecounter}{1}%
}{%
% new frame title has been provided, thus
% reset sequence counter to 1 and buffer frame title for later use
\setcounter{multiframecounter}{1}%
\gdef\lastframetitle{#1}%
}%
% start conventional frame environment and
% automatically set frame title followed by sequence counter
\begin{frame}%
\frametitle{\lastframetitle~{\normalfont(\arabic{multiframecounter})}}%
}{%
\end{frame}%
}
\makeatother

\makeatletter
\newdimen\tu@tmpa%
\newdimen\ydiffl%
\newdimen\xdiffl%
\newcommand\ydiff[2]{%
    \coordinate (tmpnamea) at (#1);%
    \coordinate (tmpnameb) at (#2);%
    \pgfextracty{\tu@tmpa}{\pgfpointanchor{tmpnamea}{center}}%
    \pgfextracty{\ydiffl}{\pgfpointanchor{tmpnameb}{center}}%
    \advance\ydiffl by -\tu@tmpa%
}
\newcommand\xdiff[2]{%
    \coordinate (tmpnamea) at (#1);%
    \coordinate (tmpnameb) at (#2);%
    \pgfextractx{\tu@tmpa}{\pgfpointanchor{tmpnamea}{center}}%
    \pgfextractx{\xdiffl}{\pgfpointanchor{tmpnameb}{center}}%
    \advance\xdiffl by -\tu@tmpa%
}
\makeatother
\newcommand{\copyrightbox}[3][r]{%
\begin{tikzpicture}%
\node[inner sep=0pt,minimum size=2em](ciimage){#2};
\usefont{OT1}{phv}{n}{n}\fontsize{4}{4}\selectfont
\ydiff{ciimage.south}{ciimage.north}
\xdiff{ciimage.west}{ciimage.east}
\ifthenelse{\equal{#1}{r}}{%
\node[inner sep=0pt,right=1ex of ciimage.south east,anchor=north west,rotate=90]%
{\raggedleft\color{black!50}\parbox{\the\ydiffl}{\raggedright{}#3}};%
}{%
\ifthenelse{\equal{#1}{l}}{%
\node[inner sep=0pt,right=1ex of ciimage.south west,anchor=south west,rotate=90]%
{\raggedleft\color{black!50}\parbox{\the\ydiffl}{\raggedright{}#3}};%
}{%
\node[inner sep=0pt,below=1ex of ciimage.south west,anchor=north west]%
{\raggedleft\color{black!50}\parbox{\the\xdiffl}{\raggedright{}#3}};%
}
}
\end{tikzpicture}
}


%% --------------------

%\usepackage[excludeor]{everyhook}
%\PushPreHook{par}{\setbox0=\lastbox\llap{MUH}}\box0}

%\vspace*{\stretch{1}

%\setbox0=\lastbox \llap{\textbullet\enskip}\box0}

\setlength{\parskip}{\fill}

\newcommand\noskips{\setlength{\parskip}{1ex}}
\newcommand\doskips{\setlength{\parskip}{\fill}}

\newcommand\xx{\par\vspace*{\stretch{1}}\par}
\newcommand\xxs{\par\vspace*{2ex}\par}
\newcommand\tuple[1]{\langle #1 \rangle}
\newcommand\code[1]{{\sf \footnotesize #1}}
\newcommand\ex[1]{\uline{Example:} \ifdefined \presentationonly \pause \fi
  \ifdefined\showexamples#1\xspace\else{\uline{\hspace*{2cm}}}\fi}

\newcommand\ceil[1]{\lceil #1 \rceil}


\AtBeginSection[]
{
   \begin{frame}
       \frametitle{Outline}
       \tableofcontents[currentsection]
   \end{frame}
}



\pgfdeclarelayer{edgelayer}
\pgfdeclarelayer{nodelayer}
\pgfsetlayers{edgelayer,nodelayer,main}

\tikzstyle{none}=[inner sep=0pt]
\tikzstyle{rn}=[circle,fill=Red,draw=Black,line width=0.8 pt]
\tikzstyle{gn}=[circle,fill=Lime,draw=Black,line width=0.8 pt]
\tikzstyle{yn}=[circle,fill=Yellow,draw=Black,line width=0.8 pt]
\tikzstyle{empty}=[circle,fill=White,draw=Black]
\tikzstyle{bw} = [rectangle, draw, fill=blue!20, 
    text width=4em, text centered, rounded corners, minimum height=2em]
    
    \newcommand{\CcNote}[1]{% longname
	This work is licensed under the \textit{Creative Commons #1 3.0 License}.%
}
\newcommand{\CcImageBy}[1]{%
	\includegraphics[scale=#1]{creative_commons/cc_by_30.pdf}%
}
\newcommand{\CcImageSa}[1]{%
	\includegraphics[scale=#1]{creative_commons/cc_sa_30.pdf}%
}
\newcommand{\CcImageNc}[1]{%
	\includegraphics[scale=#1]{creative_commons/cc_nc_30.pdf}%
}
\newcommand{\CcGroupBySa}[2]{% zoom, gap
	\CcImageBy{#1}\hspace*{#2}\CcImageNc{#1}\hspace*{#2}\CcImageSa{#1}%
}
\newcommand{\CcLongnameByNcSa}{Attribution-NonCommercial-ShareAlike}


\newenvironment{changemargin}[1]{% 
  \begin{list}{}{% 
    \setlength{\topsep}{0pt}% 
    \setlength{\leftmargin}{#1}% 
    \setlength{\rightmargin}{1em}
    \setlength{\listparindent}{\parindent}% 
    \setlength{\itemindent}{\parindent}% 
    \setlength{\parsep}{\parskip}% 
  }% 
  \item[]}{\end{list}} 




\title{Lecture 33 --- Files \& Streams }

\author{J. Zarnett\\
\texttt{jzarnett@uwaterloo.ca}}
\institute{Department of Electrical and Computer Engineering \\
  University of Waterloo}
\date{\today}

\begin{document}

\begin{frame}
  \titlepage
  
  \begin{center}
  \small{Acknowledgments: W.D. Bishop}
  \end{center}
\end{frame}

\begin{frame}
\frametitle{Files and File Systems}

Files permanently store large amounts of data.

Operating systems implement a filesystem to manage files stored on a hard disk (or other types of storage media).

Filesystems typically support the following:
\begin{itemize}
\item Creating new files
\item Reading from a file
\item Writing to a file
\item Appending to existing files
\item Renaming existing files
\item Deleting existing files
\item Checking for the existence of a file
\end{itemize}

\end{frame}

\begin{frame}
\frametitle{Streams}

File I/O and console I/O are very similar.

We have used \texttt{cout} (an output stream) and \texttt{cin} (an input stream) extensively.

\alert{Stream}s are an abstraction for data flow that can be applied to both file I/O and console I/O.

A stream denotes the flow of data from a source to a destination.



\end{frame}

\begin{frame}
\frametitle{File I/O Fundamentals}

The file I/O classes are defined in the \texttt{fstream} library.

Output streams are created using \texttt{ofstream}.

Input streams are created using \texttt{ifstream} class.

The \texttt{ofstream} and \texttt{ifstream} classes provide sequential access to text files: each line must be written or read in sequence.

Files are assumed to store text data.

\end{frame}


\begin{frame}[fragile]
\frametitle{File Output Example}

This code outputs the text ``Hello World'' to a file named ``output.txt''.

{\scriptsize
\begin{verbatim}
ofstream outStream; 
string filename = "output.txt";
outStream.open( filename );

outStream << "Hello World" << endl;

outStream.close();

\end{verbatim}
}

\end{frame}



\begin{frame}[fragile]
\frametitle{File Input Example}

{\scriptsize
\begin{verbatim}
ifstream inStream;
inStream.open("input.txt");

int a;
int b;
int c;

inStream >> a >> b >> c;

cout << a << endl;
cout << b << endl;
cout << c << endl;

inStream.close();
 
\end{verbatim}
}

[In class demo: output of this program.]

\end{frame}


\begin{frame}
\frametitle{Inputting Formatted Data}

The input stream attempts to make sense of whatever it has been fed.

But perhaps it is not that simple and we will read in a line of text that we need to then interpret and split.

There are a couple of things to consider:
\begin{itemize}
    \item How will the data be used, stored, manipulated, etc.?
    \item What errors should be detected, corrected, ignored, etc.?
\end{itemize}

\end{frame}


\begin{frame}[fragile]
\frametitle{Reading Data Example}

Given the following class declaration:

\begin{verbatim}
class Employee
{
  private: 
    ulong identifier;
    string surname;
    string givenName;
    double salary;
};
\end{verbatim}

How would you build a program to read this data from a file and display the data on the console?

\end{frame}

\begin{frame}[fragile]
\frametitle{Step 1: File Format}
Before coding anything, consider the format of the input file.

Let's assume a valid input file resembles the following:

\begin{verbatim}
3 Phaneuf Dion 6500000
81 Kessel Phil 6000000
12 Connolly Tim 5500000
8 Komisarek Mike 5500000
24 Liles John-Michael 4550000
\end{verbatim}

This is an easy file to parse since we can use the space characters as our way of splitting a line into its component parts.


\end{frame}

\begin{frame}[fragile]
\frametitle{Step 2: Updating \texttt{Employee}}
Next step: to make importing of data efficient, we should add a constructor to \texttt{Employee}.

\begin{verbatim}
Employee::Employee( ulong id, string sname, 
                 string gname, double s )
{
    identifier = id;
    surname = sname;
    givenName = gname;
    salary = s;
}
\end{verbatim}

\end{frame}

\begin{frame}
\frametitle{Questions to Ponder}
How would you deal with input data that includes spaces?

You could use commas or another special character to delimit fields.\\
\quad ...but how would this change your parsing...?

What would you do if you needed to read a file containing a fixed number of records into an array of records?
\begin{itemize}
    \item Read the file twice: once to determine the number of records and once to store them?
    \item Store the number of records at the top of the input file?
    \item Set a maximum number of entries?
    \item Re-size the array when necessary?
    \item Use a collection?
\end{itemize}

\end{frame}

\begin{frame}[fragile]
\frametitle{Field Delimiter Example}

What happens when we try to split \texttt{line} below with a space as the delimiter?

\begin{verbatim}
string line = "A,B;C D;E,F";
\end{verbatim}

We get two strings as output:\\
\quad "A,B;C"\\
\quad "D;E,F"

\end{frame}

\begin{frame}[fragile]
\frametitle{Field Delimiter Example 2}

What happens when we try to split \texttt{line} below with a comma as a delimiter?

\begin{verbatim}
string line = "A,B;C D;E,F";
\end{verbatim}

We get three strings as output:\\
\quad "A"\\
\quad "B;C D;E"\\
\quad "F"

\end{frame}

\begin{frame}[fragile]
\frametitle{Field Delimiter Example 3}

What happens when we try to split \texttt{line} below with a semicolon as a delimiter?

\begin{verbatim}
string line = "A,B;C D;E,F";
\end{verbatim}

We get three strings as output:\\
\quad "A,B"\\
\quad "C D"\\
\quad "E,F"

\end{frame}

\begin{frame}[fragile]
\frametitle{Field Delimiter Example 4}

What happens when we try to split \texttt{line} below using all three??

\begin{verbatim}
string line = "A,B;C D;E,F";
\end{verbatim}

We get six strings as output:\\
\quad "A"\\
\quad "B"\\
\quad "C"\\
\quad "D"\\
\quad "E"\\
\quad "F"

\end{frame}

\begin{frame}
\frametitle{Terminology Review}

In our discussion of file I/O, we have introduced a number of important database terms:
\begin{enumerate}
\item A \alert{file} is form of long-term, persistent data storage
\item A \alert{database file} is a collection of records
\item A \alert{record} is a collection of fields separated by delimiters
\item A \alert{field} is a group of bits (or characters)
\item A \alert{delimiter} is a special character used to separate fields
\end{enumerate}

A detailed discussion of databases is beyond the scope of this course.

\end{frame}

\end{document}

