
\documentclass[letterpaper,hide notes,xcolor={table,svgnames},pdftex]{beamer}
\def\showexamples{t}


%\usepackage[svgnames]{xcolor}

%% Demo talk
%\documentclass[letterpaper,notes=show]{beamer}

\usecolortheme{crane}%seahorse crane
\setbeamertemplate{navigation symbols}{}

\usetheme{MyPittsburgh}
%\usetheme{Frankfurt}

%\usepackage{tipa}

\usepackage{hyperref}
\usepackage{graphicx,xspace}
\usepackage[normalem]{ulem}

\newcommand\SF[1]{$\bigstar$\footnote{SF: #1}}

\usepackage{paratype}
\renewcommand*\familydefault{\sfdefault} %% Only if the base font of the document is to be sans serif
\usepackage[zerostyle=c]{newtxtt}
\usepackage[T1]{fontenc}

\newcounter{tmpnumSlide}
\newcounter{tmpnumNote}

\usepackage{xcolor}
\usepackage{tabu}
\definecolor{light-gray}{gray}{0.75}
\taburulecolor{light-gray}

% old question code
%\newcommand\question[1]{{$\bigstar$ \small \onlySlide{2}{#1}}}
% \newcommand\nquestion[1]{\ifdefined \presentationonly \textcircled{?} \fi \note{\par{\Large \textbf{?}} #1}}
% \newcommand\nanswer[1]{\note{\par{\Large \textbf{A}} #1}}


 \newcommand\mnote[1]{%
   \addtocounter{tmpnumSlide}{1}
   \ifdefined\showcues {~\tiny\fbox{\arabic{tmpnumSlide}}}\fi
   \note{\setlength{\parskip}{1ex}\addtocounter{tmpnumNote}{1}\textbf{\Large \arabic{tmpnumNote}:} {#1\par}}}

\newcommand\mmnote[1]{\note{\setlength{\parskip}{1ex}#1\par}}

%\newcommand\mnote[2][]{\ifdefined\handoutwithnotes {~\tiny\fbox{#1}}\fi
% \note{\setlength{\parskip}{1ex}\textbf{\Large #1:} #2\par}}

%\newcommand\mnote[2][]{{\tiny\fbox{#1}} \note{\setlength{\parskip}{1ex}\textbf{\Large #1:} #2\par}}

\newcommand\mquestion[2]{{~\color{red}\fbox{?}}\note{\setlength{\parskip}{1ex}\par{\Large \textbf{?}} #1} \note{\setlength{\parskip}{1ex}\par{\Large \textbf{A}} #2\par}\ifdefined \presentationonly \pause \fi}

\newcommand\blackboard[1]{%
\ifdefined   \showblackboard
  {#1}
  \else {\begin{center} \fbox{\colorbox{blue!30}{%
         \begin{minipage}{.95\linewidth}%
           \hspace{\stretch{1}} Some space intentionally left blank; done at the blackboard.%
         \end{minipage}}}\end{center}}%
         \fi%
}



%\newcommand\q{\tikz \node[thick,color=black,shape=circle]{?};}
%\newcommand\q{\ifdefined \presentationonly \textcircled{?} \fi}

\usepackage{listings}
\lstset{%
  keywordstyle=\bfseries,
  aboveskip=15pt,
  belowskip=15pt,
  captionpos=b,
  identifierstyle=\ttfamily,
  escapeinside={(*@}{@*)},
  stringstyle=\ttfamiliy,
  frame=lines,
  numbers=left, basicstyle=\scriptsize, numberstyle=\tiny, stepnumber=0, numbersep=2pt}

\usepackage{siunitx}
\newcommand\sius[1]{\num[group-separator = {,}]{#1}\si{\micro\second}}
\newcommand\sims[1]{\num[group-separator = {,}]{#1}\si{\milli\second}}
\newcommand\sins[1]{\num[group-separator = {,}]{#1}\si{\nano\second}}
\sisetup{group-separator = {,}, group-digits = true}

%% -------------------- tikz --------------------
\usepackage{tikz}
\usetikzlibrary{positioning}
\usetikzlibrary{arrows,backgrounds,automata,decorations.shapes,decorations.pathmorphing,decorations.markings,decorations.text}

\tikzstyle{place}=[circle,draw=blue!50,fill=blue!20,thick, inner sep=0pt,minimum size=6mm]
\tikzstyle{transition}=[rectangle,draw=black!50,fill=black!20,thick, inner sep=0pt,minimum size=4mm]

\tikzstyle{block}=[rectangle,draw=black, thick, inner sep=5pt]
\tikzstyle{bullet}=[circle,draw=black, fill=black, thin, inner sep=2pt]

\tikzstyle{pre}=[<-,shorten <=1pt,>=stealth',semithick]
\tikzstyle{post}=[->,shorten >=1pt,>=stealth',semithick]
\tikzstyle{bi}=[<->,shorten >=1pt,shorten <=1pt, >=stealth',semithick]

\tikzstyle{mut}=[-,>=stealth',semithick]

\tikzstyle{treereset}=[dashed,->, shorten >=1pt,>=stealth',thin]

\usepackage{ifmtarg}
\usepackage{xifthen}
\makeatletter
% new counter to now which frame it is within the sequence
\newcounter{multiframecounter}
% initialize buffer for previously used frame title
\gdef\lastframetitle{\textit{undefined}}
% new environment for a multi-frame
\newenvironment{multiframe}[1][]{%
\ifthenelse{\isempty{#1}}{%
% if no frame title was set via optional parameter,
% only increase sequence counter by 1
\addtocounter{multiframecounter}{1}%
}{%
% new frame title has been provided, thus
% reset sequence counter to 1 and buffer frame title for later use
\setcounter{multiframecounter}{1}%
\gdef\lastframetitle{#1}%
}%
% start conventional frame environment and
% automatically set frame title followed by sequence counter
\begin{frame}%
\frametitle{\lastframetitle~{\normalfont(\arabic{multiframecounter})}}%
}{%
\end{frame}%
}
\makeatother

\makeatletter
\newdimen\tu@tmpa%
\newdimen\ydiffl%
\newdimen\xdiffl%
\newcommand\ydiff[2]{%
    \coordinate (tmpnamea) at (#1);%
    \coordinate (tmpnameb) at (#2);%
    \pgfextracty{\tu@tmpa}{\pgfpointanchor{tmpnamea}{center}}%
    \pgfextracty{\ydiffl}{\pgfpointanchor{tmpnameb}{center}}%
    \advance\ydiffl by -\tu@tmpa%
}
\newcommand\xdiff[2]{%
    \coordinate (tmpnamea) at (#1);%
    \coordinate (tmpnameb) at (#2);%
    \pgfextractx{\tu@tmpa}{\pgfpointanchor{tmpnamea}{center}}%
    \pgfextractx{\xdiffl}{\pgfpointanchor{tmpnameb}{center}}%
    \advance\xdiffl by -\tu@tmpa%
}
\makeatother
\newcommand{\copyrightbox}[3][r]{%
\begin{tikzpicture}%
\node[inner sep=0pt,minimum size=2em](ciimage){#2};
\usefont{OT1}{phv}{n}{n}\fontsize{4}{4}\selectfont
\ydiff{ciimage.south}{ciimage.north}
\xdiff{ciimage.west}{ciimage.east}
\ifthenelse{\equal{#1}{r}}{%
\node[inner sep=0pt,right=1ex of ciimage.south east,anchor=north west,rotate=90]%
{\raggedleft\color{black!50}\parbox{\the\ydiffl}{\raggedright{}#3}};%
}{%
\ifthenelse{\equal{#1}{l}}{%
\node[inner sep=0pt,right=1ex of ciimage.south west,anchor=south west,rotate=90]%
{\raggedleft\color{black!50}\parbox{\the\ydiffl}{\raggedright{}#3}};%
}{%
\node[inner sep=0pt,below=1ex of ciimage.south west,anchor=north west]%
{\raggedleft\color{black!50}\parbox{\the\xdiffl}{\raggedright{}#3}};%
}
}
\end{tikzpicture}
}


%% --------------------

%\usepackage[excludeor]{everyhook}
%\PushPreHook{par}{\setbox0=\lastbox\llap{MUH}}\box0}

%\vspace*{\stretch{1}

%\setbox0=\lastbox \llap{\textbullet\enskip}\box0}

\setlength{\parskip}{\fill}

\newcommand\noskips{\setlength{\parskip}{1ex}}
\newcommand\doskips{\setlength{\parskip}{\fill}}

\newcommand\xx{\par\vspace*{\stretch{1}}\par}
\newcommand\xxs{\par\vspace*{2ex}\par}
\newcommand\tuple[1]{\langle #1 \rangle}
\newcommand\code[1]{{\sf \footnotesize #1}}
\newcommand\ex[1]{\uline{Example:} \ifdefined \presentationonly \pause \fi
  \ifdefined\showexamples#1\xspace\else{\uline{\hspace*{2cm}}}\fi}

\newcommand\ceil[1]{\lceil #1 \rceil}


\AtBeginSection[]
{
   \begin{frame}
       \frametitle{Outline}
       \tableofcontents[currentsection]
   \end{frame}
}



\pgfdeclarelayer{edgelayer}
\pgfdeclarelayer{nodelayer}
\pgfsetlayers{edgelayer,nodelayer,main}

\tikzstyle{none}=[inner sep=0pt]
\tikzstyle{rn}=[circle,fill=Red,draw=Black,line width=0.8 pt]
\tikzstyle{gn}=[circle,fill=Lime,draw=Black,line width=0.8 pt]
\tikzstyle{yn}=[circle,fill=Yellow,draw=Black,line width=0.8 pt]
\tikzstyle{empty}=[circle,fill=White,draw=Black]
\tikzstyle{bw} = [rectangle, draw, fill=blue!20, 
    text width=4em, text centered, rounded corners, minimum height=2em]
    
    \newcommand{\CcNote}[1]{% longname
	This work is licensed under the \textit{Creative Commons #1 3.0 License}.%
}
\newcommand{\CcImageBy}[1]{%
	\includegraphics[scale=#1]{creative_commons/cc_by_30.pdf}%
}
\newcommand{\CcImageSa}[1]{%
	\includegraphics[scale=#1]{creative_commons/cc_sa_30.pdf}%
}
\newcommand{\CcImageNc}[1]{%
	\includegraphics[scale=#1]{creative_commons/cc_nc_30.pdf}%
}
\newcommand{\CcGroupBySa}[2]{% zoom, gap
	\CcImageBy{#1}\hspace*{#2}\CcImageNc{#1}\hspace*{#2}\CcImageSa{#1}%
}
\newcommand{\CcLongnameByNcSa}{Attribution-NonCommercial-ShareAlike}


\newenvironment{changemargin}[1]{% 
  \begin{list}{}{% 
    \setlength{\topsep}{0pt}% 
    \setlength{\leftmargin}{#1}% 
    \setlength{\rightmargin}{1em}
    \setlength{\listparindent}{\parindent}% 
    \setlength{\itemindent}{\parindent}% 
    \setlength{\parsep}{\parskip}% 
  }% 
  \item[]}{\end{list}} 




\title{Lecture 21 --- Encapsulation \& Constructors}

\author{J. Zarnett\\
\texttt{jzarnett@uwaterloo.ca}}
\institute{Department of Electrical and Computer Engineering \\
  University of Waterloo}
\date{\today}

\begin{document}

\begin{frame}
  \titlepage
  
  \begin{center}
  \small{Acknowledgments: W.D. Bishop}
  \end{center}
\end{frame}

\part{Encapsulation}
\begin{frame}\partpage\end{frame}

\begin{frame}
\frametitle{Encapsulation}
\alert{Encapsulation} is another principle of OOP: making the fields (variables) of a class inaccessible from outside the class.

Then access to the data is granted through the use of externally accessible methods.

Encapsulation can also mean bundling the data with methods that operate on that data.

To provide access to just the data, we use \alert{getter}s and \alert{setter}s.

\end{frame}


\begin{frame}
\frametitle{Getter and Setter}

These provide a public mechanism for accessing and manipulating the private data members of a class.

They have an identifier that normally matches the name of the private data member to be accessed / manipulated.

They are a convenient mechanism for implementing object-oriented methods for accessors and mutators.

\alert{Accessors} (getters) read the value of a private data member.\\
\alert{Mutators} (setters) write the value of a private data member.

\end{frame}

\begin{frame}[fragile]
\frametitle{Get-and-Set Methods}

\begin{verbatim}
private:
  double price;

public:
  double getPrice() {
    return price;
  }

  void setPrice( double new_price ) {
    price = new_price;
  }
\end{verbatim}

\end{frame}




\begin{frame}
\frametitle{Encapsulation: \texttt{public} and \texttt{private}}

With the \texttt{class} the keyword \texttt{public} appeared but we did not discuss what it means or why it is there.

Then, in the last example, a new keyword appeared: \texttt{private}.

These are two \alert{access modifiers} or \alert{visibility modifiers}.

Access modifiers are a technique to restrict where a variable may be read/written from, or where a method can be called from.

\end{frame}

\begin{frame}
\frametitle{Encapsulation: \texttt{public} and \texttt{private}}

The two access modifiers we've seen so far are \texttt{public} and \texttt{private}.

Marking a method or variable \texttt{public} makes it accessible from everywhere in the program. \\
\quad Any function may call a \texttt{public} function, for example.

Denoting a method or variable as \texttt{private} makes it accessible only from within that object type.

It's a compile time error to try to read a \texttt{private} variable from outside its declaring class.

As a rule of thumb, methods of a class tend to be \texttt{public} and the variables tend to be \texttt{private}.

\end{frame}

\begin{frame}
\frametitle{Motivation for Encapsulation}
It may seem silly to have private methods in the system - why bother?\\
\quad Especially because we can easily change private into public.

The real purpose of the public/private distinction is to encourage good programming practice and maintainability.

In a large, complex system, any method or variable that is public will be relied upon by other classes.

In the future, it will be expensive or impossible to change.

\end{frame}

\begin{frame}
\frametitle{Motivation for Encapsulation}
Consider Microsoft's problem: maintaining program compatibility.

Any time they want to change something in Windows, they have to consider very carefully what the impact is on all the programs.

Microsoft can change private things, but not those that are public.

They know that people won't buy the new version of Windows if their old programs no longer function.

For more on this, see \url{http://blogs.msdn.com/b/oldnewthing/archive/2003/12/24/45779.aspx}
\end{frame}

\begin{frame}
\frametitle{Accessors and Mutators for All!}
A very common implementation choice is to make all member variables private and create accessor and mutator methods for each.

This provides some minor benefits over just making all variables public, but not all that much.

Accessor and mutator methods should exist only where there is a reason to create them.

\end{frame}


\begin{frame}
\frametitle{Advantages of Encapsulation}
Proper use of encapsulation results in better-designed software that:\\
\quad Maximizes \alert{cohesion} -- related elements being kept together; and\\
\quad Minimizes \alert{coupling} -- dependency of one part on another.

Much like re-using the \texttt{Date struct} we saw earlier, objects can and should be re-used to build larger objects.

If data is properly encapsulated, the implementation may be changed without having an impact on other parts of the program.

Change is inevitable in software, so we should plan for it.

\end{frame}

\part{Constructors and Destructors}
\begin{frame}\partpage\end{frame}

\begin{frame}
\frametitle{Constructors}

A \alert{constructor} is a special kind of method that is used to instantiate and initialize objects.

Every class has a default constructor that initializes each data member of a class to a default value. It has no parameters.

The default constructor for a class is accessed using the class name followed by round brackets.

For example, if you define a class named ABC, the default constructor for ABC is \texttt{ABC( )}.

\end{frame}

\begin{frame}
\frametitle{Constructors}

If you create a constructor, regardless of the number of parameters, for a class, the default constructor no longer exists.

Thus, if you want to have a no-parameters constructor in addition to any number of others, you must explicitly define it.

A constructor can have as many parameters as required.

Multiple constructors for a class may exist.\\
\quad This is an example of function overloading.

\end{frame}

\begin{frame}[fragile]
\frametitle{Coordinates Constructor Example}

{\scriptsize
\begin{verbatim}
class Coordinates
{
    float x;
    float y;
    float z;

    public Coordinates( )            // Constructor 1
    {
        x = 0.0f; 
        y = 0.0f;
        z = 0.0f;
    }

    public Coordinates( float a, float b, float c )    // Constructor 2
    {
        x = a;
        y = b;
        z = c;
    }    
}
\end{verbatim}
}
\end{frame}

\begin{frame}
\frametitle{Coordinates Constructor Comments}

In this example, an instance of a Coordinates object can be created using two different constructors:\\
\quad \texttt{public Coordinates( )}\\
\quad \texttt{public Coordinates( float a, float b, float c )}

A constructor does not have an explicitly-specified return type and always has the same name as its class.

The first constructor is used to create an instance of a Coordinates object that is initialized to (0, 0, 0).

The second constructor is used to create an instance of a Coordinates object initialized to (a, b, c).

\end{frame}

\begin{frame}
\frametitle{Using Constructors}

To utilize the class declaration on the previous slide, you need to create an object using the constructor.

To declare a variable, c1, using the first constructor:\\
\quad \texttt{Coordinates c1 = new Coordinates( );}

This creates a reference type variable named \texttt{c1}.

This variable refers to a new object of type Coordinates that has been initialized using the first constructor.

Note the use of the \texttt{new} keyword to create a new object (like an array).

\end{frame}

\begin{frame}
\frametitle{Using Constructors}
To declare a variable, c2, initialized to (10.0, -5.5, 7.5):\\
\texttt{Coordinates c2 = new Coordinates( 10.0f, -5.5f, 7.5f );}

This creates a reference named \texttt{c2} that refers to a new object of type Coordinates that has been initialized using the second constructor.

When the object is created, the coordinates x, y, and z are set to values of 10.0, -5.5, and 7.5 respectively.

\end{frame}

\begin{frame}[fragile]
\frametitle{Copy Coordinates}
Suppose we want to make a copy of the \texttt{Coordinates c2} object.

Recall that objects are reference types. \\
\quad \texttt{Coordinates c3 = c2;} is not correct; \\
\quad That statement will create another reference to the same instance.

Correct solution:\\
\quad \texttt{Coordinates c3 = new Coordinates( c2.X, c2.Y, c2.Z );}


Or by manually assigning each variable:
\begin{verbatim}
Coordinates c3 = new Coordinates( );
c3.X = c2.X;
c3.Y = c2.Y;
c3.Z = c2.Z;
\end{verbatim}

\end{frame}

\begin{frame}[fragile]
\frametitle{Copy Constructors}
We can make creation of a copy easier by adding a \alert{copy constructor}.

\begin{verbatim}
public Coordinates( Coordinates toCopy )
{
    x = toCopy.x;
    y = toCopy.y;
    z = toCopy.z;
}
\end{verbatim}

This makes the syntax of creating a copy much simpler:\\
\quad \texttt{Coordinates c3 = new Coordinates( c2 );}

Use of a copy constructor also allows better encapsulation; no need to know the internals of the class to make a correct copy.

\end{frame}

\begin{frame}[fragile]
\frametitle{Copy Constructor Alternative}
Another strategy for making a copy of an object is a static method.

\begin{verbatim}
public static Coordinates MakeCopy( Coordinates toCopy )
{
    Coordinates result = new Coordinates( );
    result.X = toCopy.X;
    result.Y = toCopy.Y;
    result.Z = toCopy.Z;

    return result;
}
\end{verbatim}


Then, to make a copy of \texttt{Coordinates c1}, call:\\
\quad \texttt{Coordinates c2 = Coordinates.MakeCopy( c1 );}

\end{frame}

\begin{frame}
\frametitle{About Destructors}

\alert{Destructors} are unnecessary in C\#; C\# uses \alert{garbage collection}.\\
\quad We will learn about this subject soon.

In a system with garbage collection, old objects that are no longer needed are automatically cleaned up.

In other languages, a destructor does the opposite of the constructor: it frees up resources that the object has accumulated over its lifetime.

This is needed in languages with manual memory management (such as C++) and the definition is included here only for completeness.

\end{frame}


\end{document}

